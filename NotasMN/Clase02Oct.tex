%===========================================
\documentclass[12pt]{article}
%===========================================
\usepackage[utf8]{inputenc}
%\usepackage[margin=2.5in]{geometry}
\usepackage{amsmath,amssymb,amsthm,amsfonts}
\usepackage{hyperref}
\usepackage[spanish]{babel}
\decimalpoint
\usepackage{fancyhdr}
\usepackage{titlesec}
\usepackage{listings}
\usepackage{graphicx,graphics}
\usepackage{multicol}
\usepackage{multirow}
\usepackage{color}
\usepackage{float} 
\usepackage{subfig}
\usepackage[figuresright]{rotating}
\usepackage{enumerate}
\usepackage{anysize} 
\usepackage{url}
\usepackage{imakeidx}
\usepackage[left=0.5in, right=0.5in, top=1in, bottom=1in]{geometry}
% Opcional: para incluir gráficos con control de tamaño
\usepackage{float}

\hyphenation{mo-de-ra-da-men-te}
\addto\captionsspanish{\renewcommand{\figurename}{Figura}}

%===========================================
% Ajustes de Sweave
%\usepackage{Sweave}
%===========================================
\title{Notas sobre Métodos Numéricos con R}
\author{
Carlos E. Martínez-Rodríguez \\
Universidad Autónoma de la Ciudad de México \\
Academia de Matemáticas \\
\texttt{carlos.martinez@uacm.edu.mx}
}
\date{Agosto 2025}
\date{}
%===========================================
\newtheorem{Criterio}{Criterio}%[section]
\newtheorem{Sup}{Supuesto}[section]
\newtheorem{Note}{Nota}[section]
\newtheorem{Ejem}{Ejemplo}[section]
\newtheorem{Ejer}{Ejercicio}[section]
\newtheorem{Prop}{Proposici\'on}[section]
\newtheorem{Def}{Definici\'on}[section]
\newtheorem{Teo}{Teorema}[section]
\newtheorem{Algthm}{Algoritmo}[section]
\newtheorem{Sol}{Soluci\'on}[section]
\newtheorem{Ses}{Sesi\'on}[section]
%===========================================
\begin{document}
\maketitle
\tableofcontents

\section{Gauss-Jacobi}
\begin{equation*}
A =
\begin{pmatrix}
10 & -1 & 2 & 0\\
-1 & 11 & -1 & 3\\
2 & -1 & 10 & -1\\
0 & 3 & -1 & 8
\end{pmatrix}, \qquad
\mathbf{b} =
\begin{pmatrix}
6\\ 25\\ -11\\ 15
\end{pmatrix}, \qquad
\mathbf{x}^{\*} = (1,2,-1,1)^{T}.
\end{equation*}

Iteraci\'on 
\begin{eqnarray*}
x_1^{(k+1)} &=& \frac{6 + x_2^{(k)} - 2x_3^{(k)}}{10},\\
x_2^{(k+1)} &=& \frac{25 + x_1^{(k)} + x_3^{(k)} - 3x_4^{(k)}}{11},\\
x_3^{(k+1)} &=& \frac{-11 - 2x_1^{(k)} + x_2^{(k)} + x_4^{(k)}}{10},\\
x_4^{(k+1)} &=& \frac{15 - 3x_2^{(k)} + x_3^{(k)}}{8}.
\end{eqnarray*}


Tomamos $\mathbf{x}^{(0)}=(0,0,0,0)^\top$ y actualizamos cada $x_i^{(k+1)}$ usa sólo valores del paso $k$.

\paragraph{Iteración 1:}
\begin{eqnarray*}
x_1^{(1)}&=&\frac{6 + 0 - 20}{10}=0.6, \\
x_2^{(1)}&=&\frac{25 + 0+ 0 - 3\cdot0}{11}=\frac{25}{11}\approx 2.272727,\\
x_3^{(1)}&=&\frac{-11 - 2\cdot0 + 0 + 0}{10}=-1.1, \\
x_4^{(1)}&=&\frac{15 - 3\cdot0 + 0}{8}=\frac{15}{8}=1.875.
\end{eqnarray*}

\paragraph{Iteración 2} Con $\mathbf{x}^{(1)}=(0.6,\,2.272727,\,-1.1,\,1.875)$,
\begin{eqnarray*}
x_1^{(2)}&=&\frac{6 + 2.272727 - 2(-1.1)}{10}=\frac{10.472727}{10}=1.047273,\\
x_2^{(2)}&=&\frac{25 + 0.6 + (-1.1) - 3(1.875)}{11}=\frac{18.875}{11}\approx 1.715909,\\
x_3^{(2)}&=&\frac{-11 - 2(0.6) + 2.272727 + 1.875}{10}=\frac{-8.052273}{10}\approx -0.805227,\\
x_4^{(2)}&=&\frac{15 - 3(2.272727) + (-1.1)}{8}=\frac{7.081819}{8}\approx 0.885227.
\end{eqnarray*}

\paragraph{Iteración 3} Con $\mathbf{x}^{(2)}=(1.047273,\,1.715909,\,-0.805227,\,0.885227)$,
\begin{eqnarray*}
x_1^{(3)}&=&\frac{6 + 1.715909 - 2(-0.805227)}{10}
=\frac{9.326363}{10}=0.932636,\\
x_2^{(3)}&=&\frac{25 + 1.047273 + (-0.805227) - 3(0.885227)}{11}
=\frac{22.586365}{11}=2.053306,\\
x_3^{(3)}&=&\frac{-11 - 2(1.047273) + 1.715909 + 0.885227}{10}
=\frac{-10.493410}{10}=-1.049341,\\
x_4^{(3)}&=&\frac{15 - 3(1.715909) + (-0.805227)}{8}
=\frac{9.047046}{8}=1.130881.
\end{eqnarray*}

\paragraph{Iteración 4} Con $\mathbf{x}^{(3)}=(0.932636,\,2.053306,\,-1.049341,\,1.130881)$,
\begin{eqnarray*}
x_1^{(4)}&=&\frac{6 + 2.053306 - 2(-1.049341)}{10}
=\frac{10.151988}{10}=1.015199,\\
x_2^{(4)}&=&\frac{25 + 0.932636 + (-1.049341) - 3(1.130881)}{11}
=\frac{21.490652}{11}=1.953696,\\
x_3^{(4)}&=&\frac{-11 - 2(0.932636) + 2.053306 + 1.130881}{10}
=\frac{-9.681085}{10}=-0.968109,\\
x_4^{(4)}&=&\frac{15 - 3(2.053306) + (-1.049341)}{8}
=\frac{7.790741}{8}=0.973843.
\end{eqnarray*}

\paragraph{Iteración 5} Con $\mathbf{x}^{(4)}=(1.015199,\,1.953696,\,-0.968109,\,0.973843)$,
\begin{eqnarray*}
x_1^{(5)}&=&\frac{6 + 1.953696 - 2(-0.968109)}{10}
=\frac{9.889914}{10}=0.988991,\\
x_2^{(5)}&=&\frac{25 + 1.015199 + (-0.968109) - 3(0.973843)}{11}
=\frac{22.125561}{11}=2.011415,\\
x_3^{(5)}&=&\frac{-11 - 2(1.015199) + 1.953696 + 0.973843}{10}
=\frac{-10.102859}{10}=-1.010286,\\
x_4^{(5)}&=&\frac{15 - 3(1.953696) + (-0.968109)}{8}
=\frac{8.170803}{8}=1.021351.
\end{eqnarray*}

Tabla de resultados

\begin{center}
\begin{tabular}{c|rrrr}
%%\toprule
k & $x_1^{(k)}$ & $x_2^{(k)}$ & $x_3^{(k)}$ & $x_4^{(k)}$\\
%%\midrule
0 & 0.0000 & 0.0000 & 0.0000 & 0.0000 \\
1 & 0.6000 & 2.2727 & -1.1000 & 1.8750 \\
2 & 1.0473 & 1.7159 & -0.8052 & 0.8852 \\
3 & 0.9326 & 2.0533 & -1.0493 & 1.1309 \\
4 & 1.0152 & 1.9537 & -0.9681 & 0.9738 \\
5 & 0.9890 & 2.0114 & -1.0103 & 1.0214 \\
%%\bottomrule
\end{tabular}
\end{center}

Tabla de errores

\begin{center}
\begin{tabular}{c|cc}
%\toprule
k & $\|x^{(k)}-x^{*}|_\infty$ & $\|x^{(k)}-x^{(k-1)}\|_\infty$\\
%\midrule
0 & 2.0000 & -- \\
1 & 0.8750 & 2.2727 \\
2 & 0.2841 & 0.9898 \\
3 & 0.1309 & 0.3374 \\
4 & 0.0463 & 0.1570 \\
5 & 0.0214 & 0.0577 \\
%\bottomrule
\end{tabular}
\end{center}

Ejercicios:instrucciones: usa $x^{(0)}=\mathbf{0}$ y detén cuando $\|x^{(k+1)}-x^{(k)}\|_\infty<10^{-3}$. 
Comprueba dominancia diagonal, escribe las fórmulas de Jacobi e itera con tabla.

\begin{Ejer}
\begin{equation*}
\begin{pmatrix}
5 & 1\\
2 & 6
\end{pmatrix}
\begin{pmatrix}x_1\\x_2\end{pmatrix}
=
\begin{pmatrix}3\\-10\end{pmatrix}.
\end{equation*}
\end{Ejer}


\begin{Ejer}
\begin{equation*}
\underbrace{\begin{pmatrix}
9 & -1 & 0 & 0\\
-1 & 12 & -2 & 1\\
0 & -1 & 11 & -2\\
0 & 2 & -1 & 10
\end{pmatrix}}_{A_{4\times4}}\,
\underbrace{\begin{pmatrix}x_1\\x_2\\x_3\\x_4\end{pmatrix}}_{x}
=
\underbrace{\begin{pmatrix}7\\26\\-15\\15\end{pmatrix}}_{b},
\end{equation*}
\end{Ejer}


\begin{Ejer}

\begin{equation*}
\underbrace{\begin{pmatrix}
10 & -1 & 2 & 0 & 0\\
-1 & 12 & -2 & 3 & 0\\
2 & -1 & 11 & -2 & 1\\
0 & 3 & -1 & 10 & -2\\
0 & 0 & 2 & -1 & 9
\end{pmatrix}}_{A_{5\times5}}\,
\underbrace{\begin{pmatrix}x_1\\x_2\\x_3\\x_4\\x_5\end{pmatrix}}_{x}
=
\underbrace{\begin{pmatrix}15\\-17\\26\\-7\\13\end{pmatrix}}_{b},
\end{equation*}

\end{Ejer}

\section{Gauss-Seidel}

Considerar

\[
A =
\begin{pmatrix}
9 & -1 & 0 & 2\\
-1 & 10 & -2 & 1\\
0 & -1 & 8 & -1\\
2 & 1 & -1 & 11
\end{pmatrix},\quad
\mathbf{b} =
\begin{pmatrix}
10\\ 8\\ 6\\ 13
\end{pmatrix},\quad
\mathbf{x}^{*}=(1,1,1,1)^\top.
\]
$A$ es diagonalmente dominante por renglones, por lo que Gauss--Seidel converge.

Iteraci\'on general
Con ecuaciones por rengl\'on:
\[
\begin{aligned}
9x_1 - x_2 + 2x_4 &= 10,&
\Rightarrow\;& x_1 \;=\; \frac{10 + x_2 - 2x_4}{9},\\
- x_1 + 10x_2 - 2x_3 + x_4 &= 8,&
\Rightarrow\;& x_2 \;=\; \frac{8 + x_1 + 2x_3 - x_4}{10},\\
- x_2 + 8x_3 - x_4 &= 6,&
\Rightarrow\;& x_3 \;=\; \frac{6 + x_2 + x_4}{8},\\
2x_1 + x_2 - x_3 + 11x_4 &= 13,&
\Rightarrow\;& x_4 \;=\; \frac{13 - 2x_1 - x_2 + x_3}{11}.
\end{aligned}
\]
\textbf{Regla de GS:} cada $x_i^{(k+1)}$ usa los \emph{valores m\'as recientes disponibles} dentro del mismo paso $k+1$.


Partimos de $\mathbf{x}^{(0)}=(0,0,0,0)^\top$.

\paragraph{Iteraci\'on 1}
\[
\begin{aligned}
x_1^{(1)}&=\frac{10+0-20}{9}=1.111111,\\
x_2^{(1)}&=\frac{8+\underline{x_1^{(1)}}+20-0}{10}
=\frac{8+1.111111}{10}=0.911111,\\
x_3^{(1)}&=\frac{6+\underline{x_2^{(1)}}+0}{8}
=\frac{6+0.911111}{8}=0.863889,\\
x_4^{(1)}&=\frac{13-2\underline{x_1^{(1)}}-\underline{x_2^{(1)}}+\underline{x_3^{(1)}}}{11}
=\frac{13-2(1.111111)-0.911111+0.863889}{11}=0.975505.
\end{aligned}
\]

\paragraph{Iteraci\'on 2}
\[
\begin{aligned}
x_1^{(2)}&=\frac{10+x_2^{(1)}-2x_4^{(1)}}{9}
=\frac{10+0.911111-2(0.975505)}{9}=0.995567,\\
x_2^{(2)}&=\frac{8+\underline{x_1^{(2)}}+2x_3^{(1)}-x_4^{(1)}}{10}
=\frac{8+0.995567+2(0.863889)-0.975505}{10}=0.974784,\\
x_3^{(2)}&=\frac{6+\underline{x_2^{(2)}}+x_4^{(1)}}{8}
=\frac{6+0.974784+0.975505}{8}=0.993786,\\
x_4^{(2)}&=\frac{13-2\underline{x_1^{(2)}}-\underline{x_2^{(2)}}+\underline{x_3^{(2)}}}{11}
=\frac{13-2(0.995567)-0.974784+0.993786}{11}=1.002534.
\end{aligned}
\]

\paragraph{Iteraci\'on 3}
\[
\begin{aligned}
x_1^{(3)}&=\frac{10+0.974784-2(1.002534)}{9}=0.996635,\\
x_2^{(3)}&=\frac{8+\underline{0.996635}+2(0.993786)-1.002534}{10}=0.998167,\\
x_3^{(3)}&=\frac{6+\underline{0.998167}+1.002534}{8}=1.000088,\\
x_4^{(3)}&=\frac{13-2\underline{0.996635}-\underline{0.998167}+\underline{1.000088}}{11}=1.000786.
\end{aligned}
\]

\paragraph{Iteraci\'on 4}
\[
\begin{aligned}
x_1^{(4)}&=\frac{10+0.998167-2(1.000786)}{9}=0.999622,\\
x_2^{(4)}&=\frac{8+\underline{0.999622}+2(1.000088)-1.000786}{10}=0.999901,\\
x_3^{(4)}&=\frac{6+\underline{0.999901}+1.000786}{8}=1.000086,\\
x_4^{(4)}&=\frac{13-2\underline{0.999622}-\underline{0.999901}+\underline{1.000086}}{11}=1.000086.
\end{aligned}
\]

\paragraph{Iteraci\'on 5}
\[
\begin{aligned}
x_1^{(5)}&=\frac{10+0.999901-2(1.000086)}{9}=0.999970,\\
x_2^{(5)}&=\frac{8+\underline{0.999970}+2(1.000086)-1.000086}{10}=1.000006,\\
x_3^{(5)}&=\frac{6+\underline{1.000006}+1.000086}{8}=1.000011,\\
x_4^{(5)}&=\frac{13-2\underline{0.999970}-\underline{1.000006}+\underline{1.000011}}{11}=1.000006.
\end{aligned}
\]

Tabla de resultados

\begin{center}
\begin{tabular}{c|rrrr}
%\toprule
k & $x_1^{(k)}$ & $x_2^{(k)}$ & $x_3^{(k)}$ & $x_4^{(k)}$\\
%\midrule
0 & 0.000000 & 0.000000 & 0.000000 & 0.000000 \\
1 & 1.111111 & 0.911111 & 0.863889 & 0.975505 \\
2 & 0.995567 & 0.974784 & 0.993786 & 1.002534 \\
3 & 0.996635 & 0.998167 & 1.000088 & 1.000786 \\
4 & 0.999622 & 0.999901 & 1.000086 & 1.000086 \\
5 & 0.999970 & 1.000006 & 1.000011 & 1.000006 \\
%\bottomrule
\end{tabular}
\end{center}

Tabla de errores

\begin{center}
\begin{tabular}{c|cc}
%\toprule
k & $\|x^{(k)}-x^{*}\|_\infty$ & $\|x^{(k)}-x^{(k-1)}\|_\infty$\\
%\midrule
0 & 1.000000 & -- \\
1 & 0.136111 & 1.111111 \\
2 & 0.025216 & 0.129897 \\
3 & 0.003365 & 0.023383 \\
4 & 0.000378 & 0.002986 \\
5 & 0.000030 & 0.000348 \\
%\bottomrule
\end{tabular}
\end{center}


Instrucciones: usa $x^{(0)}=\mathbf{0}$ y det\'en cuando $\|x^{(k+1)}-x^{(k)}\|_\infty<10^{-3}$. Escribe las f\'ormulas de GS e itera con tabla.


\begin{Ejer}
\[
\underbrace{\begin{pmatrix}
6 & -1\\
1 & 5
\end{pmatrix}}_{A_{2\times2}}
\begin{pmatrix}x_1\\x_2\end{pmatrix}
=
\underbrace{\begin{pmatrix}13\\-3\end{pmatrix}}_{b}.
\]

\end{Ejer}


\begin{Ejer}\[
\underbrace{\begin{pmatrix}
8 & 1 & -1 & 0\\
2 & 10 & -2 & 1\\
1 & -1 & 9 & -1\\
0 & 2 & -1 & 7
\end{pmatrix}}_{A_{4\times4}}
\begin{pmatrix}x_1\\x_2\\x_3\\x_4\end{pmatrix}
=
\underbrace{\begin{pmatrix}6\\-3\\20\\-9\end{pmatrix}}_{b}.
\]

\end{Ejer}


\begin{Ejer}
\[
\underbrace{\begin{pmatrix}
10 & -1 & 0 & 0 & 2\\
-1 & 11 & -1 & 0 & 0\\
0 & -1 & 12 & -2 & 1\\
0 & 0 & -2 & 9 & -1\\
2 & 0 & 1 & -1 & 8
\end{pmatrix}}_{A_{5\times5}}
\begin{pmatrix}x_1\\x_2\\x_3\\x_4\\x_5\end{pmatrix}
=
\underbrace{\begin{pmatrix}10\\21\\1\\-10\\11\end{pmatrix}}_{b}.
\]
\end{Ejer}



\end{document}
