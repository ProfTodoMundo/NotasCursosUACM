%===========================================
\documentclass[12pt]{article}
%===========================================
\usepackage[utf8]{inputenc}
\usepackage{amsmath,amssymb,amsthm,amsfonts}
\usepackage{graphicx,graphics}
\usepackage{hyperref}
\usepackage{fancyhdr}
\usepackage{geometry}
\usepackage{multicol}
\usepackage{color}
\usepackage{float}
\usepackage{verbatim}
\usepackage{booktabs}
\usepackage{tikz}
\usetikzlibrary{calc}

\geometry{left=2cm,right=2cm,top=2.2cm,bottom=2.2cm}

\pagestyle{fancy}
\fancyhf{}
\rhead{Métodos Numéricos}
\lhead{Método de la Secante — versión detallada}
\rfoot{\thepage}

\begin{document}

%===========================================
\begin{center}
\Large\textbf{Método de la Secante — Clase con 4 ejemplos, iteraciones paso a paso}\\[0.35cm]
\large Autor: Carlos E. Martínez-Rodríguez
\end{center}
%===========================================

\section*{1. Idea, fórmula y consideraciones}
El método de la secante aproxima la derivada mediante la pendiente de la recta que une dos puntos de la curva, evitando calcular \(f'(x)\).
Dado \(x_{k-1}\) y \(x_k\), definimos
\[
x_{k+1} \;=\; x_k \;-\; f(x_k)\,\frac{x_k-x_{k-1}}{\,f(x_k)-f(x_{k-1})\,}.
\]
\textbf{Criterios de paro:} \(|x_{k+1}-x_k|<\varepsilon\) o \(|f(x_{k+1})|<\varepsilon\).  
\textbf{Ventajas:} no requiere derivada; suele ser más rápido que bisección.  
\textbf{Precaución:} elegir dos semillas razonables; evitar \(f(x_k)\approx f(x_{k-1})\) (división por número muy pequeño).

\section*{2. Figura: la recta secante}


%==================== EJEMPLO A ====================
\section*{3. Ejemplo A — Cuadrática: \(f(x)=x^2-4\). Semillas: \(x_0=1\), \(x_1=3\)}
\textbf{Explicación.} Buscamos la raíz positiva (2). La secante se construye con \(x_0,x_1\) y el nuevo punto \(x_2\) es el cruce con el eje \(x\).
\[
f(x)=x^2-4,\quad f(1)=-3,\quad f(3)=5.
\]
\textbf{Iteración 1.}
\[
x_2=3-\frac{5\,(3-1)}{5-(-3)}=1.75,\qquad |x_2-x_1|=1.25.
\]
\textbf{Iteración 2.} (usar \(x_0\leftarrow 3\), \(x_1\leftarrow 1.75\))
\[
f(1.75)=-0.9375,\quad x_3=1.75-\frac{-0.9375\,(1.75-3)}{-0.9375-5}=1.947368421.
\]
\textbf{Iteración 3.} (ahora \(x_0\leftarrow 1.75\), \(x_1\leftarrow 1.947368421\))
\[
f(1.947368421)=-0.207756233,\; x_4=1.947368421-\frac{-0.207756233\,(1.947368421-1.75)}{-0.207756233-(-0.9375)}=2.003558719.
\]
\textbf{Iteración 4.}
\[
f(2.003558719)=0.014247540,\; x_5=2.003558719-\frac{0.014247540\,(2.003558719-1.947368421)}{0.014247540-(-0.207756233)}=1.999952593.
\]
\textbf{Iteración 5.}
\[
f(1.999952593)=-1.89625\times 10^{-4},\; x_6=1.999999958\;\Rightarrow\; x\approx 2.000000000.
\]

\noindent\textbf{R sin funciones ni bucles (Iteraciones 1–5 explícitas).}
\begin{verbatim}
# f(x) = x^2 - 4   (x0=1, x1=3)
x0 <- 1.0; x1 <- 3.0

# Iteración 1
f0 <- x0^2 - 4; f1 <- x1^2 - 4
x2 <- x1 - f1*(x1 - x0)/(f1 - f0)
x0 <- x1; x1 <- x2

# Iteración 2
f0 <- x0^2 - 4; f1 <- x1^2 - 4
x2 <- x1 - f1*(x1 - x0)/(f1 - f0)
x0 <- x1; x1 <- x2

# Iteración 3
f0 <- x0^2 - 4; f1 <- x1^2 - 4
x2 <- x1 - f1*(x1 - x0)/(f1 - f0)
x0 <- x1; x1 <- x2

# Iteración 4
f0 <- x0^2 - 4; f1 <- x1^2 - 4
x2 <- x1 - f1*(x1 - x0)/(f1 - f0)
x0 <- x1; x1 <- x2

# Iteración 5
f0 <- x0^2 - 4; f1 <- x1^2 - 4
x2 <- x1 - f1*(x1 - x0)/(f1 - f0)
root_aprox <- x2
print(root_aprox)
\end{verbatim}

%==================== EJEMPLO B ====================
\section*{4. Ejemplo B — Cúbica: \(f(x)=x^3-x-2\). Semillas: \(x_0=1\), \(x_1=2\)}
\textbf{Explicación.} La raíz real está cerca de 1.52. Partimos con \([1,2]\), la secante cae dentro de la cuenca.
\[
f(1)=-2,\qquad f(2)=4.
\]
\textbf{Iteración 1.}
\[
x_2=2-\frac{4\,(2-1)}{4-(-2)}=1.333333333.
\]
\textbf{Iteración 2.} (con \(x_0=2\), \(x_1=1.333333333\))
\[
f(1.333333333)=-0.962962963,\;\; x_3=1.462686567.
\]
\textbf{Iteración 3.}
\[
f(1.462686567)=-0.333338875,\;\; x_4=1.531169432.
\]
\textbf{Iteración 4.}
\[
f(1.531169432)=0.058626418,\;\; x_5=1.520926421.
\]
\textbf{Iteración 5.}
\[
f(1.520926421)=-0.002693300,\;\; x_6=1.521376317\;\Rightarrow\; 
\text{una más: } x_7\approx 1.521379708.
\]

\noindent\textbf{R sin funciones ni bucles (Iteraciones 1–5).}
\begin{verbatim}
# f(x) = x^3 - x - 2   (x0=1, x1=2)
x0 <- 1.0; x1 <- 2.0

# Iteración 1
f0 <- x0^3 - x0 - 2; f1 <- x1^3 - x1 - 2
x2 <- x1 - f1*(x1 - x0)/(f1 - f0)
x0 <- x1; x1 <- x2

# Iteración 2
f0 <- x0^3 - x0 - 2; f1 <- x1^3 - x1 - 2
x2 <- x1 - f1*(x1 - x0)/(f1 - f0)
x0 <- x1; x1 <- x2

# Iteración 3
f0 <- x0^3 - x0 - 2; f1 <- x1^3 - x1 - 2
x2 <- x1 - f1*(x1 - x0)/(f1 - f0)
x0 <- x1; x1 <- x2

# Iteración 4
f0 <- x0^3 - x0 - 2; f1 <- x1^3 - x1 - 2
x2 <- x1 - f1*(x1 - x0)/(f1 - f0)
x0 <- x1; x1 <- x2

# Iteración 5
f0 <- x0^3 - x0 - 2; f1 <- x1^3 - x1 - 2
x2 <- x1 - f1*(x1 - x0)/(f1 - f0)
root_aprox <- x2
print(root_aprox)
\end{verbatim}

%==================== EJEMPLO C ====================
\section*{5. Ejemplo C — Trigonométrica: \(f(x)=\cos x - x\). Semillas: \(x_0=0.5\), \(x_1=1.0\)}
\textbf{Explicación.} La intersección \(\cos x=x\) está en \(x\approx 0.73908\). Con estas semillas la secante es estable.
\[
f(0.5)=0.37758256,\quad f(1.0)=-0.45969769.
\]
\textbf{Iteración 1.}
\[
x_2=0.725481587,\quad f(x_2)\approx 0.022698391.
\]
\textbf{Iteración 2.}
\[
x_3=0.738398620,\quad f(x_3)\approx 0.001148782.
\]
\textbf{Iteración 3.}
\[
x_4=0.739087211,\quad f(x_4)\approx -3.4771\times 10^{-6}.
\]
\textbf{Iteración 4.}
\[
x_5=0.739085133\;\Rightarrow\; x\approx 0.739085133.
\]

\noindent\textbf{R sin funciones ni bucles (Iteraciones 1–4).}
\begin{verbatim}
# f(x) = cos(x) - x   (x0=0.5, x1=1.0)
x0 <- 0.5; x1 <- 1.0

# Iteración 1
f0 <- cos(x0) - x0; f1 <- cos(x1) - x1
x2 <- x1 - f1*(x1 - x0)/(f1 - f0)
x0 <- x1; x1 <- x2

# Iteración 2
f0 <- cos(x0) - x0; f1 <- cos(x1) - x1
x2 <- x1 - f1*(x1 - x0)/(f1 - f0)
x0 <- x1; x1 <- x2

# Iteración 3
f0 <- cos(x0) - x0; f1 <- cos(x1) - x1
x2 <- x1 - f1*(x1 - x0)/(f1 - f0)
x0 <- x1; x1 <- x2

# Iteración 4
f0 <- cos(x0) - x0; f1 <- cos(x1) - x1
x2 <- x1 - f1*(x1 - x0)/(f1 - f0)
root_aprox <- x2
print(root_aprox)
\end{verbatim}

%==================== EJEMPLO D ====================
\section*{6. Ejemplo D — Exponencial/asimilada: \(f(x)=e^{-x}-x\). Semillas: \(x_0=0\), \(x_1=1\)}
\textbf{Explicación.} La solución de \(e^{-x}=x\) está en \(x\approx 0.567143\). Este ejemplo muestra estabilidad típica.
\[
f(0)=1,\quad f(1)=e^{-1}-1\approx -0.632120559.
\]
\textbf{Iteración 1.}
\[
x_2=0.612699837,\quad f(x_2)\approx -0.070813948.
\]
\textbf{Iteración 2.}
\[
x_3=0.563838389,\quad f(x_3)\approx 0.005182355.
\]
\textbf{Iteración 3.}
\[
x_4=0.567170358,\quad f(x_4)\approx -4.2419\times 10^{-5}.
\]
\textbf{Iteración 4.}
\[
x_5=0.567143307,\quad f(x_5)\approx -2.5380\times 10^{-8}.
\]
\textbf{Iteración 5.}
\[
x_6=0.567143290\;\Rightarrow\; x\approx 0.567143290.
\]

\noindent\textbf{R sin funciones ni bucles (Iteraciones 1–5).}
\begin{verbatim}
# f(x) = exp(-x) - x   (x0=0, x1=1)
x0 <- 0.0; x1 <- 1.0

# Iteración 1
f0 <- exp(-x0) - x0; f1 <- exp(-x1) - x1
x2 <- x1 - f1*(x1 - x0)/(f1 - f0)
x0 <- x1; x1 <- x2

# Iteración 2
f0 <- exp(-x0) - x0; f1 <- exp(-x1) - x1
x2 <- x1 - f1*(x1 - x0)/(f1 - f0)
x0 <- x1; x1 <- x2

# Iteración 3
f0 <- exp(-x0) - x0; f1 <- exp(-x1) - x1
x2 <- x1 - f1*(x1 - x0)/(f1 - f0)
x0 <- x1; x1 <- x2

# Iteración 4
f0 <- exp(-x0) - x0; f1 <- exp(-x1) - x1
x2 <- x1 - f1*(x1 - x0)/(f1 - f0)
x0 <- x1; x1 <- x2

# Iteración 5
f0 <- exp(-x0) - x0; f1 <- exp(-x1) - x1
x2 <- x1 - f1*(x1 - x0)/(f1 - f0)
root_aprox <- x2
print(root_aprox)
\end{verbatim}

\section*{7. Qué enfatizar al explicar}
\begin{itemize}
\item \textbf{Geometría:} dibuja la recta secante y muestra el cruce con el eje \(x\).
\item \textbf{Semillas:} dos puntos iniciales razonables evitan oscilaciones y divisores pequeños.
\item \textbf{Comparación:} más rápido que bisección; compite con Newton cuando no puedes derivar.
\item \textbf{Orden de convergencia:} \(\approx 1.618\) (número áureo).
\end{itemize}


\section*{1. Idea y fórmula}
Buscamos una raíz de \(f(x)=0\) usando la recta \emph{tangente} en \(x_k\). La actualización es
\begin{equation*}
x_{k+1} = x_k - \frac{f(x_k)}{f'(x_k)}.
\end{equation*}
Si la raíz \(\alpha\) es simple \((f'(\alpha)\neq 0)\) y \(x_0\) está cerca, la convergencia es típicamente de \textbf{orden 2}.\\
Criterios de paro: \(|x_{k+1}-x_k|<\varepsilon\) o \(|f(x_{k+1})|<\varepsilon\).

\section*{2. Figura: tangente en \(x_k\)}


\section*{3. Algoritmo (pizarrón)}
\begin{verbatim}
Entrada: f(x), f'(x), x0, tolerancia eps, Nmax
x = x0
Para k = 1..Nmax:
    fx = f(x); dfx = f'(x)
    si dfx = 0: detener (tangente horizontal)
    x1 = x - fx/dfx
    si |x1 - x| < eps o |f(x1)| < eps: retornar x1
    x = x1
Fin
\end{verbatim}

%========================= EJEMPLOS =========================

\section*{4. Ejemplo A (cuadrática): \(f(x)=x^2-4\), \(f'(x)=2x\).  Semilla \(x_0=3\)}
\textbf{Iteraciones “en lápiz”:}
\begin{align*}
&x_1 = 3 - \frac{9-4}{2\cdot 3} = 3 - \frac{5}{6} = 2.166666667.\\
&x_2 = 2.166666667 - \frac{(2.166666667)^2-4}{2\cdot 2.166666667}
     = 2.166666667 - \frac{0.694444444}{4.333333334}
     = 2.006410256.\\
&x_3 = 2.006410256 - \frac{(2.006410256)^2-4}{2\cdot 2.006410256}
     \approx 2.000006410.\\
&x_4 \approx 2.000000000.
\end{align*}
\(\Rightarrow\) Raíz aproximada \(x\approx 2\).

\noindent\textbf{R (sin funciones ni bucles):}
\begin{verbatim}
# f(x) = x^2 - 4,  f'(x) = 2*x,  x0 = 3
x <- 3.0

# k=1
fx <- x^2 - 4; dfx <- 2*x
x1 <- x - fx/dfx; err <- abs(x1 - x); x <- x1

# k=2
fx <- x^2 - 4; dfx <- 2*x
x1 <- x - fx/dfx; err <- abs(x1 - x); x <- x1

# k=3
fx <- x^2 - 4; dfx <- 2*x
x1 <- x - fx/dfx; err <- abs(x1 - x); x <- x1

root_aprox <- x
print(root_aprox)
\end{verbatim}

\section*{5. Ejemplo B (cúbica): \(f(x)=x^3-x-2\), \(f'(x)=3x^2-1\).  Semilla \(x_0=1.5\)}
\textbf{Iteraciones “en lápiz”:}
\begin{align*}
&f(1.5) = -0.125,\;\; f'(1.5)=5.75,\;\; x_1=1.5 - \frac{-0.125}{5.75}=1.521739130.\\
&f(1.521739130)\approx 0.002137,\;\; f'(1.521739130)\approx 5.947070,\\
&\quad x_2 = 1.521739130 - \frac{0.002137}{5.947070} \approx 1.521380215.\\
&f(1.521380215)\approx 6.03\times 10^{-7},\;\; f'(1.521380215)\approx 5.943790,\\
&\quad x_3 \approx 1.521380005,\;\; x_4 \approx 1.521379706.
\end{align*}
\(\Rightarrow\) Raíz \(x\approx 1.521379706\).

\noindent\textbf{R (sin funciones ni bucles):}
\begin{verbatim}
# f(x) = x^3 - x - 2,  f'(x) = 3*x^2 - 1,  x0 = 1.5
x <- 1.5

# k=1
fx <- x^3 - x - 2; dfx <- 3*x^2 - 1
x1 <- x - fx/dfx; err <- abs(x1 - x); x <- x1

# k=2
fx <- x^3 - x - 2; dfx <- 3*x^2 - 1
x1 <- x - fx/dfx; err <- abs(x1 - x); x <- x1

# k=3
fx <- x^3 - x - 2; dfx <- 3*x^2 - 1
x1 <- x - fx/dfx; err <- abs(x1 - x); x <- x1

root_aprox <- x
print(root_aprox)
\end{verbatim}

\section*{6. Ejemplo C (trigonométrica): \(f(x)=\cos x - x\), \(f'(x)=-\sin x - 1\).  Semilla \(x_0=0.5\)}
\textbf{Iteraciones “en lápiz”:}
\begin{align*}
&f(0.5)=0.37758256,\;\; f'(0.5)=-\sin(0.5)-1\approx -1.47942554,\\
&\quad x_1 = 0.5 - \frac{0.37758256}{-1.47942554} = 0.755222.\\
&f(0.755222)\approx -0.027103,\;\; f'(0.755222)\approx -1.685451,\\
&\quad x_2 = 0.755222 - \frac{-0.027103}{-1.685451} \approx 0.739142.\\
&f(0.739142)\approx -9.46\times 10^{-5},\;\; f'(0.739142)\approx -1.673654,\\
&\quad x_3 \approx 0.739085,\;\; x_4 \approx 0.739085133.
\end{align*}
\(\Rightarrow\) Raíz \(x\approx 0.739085133\).

\section*{7. Ejemplo D (exponencial/asimilada): \(f(x)=e^{-x}-x\), \(f'(x)=-e^{-x}-1\).  Semilla \(x_0=0\)}
\textbf{Iteraciones “en lápiz”:}
\begin{align*}
&f(0)=1,\;\; f'(0)=-2 \;\Rightarrow\; x_1 = 0 - \frac{1}{-2} = 0.5.\\
&f(0.5)=e^{-0.5}-0.5\approx 0.10653066,\;\; f'(0.5)=-e^{-0.5}-1\approx -1.60653066,\\
&\quad x_2 = 0.5 - \frac{0.10653066}{-1.60653066} \approx 0.566311.\\
&f(0.566311)\approx 0.001157,\;\; f'(0.566311)\approx -1.567468,\\
&\quad x_3 \approx 0.567049,\;\; x_4 \approx 0.56714329.
\end{align*}
\(\Rightarrow\) Raíz \(x\approx 0.56714329\).

\section*{8. Comentarios finales}
Newton–Raphson es muy rápido cuando aplica (orden \(2\)), pero requiere derivada y puede fallar si \(f'(x_k)\) es pequeño o nulo, o si la semilla está lejos.  
Para raíces múltiples, la convergencia se degrada a lineal.

\section*{1. Idea, fórmula y condición de convergencia}
Dada una ecuación \(f(x)=0\), elegimos una función \(g\) tal que
\[
f(x)=0 \quad \Longleftrightarrow \quad x=g(x).
\]
El método de \textbf{punto fijo} construye la sucesión
\[
x_{k+1}=g(x_k),\qquad k=0,1,2,\dots
\]
\textbf{Teorema (contracción local).} Si \(g\) es continua en un intervalo \(I\) que contiene a la raíz \(r\), \(g(I)\subset I\) y
\(\displaystyle \max_{x\in I}|g'(x)|=L<1\), entonces existe un único punto fijo \(r\in I\) y para \(x_0\in I\) se cumple \(x_k\to r\) con convergencia al menos lineal (factor \(\le L\)).

\textbf{Comentarios prácticos.} La elección de \(g\) es crucial: distintas reescrituras de \(f(x)=0\) dan distintos \(g\). Buscamos un \(g\) con \(|g'(r)|<1\) y, si es posible, pequeño para acelerar.

\section*{2. Figura: intersección \(y=g(x)\) con \(y=x\)}


\section*{3. Algoritmo (pizarrón)}
\begin{verbatim}
Entrada: g(x), x0, tolerancia eps, Nmax
x = x0
Para k = 1..Nmax:
    x1 = g(x)
    si |x1 - x| < eps  o  |f(x1)| < eps: retornar x1
    x = x1
Fin
\end{verbatim}

%==================== EJEMPLO A ====================
\section*{4. Ejemplo A — Cuadrática \(f(x)=x^2-4=0\)}
\textbf{Elección de \(g\).} Para la raíz positiva, una elección estable es
\[
g(x)=\frac{1}{2}\Big(x+\frac{4}{x}\Big)\quad\text{(iteración de Herón para }\sqrt{4}\text{)}.
\]
Entonces \(x_{k+1}=g(x_k)\) y el punto fijo es \(r=2\). Tomamos \(x_0=3\).

\textbf{Iteraciones “en lápiz” (5 pasos):}
\[
\begin{aligned}
\text{Iteración 1: }& x_1=\tfrac12(3+\tfrac{4}{3})=2.166666667.\\
\text{Iteración 2: }& x_2=\tfrac12\!\Big(2.166666667+\tfrac{4}{2.166666667}\Big)=2.006410256.\\
\text{Iteración 3: }& x_3=\tfrac12\!\Big(2.006410256+\tfrac{4}{2.006410256}\Big)=2.000003626.\\
\text{Iteración 4: }& x_4=\tfrac12\!\Big(2.000003626+\tfrac{4}{2.000003626}\Big)=2.000000000.\\
\text{Iteración 5: }& x_5\approx 2.000000000\quad(\text{ya estable}).\\
\end{aligned}
\]
\(\Rightarrow\) Raíz \(r\approx 2\). Aquí \(|g'(r)|=\tfrac12(1-\tfrac{4}{r^2})=0\), por eso converge muy rápido.

\noindent\textbf{R sin funciones ni bucles (Iteraciones 1–5).}
\begin{verbatim}
# g(x) = 0.5*(x + 4/x)   (x0 = 3)
x <- 3.0

# Iteración 1
x1 <- 0.5*(x + 4/x); x <- x1

# Iteración 2
x1 <- 0.5*(x + 4/x); x <- x1

# Iteración 3
x1 <- 0.5*(x + 4/x); x <- x1

# Iteración 4
x1 <- 0.5*(x + 4/x); x <- x1

# Iteración 5
x1 <- 0.5*(x + 4/x); x <- x1

root_aprox <- x
print(root_aprox)
\end{verbatim}

%==================== EJEMPLO B ====================
\section*{5. Ejemplo B — Cúbica \(f(x)=x^3-x-2=0\)}
\textbf{Elección de \(g\).} Reescribimos \(x=\sqrt[3]{x+2}\):
\[
g(x)=(x+2)^{1/3}.
\]
Cerca de la raíz \(r\approx 1.52138\), \(|g'(r)|=\frac{1}{3}(r+2)^{-2/3}<1\) (estable). Tomamos \(x_0=1\).

\textbf{Iteraciones “en lápiz” (5 pasos):}
\[
\begin{aligned}
\text{Iteración 1: }& x_1=(1+2)^{1/3}=1.44224957.\\
\text{Iteración 2: }& x_2=(1.44224957+2)^{1/3}=1.51571657.\\
\text{Iteración 3: }& x_3=(1.51571657+2)^{1/3}=1.52137900.\\
\text{Iteración 4: }& x_4=(1.52137900+2)^{1/3}=1.52137966.\\
\text{Iteración 5: }& x_5=(1.52137966+2)^{1/3}=1.52137971.\\
\end{aligned}
\]
\(\Rightarrow\) Raíz \(r\approx 1.52137971\).

\noindent\textbf{R sin funciones ni bucles (Iteraciones 1–5).}
\begin{verbatim}
# g(x) = (x + 2)^(1/3)   (x0 = 1)
x <- 1.0

# Iteración 1
x1 <- (x + 2)^(1/3); x <- x1

# Iteración 2
x1 <- (x + 2)^(1/3); x <- x1

# Iteración 3
x1 <- (x + 2)^(1/3); x <- x1

# Iteración 4
x1 <- (x + 2)^(1/3); x <- x1

# Iteración 5
x1 <- (x + 2)^(1/3); x <- x1

root_aprox <- x
print(root_aprox)
\end{verbatim}

%==================== EJEMPLO C ====================
\section*{6. Ejemplo C — Trigonométrica \(f(x)=\cos x - x=0\)}
\textbf{Elección de \(g\).} La forma natural es \(x=\cos x\):
\[
g(x)=\cos x.
\]
En la raíz \(r\approx 0.739085\), \(|g'(r)|=|\!- \sin r|\approx 0.673<1\). Tomamos \(x_0=0.5\).

\textbf{Iteraciones “en lápiz” (5 pasos):}
\[
\begin{aligned}
\text{Iteración 1: }& x_1=\cos(0.5)=0.87758256.\\
\text{Iteración 2: }& x_2=\cos(0.87758256)=0.63901249.\\
\text{Iteración 3: }& x_3=\cos(0.63901249)=0.80268510.\\
\text{Iteración 4: }& x_4=\cos(0.80268510)=0.69477803.\\
\text{Iteración 5: }& x_5=\cos(0.69477803)=0.76819583.\\
\end{aligned}
\]
(Continúa acercándose: \(0.71916545,\,0.75235576,\,0.73008106,\dots\) hasta \(0.739085\).)  
\(\Rightarrow\) Convergencia \emph{lenta pero segura} con esta \(g\). Si deseas acelerar, pueden usarse otras transformaciones o aceleración de Aitken.

\noindent\textbf{R sin funciones ni bucles (Iteraciones 1–5).}
\begin{verbatim}
# g(x) = cos(x)   (x0 = 0.5)
x <- 0.5

# Iteración 1
x1 <- cos(x); x <- x1

# Iteración 2
x1 <- cos(x); x <- x1

# Iteración 3
x1 <- cos(x); x <- x1

# Iteración 4
x1 <- cos(x); x <- x1

# Iteración 5
x1 <- cos(x); x <- x1

root_aprox <- x
print(root_aprox)
\end{verbatim}

%==================== EJEMPLO D ====================
\section*{7. Ejemplo D — Exponencial/asimilada \(f(x)=e^{-x}-x=0\)}
\textbf{Elección de \(g\).} La reescritura directa es
\[
g(x)=e^{-x}.
\]
En la raíz \(r\approx 0.567143\), \(|g'(r)|=e^{-r}\approx 0.567<1\). Tomamos \(x_0=1\).

\textbf{Iteraciones “en lápiz” (5 pasos):}
\[
\begin{aligned}
\text{Iteración 1: }& x_1=e^{-1}=0.36787944.\\
\text{Iteración 2: }& x_2=e^{-0.36787944}=0.69220063.\\
\text{Iteración 3: }& x_3=e^{-0.69220063}=0.50047350.\\
\text{Iteración 4: }& x_4=e^{-0.50047350}=0.60624354.\\
\text{Iteración 5: }& x_5=e^{-0.60624354}=0.54539579.\\
\end{aligned}
\]
(Sigue: \(0.57961234,\,0.56011546,\,0.57114312,\,0.56487935,\,0.56842873,\dots\to 0.56714329\).)

\noindent\textbf{R sin funciones ni bucles (Iteraciones 1–5).}
\begin{verbatim}
# g(x) = exp(-x)   (x0 = 1)
x <- 1.0

# Iteración 1
x1 <- exp(-x); x <- x1

# Iteración 2
x1 <- exp(-x); x <- x1

# Iteración 3
x1 <- exp(-x); x <- x1

# Iteración 4
x1 <- exp(-x); x <- x1

# Iteración 5
x1 <- exp(-x); x <- x1

root_aprox <- x
print(root_aprox)
\end{verbatim}

\section*{8. Qué enfatizar al explicar}
\begin{itemize}
\item \textbf{Elección de \(g\):} no todas las reescrituras convergen; verifica que \(|g'(r)|<1\).
\item \textbf{Geometría:} el punto fijo es la intersección de \(y=g(x)\) con \(y=x\); la “escalera” (cobweb) ilustra el proceso.
\item \textbf{Velocidad:} típica \emph{lineal}. Puede ser lenta (ej. \(g(x)=\cos x\)).
\item \textbf{Práctica:} si dispones de \(f'(x)\), Newton suele ser más rápido; si no, secante es un buen compromiso.
\end{itemize}


\section*{1. Idea, fórmula y condición de convergencia}
Dada una ecuación \(f(x)=0\), elegimos una transformación \(g\) tal que
\[
f(x)=0\quad \Longleftrightarrow\quad x=g(x).
\]
El método de \textbf{punto fijo} genera la sucesión
\[
x_{k+1}=g(x_k),\qquad k=0,1,2,\ldots
\]
\textbf{Convergencia local (contracción).} Si existe un intervalo \(I\) con \(r\in I\), \(g(I)\subset I\) y
\(\displaystyle \max_{x\in I}|g'(x)|=L<1\), entonces \(r\) es único en \(I\) y \(x_k\to r\) con, al menos, convergencia lineal (factor \(\le L\)).\\
\textbf{Comentarios.} La \underline{elección de \(g\)} es clave: buscamos \(g\) con \(|g'(r)|<1\). Distintas reescrituras de \(f(x)=0\) pueden converger o no.

\section*{2. Figura: intersección \(y=g(x)\) con \(y=x\)}

\section*{3. Algoritmo (pizarrón)}
\begin{verbatim}
Entrada: g(x), x0, tolerancia eps, Nmax
x = x0
Para k = 1..Nmax:
    x1 = g(x)
    si |x1 - x| < eps  o  |f(x1)| < eps: retornar x1
    x = x1
Fin
\end{verbatim}

%==================== EJEMPLO A ====================
\section*{4. Ejemplo A — Cuadrática \(f(x)=x^2-4=0\)}
\textbf{Elección de \(g\).} Para \(\sqrt{4}\), la iteración de Herón:
\[
g(x)=\frac{1}{2}\left(x+\frac{4}{x}\right).
\]
El punto fijo es \(r=2\). Semilla \(x_0=3\).

\noindent\textbf{Iteraciones “en lápiz” (1–5):}
\[
\begin{aligned}
\text{Iteración 1: }& x_1=\tfrac12\!\left(3+\tfrac{4}{3}\right)=2.166666667.\\
\text{Iteración 2: }& x_2=\tfrac12\!\left(2.166666667+\tfrac{4}{2.166666667}\right)=2.006410256.\\
\text{Iteración 3: }& x_3=\tfrac12\!\left(2.006410256+\tfrac{4}{2.006410256}\right)=2.000003626.\\
\text{Iteración 4: }& x_4=\tfrac12\!\left(2.000003626+\tfrac{4}{2.000003626}\right)=2.000000000.\\
\text{Iteración 5: }& x_5\approx 2.000000000\ (\text{estabilizado}).\\
\end{aligned}
\]
\(\Rightarrow\) Raíz \(r\approx 2\). Aquí \(|g'(r)|=0\) \(\Rightarrow\) convergencia súper-rápida.

\noindent\textbf{R sin funciones ni bucles (Iteraciones 1–5):}
\begin{verbatim}
# g(x) = 0.5*(x + 4/x)   (x0 = 3)
x <- 3.0

# Iteración 1
x1 <- 0.5*(x + 4/x); x <- x1

# Iteración 2
x1 <- 0.5*(x + 4/x); x <- x1

# Iteración 3
x1 <- 0.5*(x + 4/x); x <- x1

# Iteración 4
x1 <- 0.5*(x + 4/x); x <- x1

# Iteración 5
x1 <- 0.5*(x + 4/x); x <- x1

root_aprox <- x
print(root_aprox)
\end{verbatim}

%==================== EJEMPLO B ====================
\section*{5. Ejemplo B — Cúbica \(f(x)=x^3-x-2=0\)}
\textbf{Elección de \(g\).} Reescritura simple:
\[
g(x)=(x+2)^{1/3}.
\]
Cerca de la raíz \(r\approx 1.52138\), \(|g'(r)|=\frac{1}{3}(r+2)^{-2/3}<1\) (estable). Semilla \(x_0=1\).

\noindent\textbf{Iteraciones “en lápiz” (1–5):}
\[
\begin{aligned}
\text{Iteración 1: }& x_1=(1+2)^{1/3}=1.44224957.\\
\text{Iteración 2: }& x_2=(1.44224957+2)^{1/3}=1.51571657.\\
\text{Iteración 3: }& x_3=(1.51571657+2)^{1/3}=1.52137900.\\
\text{Iteración 4: }& x_4=(1.52137900+2)^{1/3}=1.52137966.\\
\text{Iteración 5: }& x_5=(1.52137966+2)^{1/3}=1.52137971.\\
\end{aligned}
\]
\(\Rightarrow\) Raíz \(r\approx 1.52137971\).

\noindent\textbf{R sin funciones ni bucles (Iteraciones 1–5):}
\begin{verbatim}
# g(x) = (x + 2)^(1/3)   (x0 = 1)
x <- 1.0

# Iteración 1
x1 <- (x + 2)^(1/3); x <- x1

# Iteración 2
x1 <- (x + 2)^(1/3); x <- x1

# Iteración 3
x1 <- (x + 2)^(1/3); x <- x1

# Iteración 4
x1 <- (x + 2)^(1/3); x <- x1

# Iteración 5
x1 <- (x + 2)^(1/3); x <- x1

root_aprox <- x
print(root_aprox)
\end{verbatim}

%==================== EJEMPLO C ====================
\section*{6. Ejemplo C — Trigonométrica \(f(x)=\cos x - x=0\)}
\textbf{Elección de \(g\).} Natural:
\[
g(x)=\cos x.
\]
En la raíz \(r\approx 0.739085\), \(|g'(r)|=|\!-\sin r|\approx 0.673<1\) (convergencia lineal). Semilla \(x_0=0.5\).

\noindent\textbf{Iteraciones “en lápiz” (1–5):}
\[
\begin{aligned}
\text{Iteración 1: }& x_1=\cos(0.5)=0.87758256.\\
\text{Iteración 2: }& x_2=\cos(0.87758256)=0.63901249.\\
\text{Iteración 3: }& x_3=\cos(0.63901249)=0.80268510.\\
\text{Iteración 4: }& x_4=\cos(0.80268510)=0.69477803.\\
\text{Iteración 5: }& x_5=\cos(0.69477803)=0.76819583.\\
\end{aligned}
\]
(Continúa: \(0.71916545,\,0.75235576,\,0.73008106,\ldots\to 0.739085\).) \\
\(\Rightarrow\) Convergencia segura pero \emph{lenta} con esta \(g\). (Posible acelerar con Aitken \(\Delta^2\).)

\noindent\textbf{R sin funciones ni bucles (Iteraciones 1–5):}
\begin{verbatim}
# g(x) = cos(x)   (x0 = 0.5)
x <- 0.5

# Iteración 1
x1 <- cos(x); x <- x1

# Iteración 2
x1 <- cos(x); x <- x1

# Iteración 3
x1 <- cos(x); x <- x1

# Iteración 4
x1 <- cos(x); x <- x1

# Iteración 5
x1 <- cos(x); x <- x1

root_aprox <- x
print(root_aprox)
\end{verbatim}

%==================== EJEMPLO D ====================
\section*{7. Ejemplo D — Exponencial/asimilada \(f(x)=e^{-x}-x=0\)}
\textbf{Elección de \(g\).} Directa:
\[
g(x)=e^{-x}.
\]
En \(r\approx 0.567143\), \(|g'(r)|=e^{-r}\approx 0.567<1\). Semilla \(x_0=1\).

\noindent\textbf{Iteraciones “en lápiz” (1–5):}
\[
\begin{aligned}
\text{Iteración 1: }& x_1=e^{-1}=0.36787944.\\
\text{Iteración 2: }& x_2=e^{-0.36787944}=0.69220063.\\
\text{Iteración 3: }& x_3=e^{-0.69220063}=0.50047350.\\
\text{Iteración 4: }& x_4=e^{-0.50047350}=0.60624354.\\
\text{Iteración 5: }& x_5=e^{-0.60624354}=0.54539579.\\
\end{aligned}
\]
(Sigue: \(0.57961234,\,0.56011546,\,0.57114312,\,0.56487935,\,0.56842873,\ldots\to 0.56714329\).)

\noindent\textbf{R sin funciones ni bucles (Iteraciones 1–5):}
\begin{verbatim}
# g(x) = exp(-x)   (x0 = 1)
x <- 1.0

# Iteración 1
x1 <- exp(-x); x <- x1

# Iteración 2
x1 <- exp(-x); x <- x1

# Iteración 3
x1 <- exp(-x); x <- x1

# Iteración 4
x1 <- exp(-x); x <- x1

# Iteración 5
x1 <- exp(-x); x <- x1

root_aprox <- x
print(root_aprox)
\end{verbatim}

\section*{8. Qué enfatizar al explicar}
\begin{itemize}
\item \textbf{Elección de \(g\):} verificar \(|g'(r)|<1\) cerca de la raíz \(\Rightarrow\) contracción.
\item \textbf{Geometría:} la “escalera” (cobweb) hacia \(y=x\) ayuda a visualizar la convergencia.
\item \textbf{Velocidad:} típicamente \emph{lineal}; puede ser lenta si \(|g'(r)|\) es cercano a 1.
\item \textbf{Comparación:} si dispones de \(f'(x)\), Newton es más rápido; sin derivada, Secante es un buen compromiso.
\end{itemize}

\section*{1. Idea, fórmula y condición de convergencia}
Dada una ecuación \(f(x)=0\), elegimos una transformación \(g\) tal que
\[
f(x)=0\quad \Longleftrightarrow\quad x=g(x).
\]
El método de \textbf{punto fijo} genera la sucesión
\[
x_{k+1}=g(x_k),\qquad k=0,1,2,\ldots
\]
\textbf{Convergencia local (contracción).} Si existe un intervalo \(I\) con \(r\in I\), \(g(I)\subset I\) y
\(\displaystyle \max_{x\in I}|g'(x)|=L<1\), entonces \(r\) es único en \(I\) y \(x_k\to r\) con convergencia lineal (factor \(\le L\)).

\textbf{Comentarios.} La elección de \(g\) es clave: buscamos \(g\) con \(|g'(r)|<1\). Distintas reescrituras de \(f(x)=0\) pueden converger o no.

\section*{2. Figura: intersección \(y=g(x)\) con \(y=x\)}


\section*{3. Algoritmo (pizarrón)}
\begin{verbatim}
Entrada: g(x), x0, tolerancia eps, Nmax
x = x0
Para k = 1..Nmax:
    x1 = g(x)
    si |x1 - x| < eps  o  |f(x1)| < eps: retornar x1
    x = x1
Fin
\end{verbatim}

%==================== EJEMPLO A ====================
\section*{4. Ejemplo A — Cuadrática \(f(x)=x^2-4=0\)}
\textbf{Elección de \(g\).} Para \(\sqrt{4}\), la iteración de Herón:
\[
g(x)=\frac{1}{2}\left(x+\frac{4}{x}\right).
\]
El punto fijo es \(r=2\). Semilla \(x_0=3\).

\textbf{Iteraciones “en lápiz”:}
\[
\begin{aligned}
\text{1: }& x_1=\tfrac12(3+\tfrac{4}{3})=2.1667.\\
\text{2: }& x_2=\tfrac12(2.1667+\tfrac{4}{2.1667})=2.0064.\\
\text{3: }& x_3=\tfrac12(2.0064+\tfrac{4}{2.0064})=2.0000.\\
\text{4: }& x_4=\tfrac12(2.0000+\tfrac{4}{2.0000})=2.0000.\\
\text{5: }& x_5\approx2.0000.
\end{aligned}
\]

\noindent\textbf{R sin funciones ni bucles:}
\begin{verbatim}
x <- 3
x1 <- 0.5*(x + 4/x); x <- x1
x1 <- 0.5*(x + 4/x); x <- x1
x1 <- 0.5*(x + 4/x); x <- x1
x1 <- 0.5*(x + 4/x); x <- x1
x1 <- 0.5*(x + 4/x); x <- x1
print(x)
\end{verbatim}

%==================== EJEMPLO B ====================
\section*{5. Ejemplo B — Cúbica \(f(x)=x^3-x-2=0\)}
\textbf{Elección de \(g\).} 
\[
g(x)=(x+2)^{1/3}.
\]
Cerca de la raíz \(r\approx1.52138\). Semilla \(x_0=1\).

\textbf{Iteraciones:}
\[
\begin{aligned}
1:&\ x_1=(1+2)^{1/3}=1.4423.\\
2:&\ x_2=(1.4423+2)^{1/3}=1.5157.\\
3:&\ x_3=(1.5157+2)^{1/3}=1.5214.\\
4:&\ x_4=(1.5214+2)^{1/3}=1.52138.\\
5:&\ x_5=(1.52138+2)^{1/3}=1.52138.
\end{aligned}
\]

\noindent\textbf{R sin funciones ni bucles:}
\begin{verbatim}
x <- 1
x1 <- (x + 2)^(1/3); x <- x1
x1 <- (x + 2)^(1/3); x <- x1
x1 <- (x + 2)^(1/3); x <- x1
x1 <- (x + 2)^(1/3); x <- x1
x1 <- (x + 2)^(1/3); x <- x1
print(x)
\end{verbatim}

%==================== EJEMPLO C ====================
\section*{6. Ejemplo C — Trigonométrica \(f(x)=\cos x - x=0\)}
\textbf{Elección de \(g\).} \(g(x)=\cos x\).  En \(r\approx0.739085\), \(|g'(r)|\approx0.67<1\). Semilla \(x_0=0.5\).

\textbf{Iteraciones:}
\[
\begin{aligned}
1:&\ x_1=\cos(0.5)=0.8776.\\
2:&\ x_2=\cos(0.8776)=0.6390.\\
3:&\ x_3=\cos(0.6390)=0.8027.\\
4:&\ x_4=\cos(0.8027)=0.6948.\\
5:&\ x_5=\cos(0.6948)=0.7682.
\end{aligned}
\]

\noindent\textbf{R sin funciones ni bucles:}
\begin{verbatim}
x <- 0.5
x1 <- cos(x); x <- x1
x1 <- cos(x); x <- x1
x1 <- cos(x); x <- x1
x1 <- cos(x); x <- x1
x1 <- cos(x); x <- x1
print(x)
\end{verbatim}

%==================== EJEMPLO D ====================
\section*{7. Ejemplo D — Exponencial \(f(x)=e^{-x}-x=0\)}
\textbf{Elección de \(g\).} \(g(x)=e^{-x}\).  En \(r\approx0.567143\), \(|g'(r)|\approx0.567<1\). Semilla \(x_0=1\).

\textbf{Iteraciones:}
\[
\begin{aligned}
1:&\ x_1=e^{-1}=0.3679.\\
2:&\ x_2=e^{-0.3679}=0.6922.\\
3:&\ x_3=e^{-0.6922}=0.5005.\\
4:&\ x_4=e^{-0.5005}=0.6062.\\
5:&\ x_5=e^{-0.6062}=0.5454.
\end{aligned}
\]

\noindent\textbf{R sin funciones ni bucles:}
\begin{verbatim}
x <- 1
x1 <- exp(-x); x <- x1
x1 <- exp(-x); x <- x1
x1 <- exp(-x); x <- x1
x1 <- exp(-x); x <- x1
x1 <- exp(-x); x <- x1
print(x)
\end{verbatim}

\section*{8. Qué enfatizar al explicar}
\begin{itemize}
\item Verificar que \(|g'(r)|<1\) antes de iterar.
\item Graficar \(y=g(x)\) y \(y=x\) para ver la intersección.
\item Mostrar el “cobweb” (escalera) que visualiza la convergencia.
\item Comparar velocidad con secante y Newton: punto fijo es más simple pero más lento.
\end{itemize}

\end{document}
