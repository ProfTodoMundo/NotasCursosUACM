
\chapter{Tareas del curso}

\begin{Ejer}
Convertir los siguientes números de base 10 a base 2.
\begin{enumerate}
\item $324$
\item $27$
\item $1423$
\item $235.25$
\item $41.596$
\end{enumerate}
\end{Ejer}


\begin{Ejer}
\begin{enumerate}
\item Realizar una revisión de la historia de los m\'etodos num\'ericos, elaborar un documento de hasta dos cuartillas.
\item Realiza las siguientes conversiones de base $10$ a base $2$:
\begin{enumerate}
\item 246
\item 345.68
\item 4586632.2846
\item 984365.27463
\item 79905523
\end{enumerate}
\item Elabora el c\'odigo en R para realizar la conversi\'on de base $10$ a base $2$.

\item Describe exhaustivamente los tipos de errores que existen

\end{enumerate}


\end{Ejer}


\begin{Note}
En los siguientes ejercicios se indicar\'a cuales ejercicios pueden realizarse sin el apoyo de R, se deben de resolver al menos dos ejercicios de cada serie sin el apoyo de R, es decir, se tienen que resolver manualmente.
\end{Note}

\begin{Ejer}
Resolver por eliminación Gaussiana Simple los siguientes sistemas de ecuaciones lineales

\begin{enumerate}
\item \begin{eqnarray*}
x_1-2x_2+0.5x_3&=&-5\\
-2x_1+5x_{2}-1.5x_3&=&0\\
-0.2x_1+1.75x_2-x_3&=&10
\end{eqnarray*}

\item \begin{eqnarray*}
3x_1-x_2+6x_4&=&2.3\\
4x_1+2x_2-x_3-5x_4&=&6.9\\
-5x_1+x_2-3x_3&=&-36\\
10x_2-4x_3+7x_4&=&-36
\end{eqnarray*}

\item \begin{eqnarray*}
0.003000x_1+59.14x_2&=&59.17\\
5.291x_1-6.130x_2&=&46.78
\end{eqnarray*}
utilizar redondeo a 4 cifras significativas.

\item \begin{eqnarray*}
4x_1+2x_2&=&2\\
2x_1+3x_2+x_3&=&-1\\
x_2+\frac{5}{2}x_3&=&3
\end{eqnarray*}

\item \begin{eqnarray*}
3x_1-0.1x_2-0.2x_3&=&7.85\\
0.1x_{1}+7x_2-0.3x_3=-19.3\\
0.3x_1-0.2x_2+10x_3&=&71.4
\end{eqnarray*}

\item \textbf{*}\begin{eqnarray*}
8x_1+2x_2-2x_3&=&-2\\
10x_1+2x_2+4x_3&=&4\\
12x_1+2x_2+2x_3&=&6
\end{eqnarray*}

\item\textbf{*} \begin{eqnarray*}
5x_1 + 2x_2 + x_3 - x_4 = 1\\
2x_1 - x_2 + 3x_3 + 2x_4 = 12\\
4x_1 + x_2 - 2x_3 + 3x_4 = 5\\
-2x_1 + 2x_2 + x_3 + x_4 = 2
\end{eqnarray*}

\item \textbf{*}\begin{eqnarray*}
2x_1+3x_2+2x_3+4x_4&=&4\\
4x_1+10x_2-4x_3&=&-8\\
-3x_1-2x_2-5x_3-2x_4&=&-4\\
-2x_1+4x_2+4x_3-7x_4&=&-1
\end{eqnarray*}

\item \begin{eqnarray*}
1.133x_1+5.281x_2-2.454x_3&=&6.414\\
24.14x_1-1.21x_2+5.281x_3&=&113.8\\
-10.123x_1+6.387x_2-x_3&=&1
\end{eqnarray*}

\item \begin{eqnarray*}A=\left(\begin{array}{ccccc}
2 & 3 & 2 & 4\\
4 & 10 &-4 & 0\\
-3 & -2 & -5 &-2\\
-2 & 4 & 4 &-7\\
\end{array}\right)\end{eqnarray*} y \begin{eqnarray*}b=\left(\begin{array}{c}\\
9 \\
-15\\
6 \\
2\\
\end{array}\right)\end{eqnarray*}

\end{enumerate}
\end{Ejer}

\begin{Ejer}
Resolver por eliminación gaussiana con pivoteo parcial los siguientes sistemas de ecuaciones lineales

\begin{enumerate}
\item \begin{eqnarray*}
0.4x_1-1.5x_2+0.75x_3&=&-20\\
-0.5x_1-15x_2+10x_3&=&-10\\
-10x_1-9x_2+2.5x_3&=&30
\end{eqnarray*}

\item \textbf{*}\begin{eqnarray*}
5x_1-8x_2+x_3&=&-71\\
-2x_1+6x_2-9x_3&=&134\\
3x_1-5x_2+2x_3&=&-58
\end{eqnarray*}

\item \begin{eqnarray*}
0.003000x_1+59.14x_2&=&59.17\\
5.291x_1-6.130x_2&=&46.78
\end{eqnarray*}
utilizar redondeo a 4 cifras significativas.


\item \textbf{*}\begin{eqnarray*}
-x_2+4x_3-x_4&=&-1\\
-x_1+4x_2-x_3&=&2\\
-x_1-x_3+4x_4&=&4\\
4x_1-x_2&=&10
\end{eqnarray*}

\item \begin{eqnarray*}
0.00031000x_1+1.000000x_2&=&3.000000\\
1.00045534x_1+1.00034333x_2&=&7.000
\end{eqnarray*}


\item \textbf{*} Resolver para $A=\left(\begin{array}{ccccc}\\
14 & 14 & -9 & 3 & -5\\
14 & 52 & -15 & 2 & -32\\
-9 & -15 & 36 &-5 & 16\\
3 &2 &-5&47 & 49\\
-5 & 32 & 16 &49 & 79\end{array}\right)$,  $b=\left(\begin{array}{c}\\
-15\\
-100\\
106\\
329\\
463\end{array}\right)$ y  $X=\left(\begin{array}{c}\\
x_1\\
x_2\\
x_3\\
x_4\\
x_5\end{array}\right)$


\item Resolver para $A=\left(\begin{array}{ccccc}
\frac{1}{4} &\frac{1}{5} &\frac{1}{6}\\
 \frac{1}{3} & \frac{1}{4} & \frac{1}{5} \\  
  \frac{1}{2} &  1& 2
\end{array}\right)$,  $b=\left(\begin{array}{c}
9\\
8\\
8\\\end{array}\right)$ y  $X=\left(\begin{array}{c}
x_1\\
x_2\\
x_3\\
\end{array}\right)$

\item  Resolver para $A=\left(\begin{array}{ccccc}
1 & \frac{1}{2} & \frac{1}{3} & \frac{1}{4}  \\
 \frac{1}{2} & \frac{1}{3} & \frac{1}{4} & \frac{1}{5}\\
 \frac{1}{3} & \frac{1}{4} & \frac{1}{5} & \frac{1}{6}\\
 \frac{1}{4} & \frac{1}{5} & \frac{1}{6} & \frac{1}{7}
 \end{array}\right)$,  $b=\left(\begin{array}{c}
 \frac{1}{6} \\
 \frac{1}{7} \\
  \frac{1}{8}\\ 
 \frac{1}{9} \\\end{array}\right)$ y  $X=\left(\begin{array}{c}
x_1\\
x_2\\
x_3\\
x_4
\end{array}\right)$

\item Resolver el sistema
\begin{eqnarray*}
2x_1+x_2-x_3+x_4-3x_5&=&7\\
x_1+2x_3-x_4+x_5&=&2\\
-2x_2-x_3+x_4-x_5&=&-5\\
3x_1+x_2-4x_3+5x_5&=&6\\
x_1-x_2-x_3-x_4+x_5&=&3
\end{eqnarray*}



\item Resolver el sistema
\begin{eqnarray*}
3.333x_1+15920x_2-10.333x_3&=&15913\\
2.222x_1+16.71x_29.612x_2&=&28.544\\
1.5611x_1+5.1791x_2+1.6852x_3&=&8.4254
\end{eqnarray*}

\end{enumerate}
\end{Ejer}


\begin{Ejer}
Resolver por el método de Gauss-Jordan los siguientes sistemas de ecuaciones lineales

\begin{enumerate}
\item \begin{eqnarray*}
x_1+2x_2+3x_3&=&1\\
-0.4x_1+2x_2-x_3&=&10\\
0.5x_1-3x_2+x_3&=&15
\end{eqnarray*}

\item \begin{eqnarray*}
2x_1-0.9x_2+3x_3&=&-3.61\\
-0.5x_1+0.1x_2-x_3&=&2.035\\
x_1-6.35x_2-0.45x_3&=&15.401
\end{eqnarray*}

\item \begin{eqnarray*}
0.7x_1+2.7x_2-6x_3+0.7x_4&=&1.6487\\
2x_1-0.8x_2+3x_3-x_4&=&-2.342\\
-x_1-1.5x_2+1.4x_3+3x_4&=&-4.189\\
7x_2-1.56x_3+x_4=15.792
\end{eqnarray*}

\item \begin{eqnarray*}
3x_1-0.1x_2-0.2x_3&=&7.85\\
0.1x_1+7x_2-0.3x_3&=&-19.3\\
0.3x_1-0.2x_2+10x_3&=&71.4
\end{eqnarray*}

\item \textbf{*}\begin{eqnarray*}
10x_1+2x_2-x_3&=&27\\
-3x_1-6x_2+2x3&=&-61.5\\
x_1+x_2+5x_3&=&-21.5
\end{eqnarray*}

\item $A=\left(\begin{array}{ccccc}\\
1 &3 & -2 & 1\\
1  &3 & -1 & 2\\
0  &1 & -1 & 4\\
2  &6 & 1 & 2\\
\end{array}\right)$ y $b=\left(\begin{array}{c}
4 \\
1 \\
5\\ 
2\\\end{array}\right)$

\item \begin{eqnarray*}
6x_1-x_2-x_3+4x_4&=&17\\
x_1-10x_2+2x_3-x_4&=&-17\\
3x_1-2x_2+8x_3-x_4&=&19\\
x_1+x_2+x_3-5x_4&=&-14
\end{eqnarray*}

\item \textbf{*}\begin{eqnarray*}
x+2y+3z+4w&=&1\\
x-4y+z+11w&=&2\\
-x+8y+7z+6w&=&-2\\
16x+8y-5z+6w&=&11
\end{eqnarray*}


\item \textbf{*}\begin{eqnarray*}
x_1+x_2&=&3\\
x_1+2x_2+x_3&=&-1\\
x_2+3x_3+x_4&=&2\\
x_3+4x_4+x_5&=&1\\
x_4+5x_5&=&3
\end{eqnarray*}

\item \begin{eqnarray*}
15x_1-18x_2+15x_3-3x_4&=&11\\
-18x_1+24_2-18x_3+4x_4&=&10\\
15x_1-18x_2+18x_3-3x_4&=&11\\
-3x_1+4x_2-3x_3+x_4&=&13
\end{eqnarray*}

\end{enumerate}
\end{Ejer}

\begin{Ejer}
Resolver por el método de Gauss-Seidel los siguientes sistemas de ecuaciones lineales

\begin{enumerate}
\item \begin{eqnarray*}
3x_1-0.2x_2-0.5x_3&=&8\\
0.1x_1+7x_2+0.4x_3&=&-19.5\\
0.4x_1-0.1x_2+10x_3&=&72.4
\end{eqnarray*}

\item \begin{eqnarray*}
-5x_1+1.4x_2-2.7x_3&=&94.2\\
0.7x_1-2.5x_2+15x_3&=&-6\\
3.3x_1-11x_2+4.4x_3&=&-27.5
\end{eqnarray*}

\item \begin{eqnarray*}
3x_1-0.5x_2+0.6x_3&=&5.24\\
0.3x_1-4x_2-x_3&=&-0.387\\
-0.7x_1+2x_2+7x_3&=&14.803
\end{eqnarray*}

\item \begin{eqnarray*}
5x_1-0.2x_2+x_3&=&1.5\\
0.1x_1+3x_2-0.5x_3&=&-2.7\\
-0.3x_1+x_2-7x_3&=&9.5
\end{eqnarray*}

\item \textbf{*}\begin{eqnarray*}
-3x_2+7x_3&=&2\\
x_1+2x_2-x_3&=&3\\
12x_1+2x_2+2x_3&=&6
\end{eqnarray*}

\item \begin{eqnarray*}
0.15_1+2.11x_2+30.75x_3&=&-26.38\\
0.64x_1+1.21x_2+2.05x_3&=&1.01\\
3.21x_1+1.53x_2+1.04x_3&=&5.23
\end{eqnarray*}


\item \textbf{*}\begin{eqnarray*}
x_1+x_2-x_3&=&-3\\
6x_1+2x_2+2x_3&=&2\\
-3x_1+4x_2+x_3&=&1
\end{eqnarray*}

\item \textbf{*}\begin{eqnarray*}
2x_1+x_2-x3&=&1\\
5x_1+2x_2+2x_3&=&-4\\
3x_1+x_2+x_3&=&5
\end{eqnarray*}

\item \begin{eqnarray*}
3x-0.1y-0.2z&=&7.85\\
0.1x+7y-0.3z&=&-19.3\\
0.3x_1-0.2x_2+10x_3=71.4
\end{eqnarray*}


\item \begin{eqnarray*}
17x_1-2x_2-3x_3=500\\
-5x_1+21x_2-2x_3&=&200\\
-5x_1-5x_2+22x_3&=&30
\end{eqnarray*}

\end{enumerate}
\end{Ejer}



\begin{Ejer}
Aplicar el método de Jacobi para resolver los siguientes sistemas de ecuaciones lineales

\begin{enumerate}
\item $A=\left(\begin{array}{cccc|c}\\
10 & 2 &  -1 &  0 &  26\\
1 & 20 & -2 & 3 & -15\\
-2 & 1 & 30 & 0 & 53\\
1 & 2 & 3 & 20 & 47
\end{array}\right)$


\item \textbf{*}$A=\left(\begin{array}{ccc|c}\\
-1 & 2 & 10 & 11\\ 
11 & -1 & 2 & 12\\
1 & 5 & 2 & 8
\end{array}\right)$

\item $A=\left(\begin{array}{ccc|c}\\
8 & 2 & 3 & 51\\
2 & 5 & 1 & 23\\
-3 & 1 & 6 & 20
\end{array}\right)$

\item $A=\left(\begin{array}{cccc|c}\\
2 & -1 & 1 & 3 & 10\\
2 & 2 & 2 & 2 & 1\\
-1 & -1 & 2 & 2 & -5\\
3 & 1 & -1 & 4 & 6
\end{array}\right)$

\item $A=\left(\begin{array}{cccc|c}\\
3 & 1 & 1 & -1 & 5\\
0 & 2 & 1 & 4 & 0\\
1 & 1 & -1 & 9 & 1\\
2 & 4 & 6 & 3 & 0
\end{array}\right)$

\item $A=\left(\begin{array}{cccc|c}\\
10 & -1 & 2 & 0 & 6\\
-1 & 11 & -1 & 3 & 25\\
2 & -1 & 10 & -1 & -11\\
0 & 2 & -1 & 8 & 15
\end{array}\right)$

\item \textbf{*}\begin{eqnarray*}
x_1+2x_2-2x_3&=&7\\
x_1+x_2+x_3&=&2\\
2x_1+2x_2+x_3&=&5
\end{eqnarray*}

\item \begin{eqnarray*}
-4x_1+14x_2=10\\
-5x_1+13x_2&=8\\
-x_1+2x_3&=&1
\end{eqnarray*}

\item \textbf{*}\begin{eqnarray*}
x+y+2z&=&1\\
x+2y+z&=&1\\
2x+y+z&=&1
\end{eqnarray*}

\item \begin{eqnarray*}
6x_1-2x_2+2x_3+4x_4&=&10\\
12x_1-8x_2+6x_3+10x_4&=&20\\
3x_1-13x_2+9x_3+3x_4&=&2\\
-6x_1+4x_2+x_3-18x_4&=&-19
\end{eqnarray*}


\end{enumerate}
\end{Ejer}

\begin{Ejer} La siguiente serie de ejercicios hay que resolverlos con el apoyo de R, excepto los indicados por un \textbf{*}

\begin{enumerate}
  \item \textbf{*} Resuelva el sistema
  \begin{eqnarray*}
  \begin{cases}
  2x+y-z=1,\\
  -x+3y+2z=12,\\
  x+2y+3z=7
  \end{cases}
  \end{eqnarray*}
  mediante eliminación gaussiana simple (sin pivoteo).

  \item Resuelva el mismo sistema anterior pero ahora aplicando eliminación gaussiana con pivoteo parcial. Compare los pasos con el ejercicio anterior.

  \item Aplique eliminación gaussiana con pivoteo y escalamiento al sistema
  \begin{eqnarray*}
  \begin{cases}
  10x+2y+z=7,\\
  2x+20y+2z=9,\\
  x+2y+30z=12
  \end{cases}
  \end{eqnarray*}
  y analice la importancia del escalamiento.

  \item \textbf{*}Utilice el método de Gauss--Jordan para calcular la inversa de
  \begin{eqnarray*}
  A=\begin{bmatrix}
  1 & 2 & 1\\
  0 & 1 & -1\\
  2 & 3 & 4
  \end{bmatrix}.
  \end{eqnarray*}

  \item \textbf{*}Resuelva $Ax=b$ con $A$ y $b$ dados por
  \begin{eqnarray*}
  A=\begin{bmatrix}
  4 & -2 & 1\\
  -2 & 4 & -2\\
  1 & -2 & 3
  \end{bmatrix},\qquad
  b=\begin{bmatrix}1\\4\\2\end{bmatrix},
  \end{eqnarray*}
  utilizando factorización $LU$ y sustitución hacia adelante y hacia atrás.

  \item Calcule la factorización de Cholesky de
  \begin{eqnarray*}
  A=\begin{bmatrix}
  25 & 15 & -5\\
  15 & 18 & 0\\
  -5 & 0 & 11
  \end{bmatrix}
  \end{eqnarray*}
  y resuelva $Ax=b$ con $b=(35,33,6)^\top$.
\end{enumerate}


\begin{enumerate}\setcounter{enumi}{6}
  \item \textbf{*}Resuelva mediante sustitución hacia atrás el sistema triangular superior:
  \begin{eqnarray*}
  \begin{cases}
  2x+3y-z=5,\\
  -y+2z=4,\\
  3z=6.
  \end{cases}
  \end{eqnarray*}

  \item \textbf{*}Resuelva mediante sustitución hacia adelante el sistema triangular inferior:
  \begin{eqnarray*}
  \begin{cases}
  x=3,\\
  2y+x=5,\\
  z-y+2x=10.
  \end{cases}
  \end{eqnarray*}

\end{enumerate}


\begin{enumerate}\setcounter{enumi}{9}
  \item Aplique el método de Jacobi para resolver
  \begin{eqnarray*}
  \begin{cases}
  10x-y+2z=6,\\
  -x+11y-z+3w=25,\\
  2x-y+10z-w=-11,\\
  3y-z+8w=15,
  \end{cases}
  \end{eqnarray*}
  realizando 3 iteraciones con $x^{(0)}=\mathbf{0}$.

  \item Repita el ejercicio anterior con el método de Gauss--Seidel. Compare la velocidad de convergencia con Jacobi.

  \item Aplique el método SOR con $\omega=1.25$ al mismo sistema y compare las tres trayectorias de convergencia.

  \item Escriba la matriz de iteración $T_J$ y el vector $c_J$ del método de Jacobi para el sistema
  \begin{eqnarray*}
  \begin{cases}
  4x+y=9,\\
  x+3y=7.
  \end{cases}
  \end{eqnarray*}
  Verifique si $\rho(T_J)<1$.

  \item Investigue experimentalmente en R cuál es el valor óptimo aproximado de $\omega$ para el método SOR en el sistema $4\times 4$ del ejercicio 10.
\end{enumerate}


\begin{enumerate}\setcounter{enumi}{14}
  \item Genere una matriz aleatoria simétrica definida positiva $5\times 5$ en R y resuelva $Ax=b$:
 \begin{itemize}
    \item[(a)] Usando factorización $LU$.  
    \item[(b)] Usando factorización de Cholesky.  
    \item[(c)] Usando Gauss--Seidel con 20 iteraciones.  
  \end{itemize}
  Compare el tiempo de cómputo y la precisión de cada método.
\end{enumerate}


\end{Ejer}



\begin{Ejer}
Genera un archivo tipo Rmd con las prácticas realizadas en el laboratorio, en las que deberá de desplegarse una barra lateral izquierda con un nivel de profundidad de 4: seccion, subseccion, subsubseccion y subsubsubseccion.
\end{Ejer}




\begin{Ejer}

Revisa en R los códigos para resolver sistemas de ecuaciones lineales, esto en un archivo tipo Rmd, mismo que deberás de entregar.



\end{Ejer}

