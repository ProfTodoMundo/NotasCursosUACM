%%%%%%%%%%%%%%%%%%%%%%%%%%%%%%%%%%%%%%%%%%%%%%%%%%%%%%%%%%%%%%
% DOCUMENTO COMPLETO DE MÉTODOS NUMÉRICOS PARA EDO
%%%%%%%%%%%%%%%%%%%%%%%%%%%%%%%%%%%%%%%%%%%%%%%%%%%%%%%%%%%%%%

\documentclass[12pt,a4paper]{article}

\usepackage[utf8]{inputenc}
\usepackage[spanish]{babel}
\usepackage{amsmath, amssymb, amsfonts}
\usepackage{geometry}
\usepackage{enumerate}
\usepackage{booktabs}
\usepackage{hyperref}
\usepackage{graphicx}
\usepackage{verbatim}

\geometry{
  a4paper,
  left=2cm,
  right=2cm,
  top=2cm,
  bottom=2cm
}

\title{Solución Numérica de Ecuaciones Diferenciales Ordinarias\\
\large Teoría, implementación numérica y ejercicios}
\author{Curso de Métodos Numéricos}
\date{}

\begin{document}

\maketitle
\tableofcontents
\newpage

%%%%%%%%%%%%%%%%%%%%%%%%%%%%%%%%%%%%%%%%%%%%%%%%%%%%%%%%%%%%%%
\section{Introducción}

Muchas ecuaciones diferenciales ordinarias (EDO) que aparecen en ingeniería, física, química, biología, economía y otras ciencias no tienen solución analítica cerrada. En tales casos recurrimos a \textbf{métodos numéricos} para aproximar la solución con una precisión controlada.

Consideraremos problemas de valor inicial (PVI) de la forma
\[
y' = f(t,y), \qquad y(t_0) = y_0.
\]

La idea general es aproximar la solución $y(t)$ en un conjunto discreto de puntos
\[
t_n = t_0 + n h, \qquad n = 0, 1, 2, \dots, N,
\]
donde $h>0$ es el \textbf{tamaño de paso}, y construir una sucesión $\{y_n\}$ que aproxime
\[
y_n \approx y(t_n).
\]

Un $h$ más pequeño suele producir mayor precisión pero requiere más cómputo; un $h$ demasiado grande puede llevar a errores grandes o incluso inestabilidades numéricas.

%%%%%%%%%%%%%%%%%%%%%%%%%%%%%%%%%%%%%%%%%%%%%%%%%%%%%%%%%%%%%%
\section{Método de Euler (explícito)}

\subsection{Esquema general}

Partimos del PVI
\[
y' = f(t,y), \qquad y(t_0) = y_0.
\]

La derivada se interpreta como razón de cambio:
\[
y'(t) \approx \frac{y(t+h)-y(t)}{h}.
\]

Aproximando $y'(t_n)$ por $f(t_n,y_n)$, se obtiene
\[
\frac{y_{n+1} - y_n}{h} \approx f(t_n,y_n),
\]
es decir,
\[
\boxed{
y_{n+1} = y_n + h\,f(t_n,y_n)
}
\]

Este es el \textbf{método de Euler explícito}.

\subsection{Ejemplo 1 (Euler): cálculo a mano}

Resolver numéricamente el PVI
\[
y' = y - t^2 + 1,\qquad y(0) = 0.5,
\]
en el intervalo $[0,1]$ con tamaño de paso $h = 0.2$.

Definimos
\[
f(t,y) = y - t^2 + 1.
\]

Los puntos de malla son:
\[
t_0 = 0.0,\quad t_1 = 0.2,\quad t_2 = 0.4,\quad t_3 = 0.6,\quad t_4 = 0.8,\quad t_5 = 1.0.
\]

\paragraph{Paso 0. Condición inicial.}
\[
t_0 = 0,\qquad y_0 = 0.5.
\]

\paragraph{Paso 1: de $t_0=0$ a $t_1=0.2$.}
\[
f(t_0,y_0) = f(0,0.5) = 0.5 - 0^2 + 1 = 1.5.
\]
\[
y_1 = y_0 + h f(t_0,y_0) = 0.5 + 0.2(1.5) = 0.5 + 0.3 = 0.8.
\]

\paragraph{Paso 2: de $t_1=0.2$ a $t_2=0.4$.}
\[
f(t_1,y_1) = f(0.2,0.8) = 0.8 - 0.2^2 + 1 = 0.8 - 0.04 + 1 = 1.76.
\]
\[
y_2 = y_1 + 0.2 f(t_1,y_1) = 0.8 + 0.2(1.76) = 0.8 + 0.352 = 1.152.
\]

\paragraph{Paso 3: de $t_2=0.4$ a $t_3=0.6$.}
\[
f(t_2,y_2) = f(0.4,1.152) = 1.152 - 0.4^2 + 1 = 1.152 - 0.16 + 1 = 1.992.
\]
\[
y_3 = 1.152 + 0.2(1.992) = 1.152 + 0.3984 = 1.5504.
\]

\paragraph{Paso 4: de $t_3=0.6$ a $t_4=0.8$.}
\[
f(t_3,y_3) = f(0.6,1.5504) = 1.5504 - 0.6^2 + 1 = 1.5504 - 0.36 + 1 = 2.1904.
\]
\[
y_4 = 1.5504 + 0.2(2.1904) = 1.5504 + 0.43808 = 1.98848.
\]

\paragraph{Paso 5: de $t_4=0.8$ a $t_5=1.0$.}
\[
f(t_4,y_4) = f(0.8,1.98848) = 1.98848 - 0.8^2 + 1 = 1.98848 - 0.64 + 1 = 2.34848.
\]
\[
y_5 = 1.98848 + 0.2(2.34848) = 1.98848 + 0.469696 = 2.458176.
\]

\paragraph{Tabla resumen (Euler).}
\[
\begin{array}{c|c|c|c}
n & t_n & f(t_n,y_n) & y_n \\ \hline
0 & 0.0 & 1.50000 & 0.50000 \\
1 & 0.2 & 1.76000 & 0.80000 \\
2 & 0.4 & 1.99200 & 1.15200 \\
3 & 0.6 & 2.19040 & 1.55040 \\
4 & 0.8 & 2.34848 & 1.98848 \\
5 & 1.0 &   -     & 2.45818 \\
\end{array}
\]

%%%%%%%%%%%%%%%%%%%%%%%%%%%%%%%%%%%%%%%%%%%%%%%%%%%%%%%%%%%%%%
\subsection{Implementación en R (Euler, versión manual)}

\begin{verbatim}
# Método de Euler para y' = y - t^2 + 1, y(0) = 0.5

f <- function(t, y) {
  y - t^2 + 1
}

t0 <- 0
y0 <- 0.5
T  <- 1
h  <- 0.2

t_vals <- seq(t0, T, by = h)
N <- length(t_vals)

y_vals <- numeric(N)
y_vals[1] <- y0  # condición inicial

for (n in 1:(N-1)) {
  t_n <- t_vals[n]
  y_n <- y_vals[n]
  y_vals[n+1] <- y_n + h * f(t_n, y_n)
}

data.frame(n = 0:(N-1),
           t = t_vals,
           y_aprox = y_vals)
\end{verbatim}

%%%%%%%%%%%%%%%%%%%%%%%%%%%%%%%%%%%%%%%%%%%%%%%%%%%%%%%%%%%%%%
\subsection{Implementación en R (Euler, versión automatizada)}

\begin{verbatim}
# Implementación genérica del método de Euler

euler <- function(f, t0, y0, T, h) {
  t_vals <- seq(t0, T, by = h)
  N <- length(t_vals)
  y_vals <- numeric(N)
  y_vals[1] <- y0
  
  for (n in 1:(N-1)) {
    y_vals[n+1] <- y_vals[n] + h * f(t_vals[n], y_vals[n])
  }
  
  data.frame(t = t_vals, y = y_vals)
}

# Ejemplo de uso:
f <- function(t, y) y - t^2 + 1
sol_euler <- euler(f = f, t0 = 0, y0 = 0.5, T = 1, h = 0.2)
\end{verbatim}

%%%%%%%%%%%%%%%%%%%%%%%%%%%%%%%%%%%%%%%%%%%%%%%%%%%%%%%%%%%%%%
\section{Método de Euler Mejorado (Heun)}

\subsection{Esquema general}

El método de Euler mejorado (Heun) utiliza dos evaluaciones de $f$ por paso.

\[
k_1 = f(t_n, y_n),
\]
\[
\tilde{y}_{n+1} = y_n + h k_1,
\]
\[
k_2 = f(t_n + h, \tilde{y}_{n+1}),
\]
\[
\boxed{
y_{n+1} = y_n + \frac{h}{2}(k_1 + k_2)
}
\]

Este método es de orden 2 en el error global, por lo que proporciona mayor precisión que Euler con el mismo tamaño de paso.

\subsection{Ejemplo 2 (Heun): cálculo a mano}

Resolvámos el PVI
\[
y' = y + t, \qquad y(0) = 1,
\]
en el intervalo $[0,0.3]$ con $h=0.1$.

Definimos
\[
f(t,y) = y + t.
\]

Los puntos de malla:
\[
t_0 = 0.0,\quad t_1 = 0.1,\quad t_2 = 0.2,\quad t_3 = 0.3.
\]

\paragraph{Paso 0. Condición inicial.}
\[
t_0 = 0,\qquad y_0 = 1.
\]

\paragraph{Paso 1: de $t_0=0$ a $t_1=0.1$.}
\[
k_1 = f(0, 1) = 1 + 0 = 1.
\]
\[
\tilde{y}_1 = y_0 + h k_1 = 1 + 0.1(1) = 1.1.
\]
\[
k_2 = f(0.1, 1.1) = 1.1 + 0.1 = 1.2.
\]
\[
y_1 = y_0 + \frac{h}{2}(k_1 + k_2)
    = 1 + 0.05(1 + 1.2)
    = 1 + 0.05(2.2)
    = 1.11.
\]

Los pasos siguientes se calculan de forma análoga.

%%%%%%%%%%%%%%%%%%%%%%%%%%%%%%%%%%%%%%%%%%%%%%%%%%%%%%%%%%%%%%
\subsection{Implementación en R (Heun, versión manual)}

\begin{verbatim}
# Método de Heun para y' = y + t, y(0) = 1

f <- function(t, y) {
  y + t
}

t0 <- 0
y0 <- 1
T  <- 0.3
h  <- 0.1

t_vals <- seq(t0, T, by = h)
N <- length(t_vals)
y_vals <- numeric(N)
y_vals[1] <- y0

for (n in 1:(N-1)) {
  t_n <- t_vals[n]
  y_n <- y_vals[n]
  
  k1 <- f(t_n, y_n)
  y_tilde <- y_n + h * k1
  k2 <- f(t_n + h, y_tilde)
  
  y_vals[n+1] <- y_n + (h/2) * (k1 + k2)
}

data.frame(n = 0:(N-1),
           t = t_vals,
           y_aprox = y_vals)
\end{verbatim}

%%%%%%%%%%%%%%%%%%%%%%%%%%%%%%%%%%%%%%%%%%%%%%%%%%%%%%%%%%%%%%
\subsection{Implementación automatizada en R (Heun)}

\begin{verbatim}
# Implementación genérica del método de Heun

heun <- function(f, t0, y0, T, h) {
  t_vals <- seq(t0, T, by = h)
  N <- length(t_vals)
  y_vals <- numeric(N)
  y_vals[1] <- y0
  
  for (n in 1:(N-1)) {
    t_n <- t_vals[n]
    y_n <- y_vals[n]
    
    k1 <- f(t_n, y_n)
    y_tilde <- y_n + h * k1
    k2 <- f(t_n + h, y_tilde)
    
    y_vals[n+1] <- y_n + (h/2) * (k1 + k2)
  }
  
  data.frame(t = t_vals, y = y_vals)
}

# Ejemplo de uso
f <- function(t, y) y + t
sol_heun <- heun(f = f, t0 = 0, y0 = 1, T = 0.3, h = 0.1)
\end{verbatim}

%%%%%%%%%%%%%%%%%%%%%%%%%%%%%%%%%%%%%%%%%%%%%%%%%%%%%%%%%%%%%%
\section{Método de Runge--Kutta de cuarto orden (RK4)}

\subsection{Esquema general}

El método de Runge--Kutta de cuarto orden (RK4) calcula cuatro pendientes:

\[
\begin{aligned}
k_1 &= f(t_n, y_n),\\
k_2 &= f\!\left(t_n+\frac{h}{2},\, y_n+\frac{h}{2}k_1\right),\\
k_3 &= f\!\left(t_n+\frac{h}{2},\, y_n+\frac{h}{2}k_2\right),\\
k_4 &= f\!\left(t_n+h,\, y_n+h k_3\right).
\end{aligned}
\]

La actualización es
\[
\boxed{
y_{n+1} = y_n + \frac{h}{6}(k_1 + 2k_2 + 2k_3 + k_4).
}
\]

Este método es de orden 4, con error global $O(h^4)$.

\subsection{Ejemplo 3 (RK4): cálculo a mano}

Consideremos
\[
y' = -3y, \qquad y(0) = 5,
\]
con $h=0.2$. El primer paso (de $t_0=0$ a $t_1=0.2$) se calcula como:

\[
k_1 = f(0,5) = -3\cdot 5 = -15,
\]
\[
k_2 = f\left(0.1, 5 + \frac{0.2}{2}(-15)\right)
    = f(0.1, 3.5) = -3\cdot 3.5 = -10.5,
\]
\[
k_3 = f\left(0.1, 5 + \frac{0.2}{2}(-10.5)\right)
    = f(0.1,3.95) = -3\cdot 3.95 = -11.85,
\]
\[
k_4 = f\left(0.2, 5 + 0.2(-11.85)\right)
    = f(0.2,2.63) = -3\cdot 2.63 = -7.89.
\]

\[
y_1 = 5 + \frac{0.2}{6}\bigl(k_1 + 2k_2 + 2k_3 + k_4\bigr).
\]

%%%%%%%%%%%%%%%%%%%%%%%%%%%%%%%%%%%%%%%%%%%%%%%%%%%%%%%%%%%%%%
\subsection{Implementación en R (RK4 manual)}

\begin{verbatim}
# Método RK4 para y' = -3y, y(0) = 5

f <- function(t,y) {
  -3*y
}

t0 <- 0
y0 <- 5
T  <- 0.6
h  <- 0.2

t_vals <- seq(t0, T, by=h)
N <- length(t_vals)
y_vals <- numeric(N)
y_vals[1] <- y0

for (n in 1:(N-1)) {
  t_n <- t_vals[n]
  y_n <- y_vals[n]
  
  k1 <- f(t_n, y_n)
  k2 <- f(t_n + h/2, y_n + h*k1/2)
  k3 <- f(t_n + h/2, y_n + h*k2/2)
  k4 <- f(t_n + h,   y_n + h*k3)
  
  y_vals[n+1] <- y_n + (h/6)*(k1 + 2*k2 + 2*k3 + k4)
}

data.frame(n = 0:(N-1),
           t = t_vals,
           y_aprox = y_vals)
\end{verbatim}

%%%%%%%%%%%%%%%%%%%%%%%%%%%%%%%%%%%%%%%%%%%%%%%%%%%%%%%%%%%%%%
\subsection{Implementación automatizada en R (RK4)}

\begin{verbatim}
# Implementación genérica del método RK4

rk4 <- function(f, t0, y0, T, h) {
  t_vals <- seq(t0, T, by=h)
  N <- length(t_vals)
  y_vals <- numeric(N)
  y_vals[1] <- y0
  
  for (n in 1:(N-1)) {
    t_n <- t_vals[n]
    y_n <- y_vals[n]
    
    k1 <- f(t_n, y_n)
    k2 <- f(t_n + h/2, y_n + h*k1/2)
    k3 <- f(t_n + h/2, y_n + h*k2/2)
    k4 <- f(t_n + h,   y_n + h*k3)
    
    y_vals[n+1] <- y_n + (h/6)*(k1 + 2*k2 + 2*k3 + k4)
  }
  
  data.frame(t = t_vals, y = y_vals)
}

# Ejemplo de uso
f <- function(t, y) -3*y
sol_rk4 <- rk4(f = f, t0 = 0, y0 = 5, T = 0.6, h = 0.2)
\end{verbatim}

%%%%%%%%%%%%%%%%%%%%%%%%%%%%%%%%%%%%%%%%%%%%%%%%%%%%%%%%%%%%%%
\section{Método de Euler implícito}

\subsection{Esquema general}

Para el PVI
\[
y' = f(t,y), \qquad y(t_0) = y_0,
\]
el método de Euler implícito define
\[
\boxed{
y_{n+1} = y_n + h\,f(t_{n+1}, y_{n+1}).
}
\]

Aquí $y_{n+1}$ aparece en ambos lados. En general, para $f$ no lineal en $y$, se debe resolver en cada paso la ecuación
\[
G(y_{n+1}) := y_{n+1} - h\,f(t_{n+1}, y_{n+1}) - y_n = 0
\]
mediante un método numérico (por ejemplo, Newton).

\subsection{Caso lineal: $f(t,y) = a(t)y + b(t)$}

Si
\[
f(t,y) = a(t)y + b(t),
\]
entonces
\[
y_{n+1} = y_n + h\,[a(t_{n+1}) y_{n+1} + b(t_{n+1})].
\]
Reordenando:
\[
y_{n+1}\bigl(1 - h a(t_{n+1})\bigr) = y_n + h b(t_{n+1}),
\]
\[
\boxed{
y_{n+1} = \frac{y_n + h b(t_{n+1})}{1 - h a(t_{n+1})}.
}
\]

\subsection{Ejemplo 4 (Euler implícito): cálculo a mano}

Consideremos la ecuación rígida
\[
y' = -50y + 10,\qquad y(0)=0,
\]
en el intervalo $[0,0.2]$ con $h=0.05$.

Aquí
\[
a(t) = -50,\qquad b(t) = 10.
\]

Entonces
\[
y_{n+1} = \frac{y_n + h\cdot 10}{1 - h(-50)}
= \frac{y_n + 0.5}{1 + 2.5}
= \frac{y_n + 0.5}{3.5}.
\]

Puntos de malla:
\[
t_0 = 0.00,\quad t_1 = 0.05,\quad t_2 = 0.10,\quad t_3 = 0.15,\quad t_4 = 0.20.
\]

\paragraph{Paso 0. Condición inicial.}
\[
y_0 = 0.
\]

\paragraph{Paso 1.}
\[
y_1 = \frac{0 + 0.5}{3.5} \approx 0.142857.
\]

\paragraph{Paso 2.}
\[
y_2 = \frac{y_1 + 0.5}{3.5}
    = \frac{0.142857 + 0.5}{3.5}
    \approx 0.183673.
\]

\paragraph{Paso 3.}
\[
y_3 = \frac{0.183673 + 0.5}{3.5}
    \approx 0.195335.
\]

\paragraph{Paso 4.}
\[
y_4 = \frac{0.195335 + 0.5}{3.5}
    \approx 0.198667.
\]

\paragraph{Tabla resumen (Euler implícito).}
\[
\begin{array}{c|c|c}
n & t_n & y_n \\ \hline
0 & 0.00 & 0.000000 \\
1 & 0.05 & 0.142857 \\
2 & 0.10 & 0.183673 \\
3 & 0.15 & 0.195335 \\
4 & 0.20 & 0.198667 \\
\end{array}
\]

%%%%%%%%%%%%%%%%%%%%%%%%%%%%%%%%%%%%%%%%%%%%%%%%%%%%%%%%%%%%%%
\subsection{Implementación en R (Euler implícito, versión manual)}

\begin{verbatim}
# Euler implícito para y' = -50y + 10, y(0) = 0, h = 0.05

h <- 0.05
t_vals <- seq(0, 0.2, by = h)
N <- length(t_vals)

y_vals <- numeric(N)
y_vals[1] <- 0  # y0

for (n in 1:(N-1)) {
  y_vals[n+1] <- (y_vals[n] + 0.5) / 3.5
}

data.frame(n = 0:(N-1),
           t = t_vals,
           y_aprox = y_vals)
\end{verbatim}

%%%%%%%%%%%%%%%%%%%%%%%%%%%%%%%%%%%%%%%%%%%%%%%%%%%%%%%%%%%%%%
\subsection{Implementación automatizada en R (Euler implícito lineal)}

\begin{verbatim}
# Euler implícito para y' = a(t)*y + b(t)

euler_imp_lineal <- function(a_fun, b_fun, t0, y0, T, h) {
  t_vals <- seq(t0, T, by = h)
  N <- length(t_vals)
  y_vals <- numeric(N)
  y_vals[1] <- y0
  
  for (n in 1:(N-1)) {
    t_next <- t_vals[n+1]
    a_next <- a_fun(t_next)
    b_next <- b_fun(t_next)
    
    # y_{n+1} = (y_n + h*b_next) / (1 - h*a_next)
    y_vals[n+1] <- (y_vals[n] + h * b_next) / (1 - h * a_next)
  }
  
  data.frame(t = t_vals, y = y_vals)
}

# Ejemplo: y' = -50y + 10
a_fun <- function(t) -50
b_fun <- function(t) 10

sol_imp <- euler_imp_lineal(a_fun = a_fun,
                            b_fun = b_fun,
                            t0 = 0, y0 = 0,
                            T = 0.2, h = 0.05)
\end{verbatim}

%%%%%%%%%%%%%%%%%%%%%%%%%%%%%%%%%%%%%%%%%%%%%%%%%%%%%%%%%%%%%%
\section{Ejercicios propuestos}

\subsection{Ejercicios para el método de Euler}

Resuelva numéricamente con Euler, construyendo la tabla de iteraciones y, si es posible, comparando con la solución exacta.

\begin{enumerate}
  \item $y' = y - t$, \quad $y(0) = 1$, \quad $h = 0.1$, \quad intervalo $[0,1]$.
  \item $y' = 3t^2 - y$, \quad $y(0) = 2$, \quad $h = 0.05$, \quad intervalo $[0,0.5]$.
  \item $y' = \sin(t) - y^2$, \quad $y(0) = 0$, \quad $h = 0.1$, \quad intervalo $[0,1]$.
  \item $y' = t y$, \quad $y(0) = 1$, \quad $h = 0.2$, \quad intervalo $[0,1]$.
  \item $y' = e^{-t} - y$, \quad $y(0) = 3$, \quad $h = 0.1$, \quad intervalo $[0,1]$.
  \item $y' = y(1-y)$ (modelo logístico), \quad $y(0) = 0.2$, \quad $h = 0.1$, \quad intervalo $[0,2]$.
  \item $y' = 4y + t$, \quad $y(0) = 1$, \quad $h = 0.05$, \quad intervalo $[0,0.5]$.
  \item $y' = \cos(t) + y$, \quad $y(0) = 0$, \quad $h = 0.1$, \quad intervalo $[0,1]$.
  \item $y' = -5y$, \quad $y(0) = 2$, \quad $h = 0.2$, \quad intervalo $[0,1]$.
  \item $y' = t^3 - y^2$, \quad $y(0) = 1$, \quad $h = 0.05$, \quad intervalo $[0,0.5]$.
\end{enumerate}

%%%%%%%%%%%%%%%%%%%%%%%%%%%%%%%%%%%%%%%%%%%%%%%%%%%%%%%%%%%%%%
\subsection{Ejercicios para el método de Heun}

Aplique el método de Euler mejorado (Heun), mostrando los valores de $k_1$ y $k_2$ en cada paso.

\begin{enumerate}
  \item $y' = y + t^2$, \quad $y(0) = 1$, \quad $h = 0.1$, \quad intervalo $[0,1]$.
  \item $y' = \cos(t) - y$, \quad $y(0) = 0$, \quad $h = 0.1$, \quad intervalo $[0,1]$.
  \item $y' = t - y^2$, \quad $y(0) = 0.5$, \quad $h = 0.05$, \quad intervalo $[0,0.5]$.
  \item $y' = 2y + e^t$, \quad $y(0) = 1$, \quad $h = 0.1$, \quad intervalo $[0,1]$.
  \item $y' = y(1+t)$, \quad $y(0) = 1$, \quad $h = 0.2$, \quad intervalo $[0,1]$.
  \item $y' = -3y + t$, \quad $y(0) = 2$, \quad $h = 0.1$, \quad intervalo $[0,1]$.
  \item $y' = t^2\sqrt{y}$ (suponer $y\ge 0$), \quad $y(0) = 1$, \quad $h = 0.1$, \quad intervalo $[0,1]$.
  \item $y' = y^2 - y$, \quad $y(0) = 0.1$, \quad $h = 0.1$, \quad intervalo $[0,1]$.
  \item $y' = 1 - t y$, \quad $y(0) = 1$, \quad $h = 0.1$, \quad intervalo $[0,1]$.
  \item $y' = 5 - 4y$, \quad $y(0) = 0$, \quad $h = 0.1$, \quad intervalo $[0,1]$.
\end{enumerate}

%%%%%%%%%%%%%%%%%%%%%%%%%%%%%%%%%%%%%%%%%%%%%%%%%%%%%%%%%%%%%%
\subsection{Ejercicios para el método RK4}

Use el método de Runge--Kutta de cuarto orden, calculando explícitamente $k_1, k_2, k_3, k_4$.

\begin{enumerate}
  \item $y' = y - t^2 + 1$, \quad $y(0) = 0.5$, \quad $h = 0.2$, \quad intervalo $[0,1]$.
  \item $y' = -2y + \sin(t)$, \quad $y(0) = 1$, \quad $h = 0.1$, \quad intervalo $[0,1]$.
  \item $y' = y^2 + t$, \quad $y(0) = 0$, \quad $h = 0.05$, \quad intervalo $[0,0.5]$.
  \item $y' = e^{-t} - y$, \quad $y(0) = 2$, \quad $h = 0.1$, \quad intervalo $[0,1]$.
  \item $y' = 3y(1-y)$ (modelo logístico), \quad $y(0) = 0.1$, \quad $h = 0.1$, \quad intervalo $[0,2]$.
  \item $y' = t\cos(y)$, \quad $y(0) = 0$, \quad $h = 0.1$, \quad intervalo $[0,1]$.
  \item $y' = -5y + 5$, \quad $y(0) = 0$, \quad $h = 0.1$, \quad intervalo $[0,1]$.
  \item $y' = \sqrt{t + y}$ (suponer $t+y \ge 0$), \quad $y(0) = 1$, \quad $h = 0.1$, \quad intervalo $[0,1]$.
  \item $y' = y\ln(t+1)$, \quad $y(0) = 1$, \quad $h = 0.2$, \quad intervalo $[0,2]$.
  \item $y' = -50y + \cos(t)$, \quad $y(0) = 0$, \quad $h = 0.02$, \quad intervalo $[0,0.4]$.
\end{enumerate}

%%%%%%%%%%%%%%%%%%%%%%%%%%%%%%%%%%%%%%%%%%%%%%%%%%%%%%%%%%%%%%
\end{document}