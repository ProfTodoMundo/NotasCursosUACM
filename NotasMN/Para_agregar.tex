\subsection{Fuentes de error}




Para ello, se introduce el concepto de \textbf{cifras significativas}, que designa formalmente la notaci\'on de un valor num\'erico, y se usa en aquellos que gu\'ian visualmente la precisi\'on.
\begin{itemize}
  \item Los \textbf{errores de truncamiento} se producen cuando utilizamos una aproximaci\'on en lugar de un procedimiento matem\'atico exacto. Para conocer las caracter\'isticas de estos errores se suelen considerar los polinomios de Taylor, que se utilizan ampliamente en los m\'etodos num\'ericos para expresar funciones de manera aproximada.  
  Una gran cantidad de m\'etodos num\'ericos est\'an basados en la utilizaci\'on de unos pocos t\'erminos de los polinomios de Taylor para aproximar una funci\'on.

  Cuando se aproxima un proceso continuo por uno discreto, para errores provocados por un tama\~no de paso finito \( h $, resulta a menudo \'util describir la dependencia del error \( e $ con \( h $ cuando \( h $ tiende a cero.

  Decimos que una funci\'on \( f(h) $ es una \( \mathcal{O} $ grande de \( h^n $ si \( |f(h)| \leq c h^n $ para alguna constante \( c $, cuando \( h $ es pr\'oximo a cero; se escribe \( f(h) = \mathcal{O}(h^n) $.

  Si un m\'etodo tiene un t\'ermino de error que es \( \mathcal{O}(h^n) $, se suele decir que es un \textbf{m\'etodo de orden \( n $}.  
  Por ejemplo, si utilizamos un polinomio de Taylor para aproximar la funci\'on \( g $ en \( x = a + h $, tenemos:
  $$
  g(x) = g(a + h) = g(a) + h g'(a) + \frac{h^2}{2!} g''(a) + \frac{h^3}{3!} g^{(3)}(\xi), \quad \text{para alg\'un } \xi \in [a, a+h].
  $$

  Suponiendo que \( g $ es suficientemente derivable, la aproximaci\'on anterior es \( \mathcal{O}(h^3) $, puesto que el error,
  $$
  \frac{h^3}{3!} g^{(3)}(\xi), \quad \text{satisface} \quad \left| \frac{h^3}{3!} g^{(3)}(\xi) \right| \leq \frac{M}{3!} |h^3|,
  $$
  donde \( M $ es el m\'aximo de \( g^{(3)}(x) $ para \( x \in [a, a+h] $.
\end{itemize}

El error de truncamiento depende del m\'etodo num\'erico utilizado para resolver un problema, es independiente del error de redondeo y se produce incluso cuando las operaciones matem\'aticas son exactas.

\begin{itemize}
  \item La suma de los errores de redondeo y de truncamiento es lo que habitualmente llamamos \textbf{error num\'erico total} (tambi\'en llamado \textit{error verdadero}), que est\'a incluido en la soluci\'on num\'erica.  
  En general, si se incrementa el n\'umero de cifras significativas en el ordenador, se minimizan los errores de redondeo, y los errores de truncamiento disminuyen a medida que los errores de redondeo se incrementan.  
  Por lo tanto, para disminuir uno de los dos sumandos del error total debemos incrementar el otro.  
  El reto consiste en identificar d\'onde los errores de redondeo no muestran los beneficios de la reducci\'on del error de truncamiento y determinar el tama\~no del incremento apropiado para un c\'alculo en particular.  
  \'estas situaciones son poco comunes en casos reales porque habitualmente los ordenadores utilizan suficientes cifras significativas como para que los errores de redondeo no predominen.

  \item Como el error total no se puede calcular en la mayor\'ia de los casos, se suelen utilizar otras medidas para estimar la exactitud de un m\'etodo num\'erico, que suelen depender del m\'etodo espec\'ifico.  
  En algunos m\'etodos el error num\'erico se puede acotar, mientras que en otros se determina una estimaci\'on del orden de magnitud del error.

  \item Terminamos recordando que cuando buscamos las soluciones num\'ericas de un problema real los resultados que obtenemos generalmente no son exactos.  
  Una fuente habitual de inexactitudes radica en la simplificaci\'on del modelo del problema original.  
  Tambi\'en suelen aparecer errores a la hora de interpretar una condici\'on de datos.  
  Adem\'as, habr\'a que estar atentos a los errores que no est\'an relacionados directamente con los m\'etodos num\'ericos aplicados, como por ejemplo, entre otros, equivocaciones, errores de formulaci\'on o del modelo, e incertidumbre en la obtenci\'on de datos.
\end{itemize}

\subsection{Representaci\'on de los n\'umeros reales en el ordenador}

La utilizaci\'on de ordenadores hoy en d\'ia juega un papel fundamental en el desarrollo de los m\'etodos num\'ericos, ya que nos permiten resolver problemas que no ser\'iamos capaces de hacerlo sin ellos.  
Sin embargo, como los ordenadores est\'an programados para que se puedan utilizar con una gran exactitud, hay que tener en cuenta algunas consecuencias de \textbf{c\'omo realizan los c\'alculos}, como consecuencia de que pueden conducir a efectos inesperados en los resultados.

\textbf{Toda operaci\'on} que realiza un ordenador (\textit{operaci\'on m\'aquina}) est\'a sujeta a errores de redondeo, que aparecen como consecuencia de que en un ordenador no se puede representar m\'as que un subconjunto finito del conjunto de los n\'umeros reales.
Mientras que el conjunto de los n\'umeros reales \( \mathbb{R} $ es conocido, la manera en la que los ordenadores los tratan es menos conocida.  
Por una parte, como los ordenadores tienen recursos limitados, s\'olo se puede representar un subconjunto de \( \mathbb{R} $ de dimensi\'on finita.  
Denotamos, y as\'i como se hace en \cite{ref23}, este subconjunto por \( F $, que est\'a formado por n\'umeros que se llaman \textbf{n\'umeros de punto flotante}.  
Por otra parte, como veremos posteriormente, \( F $ est\'a caracterizado por la base de la representaci\'on, por ejemplo, si la base adoptada por el ordenador es 2, entonces el n\'umero \( 0.1 $ no puede ser representado (en la base binaria) como \( 0.1_2 $.  
Por tanto, el conjunto \( F $ est\'a totalmente caracterizado por la base \( \beta $, el n\'umero de cifras significativas \( t $, el exponente \( L $, con \( 0 \leq L \leq J - 1 $ y la cantidad de la fracci\'on.  
Un n\'umero flotante \( x \in F $ se escribe como:

$$
x = (-1)^s \cdot (0.a_1a_2 \ldots a_t)_\beta \cdot \beta^e = (-1)^s \cdot m \cdot \beta^e, \quad a_1 \neq 0,
$$
donde \( s \in \{0, 1\} $ es el signo, \( \beta $ (un entero positivo mayor o igual que 2) es la base adoptada por el ordenador espec\'ifico que estemos manejando, \( m $ es un entero llamado \textit{mantisa} cuya \textit{longitud} \( t $ es el m\'aximo n\'umero de cifras \( a_i $, \( (0 \leq a_i < \beta, 1 \leq i \leq t) $, y \( e $ es un n\'umero entero limitado entre \( -L $ y \( L $.

\paragraph{Ejemplo de codificaci\'on:}

Si tomamos \( \beta = 2 $, \( t = 3 $, y \( L = 1 $, el conjunto de todos los n\'umeros que pueden representarse ser\'a:

$$
F = \left\{ \pm 0.10 \cdot 2^{-1}, \ \pm 0.11 \cdot 2^{-1}, \ \pm 0.10 \cdot 2^{0}, \ \pm 0.11 \cdot 2^{0}, \ \pm 0.10 \cdot 2^{1}, \ \pm 0.11 \cdot 2^{1} \right\}.
$$

Este conjunto contiene 12 n\'umeros con exponente limitado entre \( -1 $ y \( 1 $ y mantisas de tres cifras binaria significativas.  
Notamos que el 1 no puede ser representado de forma exacta con esta configuraci\'on, ya que no pertenece al conjunto generado.

\paragraph{Errores en punto flotante}

Un n\'umero real \( x \in \mathbb{R} $, cuando no pertenece a \( F $, se representa mediante una aproximaci\'on flotante \( fl(x) \in F $, de modo que:

$$
x = fl(x)(1 + \varepsilon), \quad |\varepsilon| \leq \epsilon_{\text{mach}}.
$$

Este \( \varepsilon $ se llama \textbf{error relativo} y \( \epsilon_{\text{mach}} $ es la \textbf{precisi\'on de la m\'aquina}.  
En general se tiene que:

$$
\epsilon_{\text{mach}} = \frac{1}{2} \beta^{1 - t}.
$$

\paragraph{Sobre las operaciones en punto flotante}

Puesto que \( F $ es un subconjunto de \( \mathbb{R} $, las operaciones algebraicas elementales sobre n\'umeros de punto flotante no gozan de las propiedades de los n\'umeros reales en su conjunto.  
En consecuencia, la \textbf{asociatividad} y la \textbf{conmutatividad} no se cumplen estrictamente.  
Por ejemplo, si consideramos la funci\'on \( f(x,y) = x + y $, evaluada en \( F $, podr\'iamos obtener valores distintos dependiendo del orden de evaluaci\'on.  
Tambi\'en puede ocurrir la \textbf{cancelaci\'on catastr\'ofica} cuando se restan dos n\'umeros muy cercanos.

\subsection{Almacenamiento de un n\'umero en la memoria del ordenador}

Los ordenadores almacenan los n\'umeros reales en forma binaria (base 2).  
Cada d\'igito binario (0 \'o 1) se llama \textbf{bit} (por digital binary).  
La memoria de los ordenadores est\'a organizada en \textbf{bytes}, siendo un byte = 8 bits.  
En el est\'andar IEEE, los ordenadores representan n\'umeros reales en precisi\'on simple (32 bits) y en precisi\'on doble (64 bits).  
Este est\'andar tambi\'en define los c\'odigos para representar valores especiales como NaN (Not a Number), infinitos, y ceros con signo.

Para representar un n\'umero real, se utiliza la forma normalizada:
$$
x = (-1)^s \cdot (1 + f) \cdot 2^e,
$$
donde:
- \( s $: es el bit de signo (0 para positivo, 1 para negativo),
- \( f $: es la fracci\'on,
- \( e $: es el exponente con sesgo.

Por ejemplo, el n\'umero 0.15625 en binario es 0.00101, que se escribe como \( 1.01 \cdot 2^{-3} $ y se almacena como:
- signo = 0,
- exponente = 124 (con sesgo de 127),
- mantisa = 010000...

En este formato, la precisi\'on depende de la cantidad de bits reservados a la fracci\'on \( f $.  
En doble precisi\'on (64 bits), se reservan 52 bits para la fracci\'on, 11 para el exponente y 1 para el signo.
\subsection{Estabilidad y convergencia}

La estabilidad podemos decir que aparece en todo el c\'alculo num\'erico.  
Como acabamos de ver, el simple hecho de utilizar aritm\'etica finita para representar los n\'umeros reales introduce errores.  
Por esa raz\'on, lo importante no es enfocarse por eliminar los errores, sino m\'as bien ser capaces de controlarlos si se dan.

Esto significa en particular, dise\~nar los m\'etodos de forma que el error cometido por introducir errores en los datos del problema no crezca de forma desproporcionada al resolverlo.

A lo largo de este texto se considerar\'an diversos problemas de aproximaci\'on a una secuencia, en cada caso, basada en la iteraci\'on de alg\'un algoritmo.  
Llamaremos \textbf{estabilidad} precisamente a la propiedad que tiene un m\'etodo iterativo de producir una secuencia que se aproxima a la misma.  
Daremos m\'as adelante una serie de condiciones que nos permitan capturar esta propiedad (aunque para que no se tornen condiciones t\'ecnicas ser\'an aplicadas s\'olo en ciertos contextos).

\textbf{Convergencia} de un m\'etodo se refiere a que sea posible la obtenci\'on del valor buscado cuando el n\'umero de pasos tiende a infinito.  
T\'ipicamente, la convergencia se analiza en m\'etodos iterativos, es decir, aquellos en los que el resultado final se obtiene tras una repetici\'on de c\'alculos.  
Cuando repetimos estas iteraciones, los datos iniciales producen valoraciones progresivas del resultado.

A menudo nos encontraremos con m\'etodos que convergen lentamente, otros que lo hacen muy r\'apidamente, otros que no convergen.  
Se dir\'a que un m\'etodo tiene \textbf{orden de convergencia} \( \mathcal{O}(h^p) $ cuando existe una constante \( c > 0 $ tal que:
$$
|x_n - x| \leq c h^p,
$$
donde \( h $ es un par\'ametro del algoritmo, y \( x_n $ es la aproximaci\'on generada por el m\'etodo a la soluci\'on exacta \( x $.

Existen m\'etodos cuya convergencia no es uniforme y se aplicar\'an un criterio en cada uno de los aspectos num\'ericos, ya sea el redondeo, el truncamiento, el escalado, el orden de convergencia, entre otros.

\textit{Ejemplo:} Si un m\'etodo produce una sucesi\'on de aproximaciones \( \{x_n\} $ que converge a una cantidad \( x $, y adem\'as existe una constante \( c > 0 $ tal que:
$$
|x_n - x| \leq c h^p,
$$
se dice que tiene \textbf{orden de convergencia} \( \mathcal{O}(h^p) $.  
Esto significa que al disminuir \( h $, el error \( |x_n - x| $ tambi\'en disminuye como \( h^p $.  
En general, se dice que un m\'etodo tiene mayor orden si \( p $ es mayor.

\subsection{Coste computacional y eficiencia}

El coste computacional de un algoritmo es el \textbf{n\'umero de operaciones} que se requieren en una situaci\'on precisa en formato de texto fluido, que especifica la ejecuci\'on de una lista finita de operaciones elementales.  
Estaremos interesados en el n\'umero de pasos que necesita un algoritmo para completar una tarea.

El an\'alisis del coste nos ayudar\'a a comparar la eficiencia de distintos algoritmos.  
Este an\'alisis se puede realizar experimentalmente (por ejemplo, midiendo el tiempo de ejecuci\'on), o bien te\'oricamente (por ejemplo, contando el n\'umero de operaciones en punto flotante que se pueden efectuar en un segundo, \textit{flops: floating operations}).

Aunque no siempre se conoce el algoritmo m\'as eficiente para cada tarea, existen m\'etodos que minimizan el coste en comparaci\'on con otros.  
Un ejemplo cl\'asico es la multiplicaci\'on de matrices: el algoritmo convencional tiene coste c\'ubico \( \mathcal{O}(n^3) $, mientras que otros algoritmos pueden reducir este coste a \( \mathcal{O}(n^{2.376}) $.

En este curso nos enfocaremos en m\'etodos eficientes, es decir, eficientes con respecto a su coste computacional.  
Para ello se tomar\'an en cuenta aspectos como el n\'umero de operaciones, consumo de memoria, estabilidad, precisi\'on y complejidad de implementaci\'on.  
Cuando haya m\'ultiples alternativas v\'alidas, el algoritmo que minimice el coste ser\'a el preferido.

\section{Sugerencias para seguir leyendo}

\begin{itemize}
  \item Los m\'etodos de interpolaci\'on y ajuste de curvas pueden consultarse en \cite{ref10}.
  \item Las t\'ecnicas de integraci\'on num\'erica est\'an bien tratadas en \cite{ref16}.
  \item M\'etodos iterativos para resolver sistemas de ecuaciones lineales est\'an disponibles en \cite{ref14}.
  \item Para una revisi\'on del an\'alisis de estabilidad, puede consultarse \cite{ref12}, donde se tratan los conceptos fundamentales.
\end{itemize}
\section{Ejercicios (continuaci\'on)}

\begin{enumerate}
\setcounter{enumi}{4}

\item ¿Cu\'antos n\'umeros de punto flotante hay entre sucesivas potencias de 2?

\item Determine el n\'umero en punto flotante \( fl(x) $ y calcule los errores de redondeo absoluto y relativo que se cometen al representar \( 4/5 $ por \( fl(4/5) $.

\item Determine el n\'umero en punto flotante \( fl(x) $ utilizando redondeo a 10 cifras significativas.

\item Dado los siguientes valores de \( x $ y \( y $:
$$
a) \; x = \sqrt{5} \approx 2.236, \quad b) \; x = 1.9, y = 0.2, \quad c) \; \frac{\sqrt{5}}{3}, \quad d) \; x = 0.022, \quad e) \; A = 10^{9}, y = 1.11
$$
determine los errores absoluto y relativo cuando aproximamos \( x $ por \( fl(x) $.

\item Determine cu\'antos n\'umeros diferentes se pueden representar si \( p = 3, E_{\min} = -1, E_{\max} = 2 $.

\item Determine cu\'antas muestras representan el conjunto \( F(2,2,-2,2) $ y cu\'antas de ellas son pares con signo opuesto.

\item Realice un programa que permita visualizar el est\'andar IEEE-754, en decimal y binario.

\item Explique ejemplos en los que se aprecie de manifiesto que el producto de n\'umeros en punto flotante no es conmutativo. ¿En qu\'e se diferencia la multiplicaci\'on en punto flotante de la multiplicaci\'on exacta?

\item La ecuaci\'on \( x^2 - 10.0001x + 0.01 = 0 $ tiene dos soluciones exactas; \( x_1 = 10000 $ y \( x_2 = 0.0001 $. Observe que al calcular con 6 cifras significativas, la f\'ormula general de ra\'ices flotantes da dos soluciones:
$$
x_1 = \frac{-b + \sqrt{b^2 - 4ac}}{2a}, \quad x_2 = \frac{-b - \sqrt{b^2 - 4ac}}{2a}
$$
que contienen el error de cancelaci\'on. ¿Cu\'al es la causa del error?

\item Considere la ecuaci\'on cuadr\'atica \( ax^2 + bx + c = 0 $. ¿Qu\'e ocurre? Calcule a continuaci\'on la soluci\'on \( x_1 $ mediante la expresi\'on equivalente:
$$
x_1 = \frac{2c}{-b - \sqrt{b^2 - 4ac}}
$$

\item ¿Qu\'e conclusi\'on se puede sacar?

\item Justifique por qu\'e se puede describir una sucesi\'on \( a_n = \left( \frac{1}{n} \right)^k, \; n \in \mathbb{N} $ de forma recursiva mediante \( a_0 = 1 $, \( a_{n+1} = a_n - r_n $, para \( n \in \mathbb{N} $, utilizando redondeo a 6 cifras significativas. 
\newline ¿Los errores se magnifican, cancelan o se acumulan?

\item Sea la sucesi\'on \( x_n = \frac{1}{n} $, y \( y_n = \frac{1}{n^2} $. Repres\'entelas. A partir de la sucesi\'on \( x_n $, y la sucesi\'on en diferencias \( x_{n+1} - x_n $, analiza su comportamiento.

\item La sucesi\'on de Fibonacci se define por:
$$
a_0 = 0, \quad a_1 = 1, \quad a_{n+1} = a_n + a_{n-1}, \quad n \in \mathbb{N}
$$
Demuestre que la sucesi\'on \( \left( \frac{a_{n+1}}{a_n} \right)_{n \in \mathbb{N}} $ converge a \( \frac{1 + \sqrt{5}}{2} $.

\item Dadas las sucesiones \( a_n = \frac{1}{n}, \quad b_n = \frac{(-1)^n}{n}, \quad n \in \mathbb{N} $, demu\'estrese que la sucesi\'on \( (a_n) $ converge a 0 m\'as r\'apidamente que lo hace la sucesi\'on \( (b_n) $.

\end{enumerate}
Un \textit{m\'etodo num\'erico} es un proceso matem\'atico \textit{iterativo} cuyo objetivo es encontrar la aproximaci\'on a una soluci\'on espec\'ifica con un cierto error previamente determinado.

A diferencia de las t\'ecnicas propias de la matem\'atica anal\'itica, los m\'etodos num\'ericos requieren de una aproximaci\'on a la soluci\'on real al problema, misma que es corregida a trav\'es de la repetici\'on de un cierto proceso que debe arrojar soluciones cada vez m\'as cercanas al valor real. Cada correcci\'on de un valor inicial se conoce como \textit{iteraci\'on}. El proceso es controlado por medio de la medici\'on de una cantidad de error predefinido entre dos aproximaciones sucesivas.

No existe unanimidad entre los expertos sobre si \textit{An\'alisis num\'erico} es un sin\'onimo de \textit{m\'etodos num\'ericos}. Algunos consideran que los m\'etodos num\'ericos son procesos con objetivos particulares que conforman un proceso m\'as complejo, espec\'ificamente de interpretaci\'on de los resultados al que denominan \textit{An\'alisis num\'erico}.

Resulta complicado tomar partido por alguna de las dos posturas anteriores en la consideraci\'on de que la aplicaci\'on de los procesos iterativos suele hacerse a problemas reales, con condiciones de dise\~no muy espec\'ificas, por lo que no se puede establecer una regla general para hacer un an\'alisis. En el caso que nos ocupa, dado que se har\'a una presentaci\'on te\'orica para definir cada proceso, se optar\'a por llamarlo \textit{m\'etodo num\'erico}.


\section{Introducci\'on hist\'orica de los m\'etodos num\'ericos}

La historia de los m\'etodos num\'ericos es la colecci\'on de acontecimientos matem\'aticos en los que se resuelven problemas sin el uso de la matem\'atica anal\'itica \cite{Finkelshtein}.

Algunos de los m\'etodos m\'as utilizados en la actualidad fueron creados mucho antes de la invenci\'on de la computadora; su aplicaci\'on era extenuante y complicada porque cada iteraci\'on requer\'ia de una diversidad de operaciones aritm\'eticas que se realizaban por grupos enteros de calculistas, evidentemente, de forma manual.

Todos los enterados en la materia estar\'an de acuerdo en que una computadora realiza una gran cantidad de operaciones en un intervalo muy peque\~no; las super computadoras lo hacen pero en forma paralela. Esta capacidad es la que ha dado un sentido de aplicaci\'on a los m\'etodos num\'ericos.

Por lo anterior, la historia de los m\'etodos num\'ericos es paralela, al menos desde la mitad del siglo XIX, a la historia de la computaci\'on. Las contribuciones m\'as actuales radican en la creaci\'on de software que minimiza los errores y mejora las aproximaciones de los resultados.

Esta es una relaci\'on de hechos que han marcado la historia de los m\'etodos num\'ericos y se recomienda al lector, de acuerdo a su inter\'es, profundizar en el t\'opico que le resulte de su inter\'es; en particular, la obra \textit{Los Innovadores} \cite{Isaacson2014} de Walter Isaacson ofrece un panorama muy amplio al respecto.

\begin{itemize}
    \item 1650 a.C. Se crean los Papiros de Rhynd en los que se escribe un m\'etodo para resolver expresiones matem\'aticas sin \'algebra.
    \item 250 a.C. Euclides crea el M\'etodo de Exhausti\'on, que consiste en aproximar figuras geom\'etricas (tri\'angulos, cuadrados, pent\'agonos, etc.) consecutivamente dentro de un c\'irculo para obtener una aproximaci\'on a $\pi$.
    \item Siglo IX d.C. Al Juarismi crea los \textit{algoritmos}.
    \item 1623. John Napier inventa los \textit{huesos de Napier}, que son arreglos pr\'acticos de logaritmos en tablas.
    \item Siglo XVII. Isaac Newton crea los procesos de interpolaci\'on polinomial.
    \item Siglo XVIII. Leibnitz crea el C\'alculo diferencial.
    \item 1768. Euler crea soluciones aproximadas a ecuaciones diferenciales con el principio de la integraci\'on num\'erica. Jacob Stirling y Brook Taylor presentan el C\'alculo de diferencias finitas.
    \item 1822. Charles Babbage inventa la \textit{M\'aquina diferencial}.
    \item 1843. Ada, condesa de Lovelace, publica sus notas sobre la m\'aquina anal\'itica de Charles Babbage.
    \item 1890. (IBM) Tabula el censo estadounidense empleando las m\'aquinas de tarjetas perforadas de Herman Hollerith.
    \item 1931. Vannebar Bush dise\~na el analizador diferencial, un computador anal\'ogico electromec\'anico. En 1945 publicar\'a el art\'iculo \textit{C\'omo podremos pensar} en el que describe la computaci\'on personal.
\end{itemize}
\begin{itemize}
    \item 1937. Alan Turing publica \textit{Sobre los n\'umeros computables}, en el que describe un computador universal. En este mismo a\~no, Howard Aiken propone la construcci\'on de un gran computador y descubre partes de la m\'aquina diferencial de Babbage en Harvard; tambi\'en John Vincent Atanasoff conceptualiza el computador electr\'onico (la cual completar\'a en 1939).
    
    \item 1938. William Hewlett y David Packard crean su empresa en Palo Alto, California, Estados Unidos.
    
    \item 1939. Turing comienza a descifrar los c\'odigos secretos alemanes.
    
    \item 1944. John Von Neumann redacta el primer informe sobre EDVAC. En distintas universidades de Estados Unidos se desarrollan proyectos sobre computadoras cuya aplicaci\'on (secreta) ser\'a apoyar a la milicia en c\'alculos bal\'isticos (ecuaciones diferenciales).
    
    \item 1950. Turing crea su famosa prueba sobre la inteligencia artificial; se suicidar\'a en 1954. J.H. Wilkinson acudi\'o al Laboratorio Nacional de F\'isica de Reino Unido para construir una versi\'on m\'as simple de la m\'aquina de Turing; construy\'o la \textit{ACE (Automatic Computing Engine)} para resolver c\'alculos con matrices.
    
    \item 1953. John W. Backus, empleado de IBM, desarrolla \textit{FORTRAN (Formulae Translating)}, como una alternativa al uso del lenguaje ensamblador; se us\'o por primera vez en una IBM 704.
    
    \item 1958. Se anuncia la creaci\'on de la Agencia de Proyectos de Investigaci\'on Avanzada (ARPA).
    
    \item 1962. Doug Engelbart publica \textit{Aumentar el intelecto humano}; en 1963, junto con Bill English inventar\'a el rat\'on.
    
    \item 1968. Noyce y Moore fundan \textit{INTEL}.
    
    \item 1969. Misi\'on Apolo 11. Katherine Johnson calcula la trayectoria del cohete Mercurio. Dorothy Vaughan se convierte en la supervisora de IBM dentro de la NASA. Mary Jackson es la primera ingeniera aeroespacial en Estados Unidos. Margaret Hamilton escribe el c\'odigo del programa que control\'o la nave. Todas ellas tuvieron una participaci\'on fundamental para que la misi\'on fuera un \'exito.
    
    \item 1970. Investigadores visitantes en el \textit{Argone National Laboratory} de Estados Unidos traducen c\'odigos de \textit{ALGOL} para obtener eigenvalores planteados por Wilkinson para incluirlos en \textit{FORTRAN}. De esta labor nace \textit{EISPACK} en 1976 y posteriormente \textit{LINPACK} en 1976.
    
    \item 1973. Vint Cerf y Bob Kahn completan los protocolos TCP/IP.
    
    \item 1975. Bill Gates y Paul Allen desarrollan el lenguaje de programaci\'on \textit{BASIC}; fundan \textit{Microsoft}. Steve Jobs y Steve Wozniak lanzan el \textit{Apple I}.
    
    \item 1983. Richard Stallman empieza a desarrollar el proyecto \textit{GNU}.
    
    \item 1986. Cleve Moler, a partir de \textit{EISPACK} y \textit{LINPACK}, crea \textit{MATLAB}; funda la empresa \textit{MathWorks}.
    
    \item 1991. Linus Torvalds lanza la primera versi\'on de \textit{Linux}. Tim Berbers-Lee anuncia la \textit{World Wide Web}.
\end{itemize}

\section{Necesidad de la aplicaci\'on de los m\'etodos num\'ericos en la ingenier\'ia}

\textit{An\'alisis Num\'erico} es una rama de las matem\'aticas que, mediante el uso de algoritmos iterativos, obtiene soluciones num\'ericas a problemas en los cuales la matem\'atica simb\'olica (o anal\'itica) resulta poco eficiente o no puede ofrecer un resultado. En particular, a estos algoritmos se les denomina \textit{m\'etodos num\'ericos}.

Por lo general los m\'etodos num\'ericos se componen de un n\'umero de pasos finitos que se ejecutan de manera l\'ogica, mejorando aproximaciones iniciales a cierta cantidad, tal como la ra\'iz de una ecuaci\'on, hasta que se cumple con cierta cota de error. A esta operaci\'on c\'iclica de mejora del valor se le conoce como \textit{iteraci\'on}.

\textbf{Ejemplo.} Uno de los ejercicios m\'as comunes en los cursos b\'asicos de \'algebra universitaria consiste en encontrar las ra\'ices de un polinomio. El estudiante conoce principios tales como que posee $n$ ra\'ices, donde $n$ es el grado del polinomio. Conoce tambi\'en que es posible que existan exclusivamente ra\'ices reales o bien, una combinaci\'on entre ra\'ices reales y ra\'ices complejas, existiendo estas \'ultimas en parejas conjugadas. El m\'etodo de soluci\'on com\'unmente utilizado es la divisi\'on sint\'etica (que es un m\'etodo num\'erico). El estudiante lo aplica tantas veces como sea necesario para lograr que el residuo de la divisi\'on sea cero, o muy cercano a cero.

No obstante, este procedimiento podr\'ia dejar insatisfecho a un estudiante acucioso pues a\'un cuando existen mecanismos para elegir un valor inicial de una ra\'iz, se invierte mucho tiempo mejorando este valor inicial; adicionalmente es complicado obtener las ra\'ices complejas, cosa que usualmente debe lograrse a trav\'es de un cambio de variable y del uso de la f\'ormula general para ecuaciones de segundo grado. Finalmente, este proceso s\'olo es aplicable en polinomios; no es posible su aplicaci\'on en ecuaciones trascendentes.

El an\'alisis num\'erico es una alternativa muy eficiente para la resoluci\'on de ecuaciones, tanto algebraicas (polinomios) como trascendentes teniendo una ventaja muy importante respecto a otro tipo de m\'etodos: La repetici\'on de instrucciones l\'ogicas (iteraciones), proceso que permite mejorar los valores inicialmente considerados como soluci\'on. Dado que se trata siempre de la misma operaci\'on l\'ogica, resulta muy pertinente el uso de recursos de c\'omputo para realizar esta tarea.

Sin embargo, debe haber claridad en el sentido de que el an\'alisis num\'erico no es la panacea en la soluci\'on de problemas matem\'aticos; los m\'etodos num\'ericos arrojan \textit{aproximaciones}, es decir, est\'an sujetos a un error. Esto quiere decir que si se puede ser tan preciso como los recursos de c\'alculo lo permitan, siempre est\'a presente y debe considerarse su manejo en el desarrollo de las soluciones requeridas.

Es muy posible que el uso de diversos sistemas de c\'omputo determine qu\'e soluciones anal\'itico–num\'ericas son viables en la pr\'actica, lo que implica que se deben tomar en cuenta el proceso iterativo, el costo de los recursos f\'isicos que se emplean en el an\'alisis, y el tipo de pr\'actica de la Ingenier\'ia. 

\subsection{Clasificaci\'on de los errores}

Las diferencias (errores) son m\'ultiples y de diversa naturaleza, aunque pueden separarse en dos grupos gen\'ericos:

\begin{itemize}
    \item \textbf{Los errores que provienen del modelado te\'orico} (o abstracci\'on matem\'atica) del fen\'omeno real; estos errores se denominan \textit{Errores del modelo o inherentes}. Los errores inherentes son producto de factores intr\'insecos a la naturaleza, al ambiente y las personas mismas. Los errores inherentes son imposibles de remediar aunque pueden minimizarse; en consecuencia, no pueden cuantificarse.

    \begin{quote}
    Se distinguen dos tipos de errores inherentes: \textbf{Las incertidumbres} hacen referencia a las dimensiones f\'isicas que nunca podr\'an ser medidas en forma exacta debido a la naturaleza de la materia y a las imperfecciones de los instrumentos de medici\'on. \textbf{Las verdaderas equivocaciones} son las situaciones que se producen en la lectura de instrumentos de medici\'on o en el traslado de informaci\'on y que son inadvertidas a las personas; un claro ejemplo de estas situaciones es la denominada \textit{ceguera de taller}.
    \end{quote}

    \item \textbf{Los errores del m\'etodo} son producto de la limitante en la representaci\'on y manipulaci\'on de cantidades num\'ericas utilizadas en los c\'alculos necesarios en el desarrollo del modelo matem\'atico. Es de destacar que los dispositivos de c\'alculo (tales como calculadoras y computadoras) utilizan y manipulan cantidades en forma imprecisa.

    Existen dos grandes tipos de errores del m\'etodo: \textit{El truncamiento} se provoca ante la imposibilidad de manipular, por parte de un instrumento de c\'omputo, una cantidad infinita de t\'erminos o cifras. Los t\'erminos o cifras omitidas (que son infinitas en n\'umero) introducen un error en los resultados calculados. \textit{El redondeo} se produce por el mismo motivo que el truncamiento pero, a diferencia de \'este, las cifras omitidas s\'i son consideradas en la cifra resultante. Esta consideraci\'on se hace aplicando el siguiente esquema al d\'igito menos significativo (dms) de la cifra a redondear de acuerdo al siguiente esquema:

    \begin{enumerate}
        \item Si el dms es mayor a 5, se incrementa en una unidad la cifra anterior.
        \item Si el dms es menor a 5, la cifra anterior no se modifica.
        \item Si el dms es igual a 5, deber\'a observarse a la cifra anterior; si \'esta es par no sufre modificaci\'on, pero por el contrario, si es impar, deber\'a incrementarse en una unidad.
    \end{enumerate}

    Quiz\'as se conozca una versi\'on pr\'actica y popular del redondeo sim\'etrico en el cual la consideraci\'on tres se incluye en la primera de este esquema. Finalmente, existen tambi\'en esquemas que permiten minimizar la ocurrencia de estos errores, el igual forma es importante destacar que los errores del m\'etodo s\'i pueden ser cuantificados.
\end{itemize}

\section{Clasificaci\'on de los errores}

Las diferencias (errores) son m\'ultiples y de diversa naturaleza, aunque pueden separarse en dos grupos gen\'ericos:

\begin{itemize}
  \item Los errores que provienen del modelado te\'orico (o abstracci\'on matem\'atica) del fen\'omeno real; estos errores se denominan \textit{Errores del modelo o inherentes}. Los errores inherentes son producto de factores intr\'insecos a la naturaleza, al ambiente y las personas mismas. Los errores inherentes son imposibles de remediar aunque pueden minimizarse; en consecuencia, no pueden cuantificarse.

    Se distinguen dos tipos de errores inherentes: \textit{Las incertidumbres} hacen referencia a las dimensiones f\'isicas que nunca podr\'an ser medidas en forma exacta debido a la naturaleza de la materia y a las imperfecciones de los instrumentos de medici\'on. \textit{Las verdaderas equivocaciones} son las situaciones que se producen en la lectura de instrumentos de medici\'on o en el traslado de informaci\'on y que son inadvertidas a las personas; un claro ejemplo de estas situaciones es la denominada \textit{ceguera de taller}.

  \item Los \textit{errores del m\'etodo} son producto de la limitante en la representaci\'on y manipulaci\'on de cantidades num\'ericas utilizadas en los c\'alculos necesarios en el desarrollo del modelo matem\'atico. Es de destacar que los dispositivos de c\'alculo (tales como calculadoras y computadoras) utilizan y manipulan cantidades en forma imprecisa.

    Existen dos grandes tipos de errores del m\'etodo: \textit{El truncamiento} se provoca ante la imposibilidad de manipular, por parte de un instrumento de c\'omputo, una cantidad infinita de t\'erminos o cifras. Los t\'erminos o cifras omitidas (que son infinitas en n\'umero) introducen un error en los resultados calculados. \textit{El redondeo} se produce por el mismo motivo que el truncamiento pero, a diferencia de \'este, las cifras omitidas s\'i son consideradas en la cifra resultante. Esta consideraci\'on se hace aplicando el siguiente esquema al d\'igito menos significativo (dms) de la cifra a redondear de acuerdo al siguiente esquema:

    \begin{enumerate}
        \item Si el dms es mayor a 5, se incrementa en una unidad la cifra anterior.
        \item Si el dms es menor a 5, la cifra anterior no se modifica.
        \item Si el dms es igual a 5, deber\'a observarse a la cifra anterior; si \'esta es par no sufre modificaci\'on, pero por el contrario, si es impar, deber\'a incrementarse en una unidad.
    \end{enumerate}

    Quiz\'as se conozca una versi\'on pr\'actica y popular del redondeo sim\'etrico en el cual la consideraci\'on tres se incluye en la primera de este esquema. Finalmente, existen tambi\'en esquemas que permiten minimizar la ocurrencia de estos errores, el cual forma una importante destacar que los errores del m\'etodo s\'i pueden ser cuantificados.
\end{itemize}

\section{Cuantificaci\'on de errores}

Los errores se cuantifican de dos formas diferentes:

\begin{enumerate}
  \item \textbf{Error absoluto}. El error absoluto es la diferencia absoluta entre un valor real y un aproximado. Est\'a dado por la siguiente f\'ormula:

  $$
  E = \left| V_{real} - V_{aprox} \right|
  $$

  El error absoluto recibe este nombre ya que posee las mismas dimensiones que la variable bajo estudio.

  \item \textbf{Error relativo}. Corresponde a la expresi\'on en porcentaje de un error absoluto; en consecuencia, este error es adimensional.

  $$
  e = \left| \frac{V_{real} - V_{aprox}}{V_{real}} \right| \cdot 100\%
  $$
\end{enumerate}

La diferencia entre la preferencia en el uso de los dos tipos de error consiste precisamente en la presencia de las dimensiones f\'isicas. Debido a las unidades de medici\'on utilizadas, el manejo y la percepci\'on del error absoluto suele ser enga\~noso o dif\'icil de comprender r\'apidamente. Sin embargo, el manejo de porcentajes (o valores relativos) resulta m\'as natural y sencillo de comprender. Sin embargo, el uso de estos dos tipos de errores est\'a sujeto siempre al objetivo de las actividades desarrolladas.

\subsection*{Consideraciones sobre el Valor Real ($V_{real}$)}

Las expresiones que definen a los errores absoluto y relativo requieren del conocimiento de la variable $V_{real}$ que representa un valor ideal que no posee error alguno. Como podr\'a suponerse, en la realidad resulta imposible determinar este valor.

Una pr\'actica com\'un en los an\'alisis elementales sobre errores es considerar como un valor real a los resultados arrojados por la medici\'on experimental de los fen\'omenos y a los valores aproximados como los proporcionados por los modelos matem\'aticos (o viceversa). El lector ha percibido que en ambos valores existe un error, por lo cual ninguno de ellos puede ser considerado como valor real. En realidad, ambos valores son valores aproximados. Por ejemplo, los valores nominales son considerados como reales.

Para lograr un resultado coherente, en la pr\'actica debe sustituirse al valor real por un valor que se considere posee un error menor. Por ejemplo, en un proceso de mediciones suele utilizarse como valor real a los valores nominales citados en las especificaciones de los objetos a medir.

En el caso del an\'alisis num\'erico, dado que los resultados se obtienen a partir de procesos iterativos que se mejoran los inicialmente obtenidos, debe partirse del supuesto que el \'ultimo valor obtenido posee un nivel menor de error que el valor previo. Dado lo anterior, los errores absoluto y relativo se calcular\'an de la siguiente forma:

\textbf{Error absoluto:}
$$
E = \left| V_i - V_{i-1} \right|
$$

\textbf{Error relativo:}
$$
e = \left| \frac{V_i - V_{i-1}}{V_i} \right| \cdot 100\%
$$

En ambas ecuaciones, $V_i$ es el valor de la \'ultima iteraci\'on y $V_{i-1}$ es el valor de la iteraci\'on anterior i – 1.

Al intentar determinar experimentalmente el tiempo que demora la ca\'ida se utilizar\'a un cron\'ometro. Este instrumento introduce un error en la medici\'on. Por otro lado, el valor de la gravedad en cada lugar de la Tierra var\'ia dependiendo de la latitud del mismo. El uso de un valor promedio de la gravedad tambi\'en implica un error. La diferencia entre el valor que arroje el cron\'ometro y el valor que resulta del c\'alculo con la f\'ormula matem\'atica es lo que se conoce como error.

\section{Clasificaci\'on de los errores}

Las diferencias (errores) son m\'ultiples y de diversa naturaleza, aunque pueden separarse en dos grupos gen\'ericos:

\begin{itemize}
  \item Los errores que provienen del modelado te\'orico (o abstracci\'on matem\'atica) del fen\'omeno real; estos errores se denominan \textit{Errores del modelo o inherentes}. Los errores inherentes son producto de factores intr\'insecos a la naturaleza, al ambiente y las personas mismas. Los errores inherentes son imposibles de remediar aunque pueden minimizarse; en consecuencia, no pueden cuantificarse.

  Se distinguen dos tipos de errores inherentes:
  \begin{itemize}
    \item \textit{Las incertidumbres}, que hacen referencia a las dimensiones f\'isicas que nunca podr\'an ser medidas en forma exacta debido a la naturaleza de la materia y a las imperfecciones de los instrumentos de medici\'on.
    \item \textit{Las verdaderas equivocaciones}, que son las situaciones que se producen en la lectura de instrumentos de medici\'on o en el traslado de informaci\'on y que son inadvertidas a las personas; un claro ejemplo de estas situaciones es la denominada \textit{ceguera de taller}.
  \end{itemize}

  \item Los \textit{errores del m\'etodo} son producto de la limitante en la representaci\'on y manipulaci\'on de cantidades num\'ericas utilizadas en los c\'alculos necesarios en el desarrollo del modelo matem\'atico. Es de destacar que los dispositivos de c\'alculo (tales como calculadoras y computadoras) utilizan y manipulan cantidades en forma imprecisa.
\end{itemize}


\section{Aproximaci\'on num\'erica y errores}

Una \textit{aproximaci\'on} es un valor cercano a uno considerado como real o verdadero. Esta cercan\'ia, o diferencia, se conoce como \textit{error}.

Normalmente, la consideraci\'on de la validez de una aproximaci\'on depende de la cota de error que el experimentador considere pertinente en funci\'on del contexto del fen\'omeno bajo estudio. Esto implica que tambi\'en debe considerarse que una magnitud debe ser un valor real, que en el \'ambito de la Ingenier\'ia pocas veces se conoce, lo que obliga a adoptar convenciones.

\subsection*{Exactitud y precisi\'on}

En Ingenier\'ia, se denomina \textit{exactitud} a la capacidad de un instrumento de medir un valor cercano al de la magnitud real. Exactitud implica precisi\'on, pero no al contrario. Exactitud y precisi\'on no son equivalentes. Exactitud es capacidad para acercarse a la magnitud real, y precisi\'on es la capacidad de generar resultados similares. La precisi\'on se logra cuando un instrumento para repetir mediciones exactas cuando \'estas se realizan consecutivamente.

De acuerdo con la definici\'on de aproximaci\'on num\'erica, la exactitud se aplica en los m\'etodos num\'ericos en cuanto a la capacidad del m\'etodo de generar un resultado muy cercano al valor real; se percibe la cercan\'ia entre la exactitud y el concepto de error. Por otra parte, los m\'etodos num\'ericos a trav\'es de iteraciones generan valores aproximados cada vez m\'as exactos, es decir, estas iteraciones deber\'an ser precisas. Dado lo anterior, los m\'etodos num\'ericos deber\'an tener como cualidades la exactitud y la precisi\'on.

\section{Definici\'on de errores}

Una actividad frecuente del profesional de la Ingenier\'ia consiste en trabajar con modelos matem\'aticos representativos de un fen\'omeno f\'isico. Estos modelos son abstracciones matem\'aticas que distan mucho de representar exactamente el fen\'omeno bajo estudio debido principalmente a las carencias y dificultades que a\'un posee el humano de la comprensi\'on total de la naturaleza.

Como consecuencia de esto existen diferencias entre los resultados obtenidos experimentalmente y los emanados propiamente del modelo matem\'atico.

A las diferencias cuantitativas entre los dos modelos se les denomina \textit{Errores}.

\textbf{Ejemplo.} Sea \( h $ la altura a la que se encuentra un cuerpo, \( g $ la constante de la aceleraci\'on de la gravedad y \( t $ el tiempo que dura la ca\'ida, se define el modelo matem\'atico como:

$$
t = \sqrt{\frac{2h}{g}}
$$

Al intentar determinar experimentalmente el tiempo que demora la ca\'ida se utilizar\'a un cron\'ometro. Este instrumento introduce un error en la medici\'on. Por otro lado, el valor de la gravedad en cada lugar de la Tierra var\'ia dependiendo de la latitud del mismo. El uso de un valor promedio de la gravedad tambi\'en implica un error. La diferencia entre el valor que arroje el cron\'ometro y el valor que resulta del c\'alculo con la f\'ormula matem\'atica es lo que se conoce como error.


\section{Aproximaci\'on num\'erica y errores}

Una \textit{aproximaci\'on} es un valor cercano a uno considerado como real o verdadero. Esta cercan\'ia, o diferencia, se conoce como \textit{error}.

Normalmente, la consideraci\'on de la validez de una aproximaci\'on depende de la cota de error que el experimentador considere pertinente en funci\'on del contexto del fen\'omeno bajo estudio. Esto implica que tambi\'en debe considerarse que una magnitud debe ser un valor real, que en el \'ambito de la Ingenier\'ia pocas veces se conoce, lo que obliga a adoptar convenciones.

\subsection*{Exactitud y precisi\'on}

En Ingenier\'ia, se denomina \textit{exactitud} a la capacidad de un instrumento de medir un valor cercano al de la magnitud real. Exactitud implica precisi\'on, pero no al contrario. Exactitud y precisi\'on no son equivalentes. Exactitud es capacidad para acercarse a la magnitud real, y precisi\'on es la capacidad de generar resultados similares. La precisi\'on se logra cuando un instrumento para repetir mediciones exactas cuando \'estas se realizan consecutivamente.

De acuerdo con la definici\'on de aproximaci\'on num\'erica, la exactitud se aplica en los m\'etodos num\'ericos en cuanto a la capacidad del m\'etodo de generar un resultado muy cercano al valor real; se percibe la cercan\'ia entre la exactitud y el concepto de error. Por otra parte, los m\'etodos num\'ericos a trav\'es de iteraciones generan valores aproximados cada vez m\'as exactos, es decir, estas iteraciones deber\'an ser precisas. Dado lo anterior, los m\'etodos num\'ericos deber\'an tener como cualidades la exactitud y la precisi\'on.

\section{Definici\'on de errores}

Una actividad frecuente del profesional de la Ingenier\'ia consiste en trabajar con modelos matem\'aticos representativos de un fen\'omeno f\'isico. Estos modelos son abstracciones matem\'aticas que distan mucho de representar exactamente el fen\'omeno bajo estudio debido principalmente a las carencias y dificultades que a\'un posee el humano de la comprensi\'on total de la naturaleza.

Como consecuencia de esto existen diferencias entre los resultados obtenidos experimentalmente y los emanados propiamente del modelo matem\'atico.

A las diferencias cuantitativas entre los dos modelos se les denomina \textit{Errores}.

\textbf{Ejemplo.} Sea \( h $ la altura a la que se encuentra un cuerpo, \( g $ la constante de la aceleraci\'on de la gravedad y \( t $ el tiempo que dura la ca\'ida, se define el modelo matem\'atico como:

$$
t = \sqrt{\frac{2h}{g}}
$$


En un caso ideal, si se suponen condiciones ideales de ca\'ida, es decir, sin fricci\'on del aire y con una aceleraci\'on constante de la gravedad, un cuerpo que cae desde una altura de \( h = 100 $ m, usando \( g = 9.8 $ m/s\(^2$, se obtendr\'ia que el tiempo de ca\'ida es:

$$
t = \sqrt{\frac{2 \cdot 100}{9.8}} = \sqrt{20.41} = 4.51 \text{ s}
$$

En un caso real, el experimento reporta que el tiempo de ca\'ida es de 4.85 s, debido a la fricci\'on del aire y variaciones en la gravedad. El error cometido por el modelo es:

$$
\text{Error} = 4.85 - 4.51 = 0.34 \text{ s}
$$

\subsection*{Tipos de errores}

\begin{itemize}
    \item \textbf{Error absoluto:} Se define como la diferencia entre el valor real (experimental o exacto) y el valor aproximado (te\'orico o calculado):

    $$
    E_a = |x_{\text{real}} - x_{\text{aproximado}}|
    $$

    \item \textbf{Error relativo:} Es la relaci\'on entre el error absoluto y el valor real:

    $$
    E_r = \frac{E_a}{|x_{\text{real}}|}
    $$

    \item \textbf{Error porcentual:} Es el error relativo expresado en porcentaje:

    $$
    E_p = E_r \cdot 100 = \frac{E_a}{|x_{\text{real}}|} \cdot 100
    $$
\end{itemize}


\begin{itemize}
  \item Sea \(Y$ la variable aleatoria respuesta binaria:
  $$
  Y = 
  \begin{cases}
    1 & \text{si sobrevive al tratamiento} \\
    0 & \text{si no sobrevive}
  \end{cases}
  $$

  \item El modelo lineal general:
  $$
  \eta = \beta_0 + \beta_1 x_1 + \cdots + \beta_k x_k
  $$
  no es adecuado para variables respuesta binarias, ya que no garantiza
  que las probabilidades resultantes est\'en en el intervalo \([0,1]$.

  \item El modelo de regresi\'on log\'istica usa una funci\'on de enlace para
  asegurar que \(P(Y = 1)$ est\'e en \([0,1]$.

  \item La funci\'on de enlace logit es:
  $$
  \text{logit}(p) = \log\left(\frac{p}{1 - p}\right)
  $$
  y su inversa es:
  $$
  p = \frac{1}{1 + e^{-\eta}}
  $$

  \item Entonces, el modelo de regresi\'on log\'istica se escribe como:
  $$
  \log\left(\frac{P(Y = 1 \mid \mathbf{x})}{1 - P(Y = 1 \mid \mathbf{x})}\right) = 
  \beta_0 + \beta_1 x_1 + \cdots + \beta_k x_k
  $$
\end{itemize}


\noindent
Un \textit{m\'etodo num\'erico} es un proceso matem\'atico \textit{iterativo} cuyo objetivo es encontrar la aproximaci\'on a una soluci\'on espec\'ifica con un cierto error previamente determinado.

\medskip

A diferencia de las t\'ecnicas propias de la matem\'atica anal\'itica, los m\'etodos num\'ericos requieren de una aproximaci\'on a la soluci\'on real al problema, misma que es corregida a trav\'es de la repetici\'on de un cierto proceso que debe arrojar soluciones cada vez m\'as cercanas al valor real. Cada correcci\'on de un valor inicial se conoce como \textit{iteraci\'on}. El proceso es controlado por medio de la medici\'on de una cantidad de error predefinido entre dos aproximaciones sucesivas.

\medskip

No existe unanimidad entre los expertos sobre si \textit{An\'alisis num\'erico} es un sin\'onimo de \textit{m\'etodos num\'ericos}. Algunos consideran que los m\'etodos num\'ericos son procesos con objetivos particulares que conforman un proceso m\'as complejo, espec\'ificamente de interpretaci\'on de los resultados al que denominan \textit{An\'alisis num\'erico}.

\medskip

Resulta complicado tomar partido por alguna de las dos posturas anteriores en la consideraci\'on de que la aplicaci\'on de los procesos iterativos suele hacerse a problemas reales, con condiciones de dise\~no muy espec\'ificas, por lo que no se puede establecer una regla general para hacer un an\'alisis. En el caso que nos ocupa, dado que se har\'a una presentaci\'on te\'orica para definir cada proceso, se optar\'a por llamarlo \textit{m\'etodo num\'erico}.

\section{Introducci\'on hist\'orica de los m\'etodos num\'ericos}

La historia de los m\'etodos num\'ericos es la colecci\'on de acontecimientos matem\'aticos en los que se resuelven problemas sin el uso de la matem\'atica anal\'itica (Finkelshtein, s.f.).

Algunos de los m\'etodos m\'as utilizados en la actualidad fueron creados mucho antes de la invenci\'on de la computadora; su aplicaci\'on era extenuante y complicada porque cada iteraci\'on requer\'ia de una diversidad de operaciones aritm\'eticas que se realizaban por grupos enteros de calculistas, evidentemente, de forma manual.

Todos los enterados en la materia estar\'an de acuerdo en que una computadora realiza una gran cantidad de operaciones en un intervalo muy peque\~no; las s\'uper computadoras lo hace pero en forma paralela. Esta capacidad es la que ha dado un sentido de aplicaci\'on a los m\'etodos num\'ericos.

Por lo anterior, la historia de los m\'etodos num\'ericos es paralela, al menos desde la mitad del siglo XIX, a la historia de la computaci\'on. Las contribuciones m\'as actuales radican en la creaci\'on de software que minimiza los errores y mejora las aproximaciones de los resultados.

Esta es una relaci\'on de hechos que han marcado la historia de los m\'etodos num\'ericos y se recomienda al lector, de acuerdo a su inter\'es, profundizar en el t\'opico que le resulte de su inter\'es; en particular, la obra \textit{Los Innovadores} (Isaacson, 2014) de Walter Isaacson ofrece un panorama muy amplio al respecto.

\begin{itemize}
    \item 1650 a.C. Se crean los Papiros de Rhyad en los que se escribe un m\'etodo para resolver expresiones matem\'aticas sin \'algebra.
    \item 250 a.C. Euclides crea el M\'etodo de Exhausti\'on, que consiste en aproximar figuras geom\'etricas (tri\'angulos, cuadrados, pent\'agonos, etc.) consecutivamente dentro de un c\'irculo para obtener una aproximaci\'on a $\pi$.
    \item Siglo IX d.C. Al Juarismi crea los \textit{algoritmos}.
    \item 1623. John Napier inventa los \textit{huesos de Napier}, que son arreglos pr\'acticos de logaritmos en tablas.
    \item Siglo XVII. Isaac Newton crea los procesos de interpolaci\'on polinomial.
    \item Siglo XVIII Leibnitz crea el C\'alculo diferencial.
    \item 1768. Euler crea soluciones aproximadas a ecuaciones diferenciales con el principio de la integraci\'on num\'erica. Jacob Stirling y Brook Taylor presentan el C\'alculo de diferencias finitas.
    \item 1822. Charles Babbage inventa la \textit{M\'aquina diferencial}.
    \item 1843. Ada, condesa de Lovelace, publica sus notas sobre la m\'aquina anal\'itica de Charles Babbage.
    \item 1890. (IBM) Tabula el censo estadounidense empleando las m\'aquinas de tarjetas perforadas de Herman Hollerith.
    \item 1931. Vannebar Bush dise\~na el analizador diferencial, un computador anal\'ogico electromec\'anico. En 1945 publicar\'a el art\'iculo \textit{C\'omo podremos pensar} en el que describe la computaci\'on personal.

\item 1937. Alan Turing publica \textit{Sobre los n\'umeros computables}, en el que describe un computador universal. En este mismo a\~no, Howard Aiken propone la construcci\'on de un gran computador y descubre partes de la m\'aquina diferencial de Babbage en Harvard; tambi\'en John Vincent Atanasoff conceptualiza el computador electr\'onico la cual completar\'a en 1939.
    
\item 1938. William Hewlett y David Packard crean su imprenta en Palo Alto, California, Estados Unidos.
    
\item 1939. Turing comienza a descifrar los c\'odigos secretos alemanes.
    
\item 1944. John Von Neumann redacta el primer informe sobre EDVAC. En distintas universidades de Estados Unidos se desarrollan proyectos sobre computadoras cuya aplicaci\'on (secreta) ser\'a apoyar a la milicia en c\'alculos bal\'isticos (ecuaciones diferenciales).
    
\item 1950. Turing crea su famosa prueba sobre la inteligencia artificial; se suicidar\'a en 1954. J.H. Wilkinson acudi\'o al Laboratorio Nacional de F\'isica de Reino Unido para construir una versi\'on m\'as simple de la m\'aquina de Turing; construy\'o la ACE (\textit{Automatic Computing Engine}) para resolver c\'alculos con matrices.
    
\item 1953. John W. Backus, empleado de IBM, desarrolla \textsc{FORTRAN} (\textit{Formulae Translating}) como una alternativa al uso del lenguaje ensamblador; se us\'o por primera vez en una IBM 704.
    
\item 1958. Se anuncia la creaci\'on de la Agencia de Proyectos de Investigaci\'on Avanzada (ARPA).
    
\item 1962. Doug Engelbart publica \textit{Aumentar el intelecto humano}; en 1963, junto con Bill English inventar\'a el rat\'on.
    
\item 1968. Noyce y Moore fundan \textsc{INTEL}.
    
\item 1969. Misi\'on Apolo 11. Katherine Johnson calcula la trayectoria del cohete Mercurio. Dorothy Vaughan se convierte en la supervisora de IBM dentro de la NASA. Mary Jackson es la primer ingeniera aeroespacial en Estados Unidos. Margaret Hamilton escribe el c\'odigo del programa que control\'o la nave. Todas ellas tuvieron una participaci\'on fundamental para que la misi\'on fuera un \'exito.
    
\item 1970. Investigadores visitantes en el \textit{Argone National Laboratory} de Estados Unidos traducen c\'odigos de \textsc{ALGOL} para obtener eigenvalores planteados por Wilkinson para incluirlos en \textsc{FORTRAN}. De esta labor nace \textsc{EISPACK} en 1976 y posteriormente \textsc{LINPACK} en 1979.
    
\item 1973. Vint Cerf y Bob Kahn completan los protocolos \textsc{TCP/IP}.
    
\item 1975. Bill Gates y Paul Allen desarrollan el lenguaje de programaci\'on \textsc{BASIC}; fundan \textit{Microsoft}. Steve Jobs y Steve Wozniak lanzan el \textit{Apple I}.
    
\item 1983. Richard Stallman empieza a desarrollar el proyecto \textsc{GNU}.
    
\item 1985. Cleve Moler, a partir de \textsc{EISPACK} y \textsc{LINPACK} crea \textsc{MATLAB}; funda la empresa \textsc{MathWorks}.
    
\item 1991. Linus Torvalds lanza la primera versi\'on de \textsc{Linux}. Tim Berbers-Lee anuncia la \textit{World Wide Web}.

\item 1997. \textit{Deep Blue}, de IBM, derrota a Gari Kasparov en una partida de ajedrez.

\item 1998. Larry Page y Sergu\'ei Brin liberan \textit{Google}.
\end{itemize}

\section{Necesidad de la aplicaci\'on de los m\'etodos num\'ericos en la ingenier\'ia}

El \textit{An\'alisis Num\'erico} es una rama de las matem\'aticas que, mediante el uso de algoritmos iterativos, obtiene soluciones num\'ericas a problemas en los cuales la matem\'atica simb\'olica (o anal\'itica) resulta poco eficiente o no puede ofrecer un resultado. En particular, a estos algoritmos se les denomina \textit{m\'etodos num\'ericos}.

Por lo general los m\'etodos num\'ericos se componen de un n\'umero de pasos finitos que se ejecutan de manera l\'ogica, mejorando aproximaciones iniciales a cierta cantidad, tal como la ra\'iz de una ecuaci\'on, hasta que se cumple con cierta cota de error. A esta operaci\'on c\'iclica de mejora del valor se le conoce como \textit{iteraci\'on}.

\textbf{Ejemplo.} Uno de los ejercicios m\'as comunes en los cursos b\'asicos de \'algebra universitaria consiste en encontrar las ra\'ices de un polinomio. El estudiante conoce principios tales como que posee $n$ ra\'ices, donde $n$ es el grado del polinomio. Conoce tambi\'en que es posible que existan exclusivamente ra\'ices reales o bien, una combinaci\'on entre ra\'ices reales y ra\'ices complejas, existiendo estas \'ultimas en parejas conjugadas. El m\'etodo de soluci\'on com\'unmente utilizado es la divisi\'on sint\'etica (que es un m\'etodo num\'erico). El estudiante lo aplica tantas veces como sea necesario para lograr que el residuo de la divisi\'on sea cero, o muy cercano a cero.

No obstante, este procedimiento podr\'ia dejar insatisfecho a un estudiante acucioso pues a\'un cuando existen mecanismos para elegir un valor inicial de una ra\'iz, se invierte mucho tiempo mejorando este valor inicial; adicionalmente es complicado obtener las ra\'ices complejas, cosa que usualmente debe lograrse a trav\'es de un cambio de variable y del uso de la f\'ormula general para ecuaciones de segundo grado. Finalmente, este proceso s\'olo es aplicable en polinomios; no es posible su aplicaci\'on en ecuaciones trascendentes.

El an\'alisis num\'erico es una alternativa muy eficiente para la resoluci\'on de ecuaciones, tanto algebraicas (polinomios) como trascendentes teniendo una ventaja muy importante respecto a otro tipo de m\'etodos: La repetici\'on de instrucciones l\'ogicas (iteraciones), proceso que permite mejorar los valores inicialmente considerados como soluci\'on. Dado que se trata siempre de la misma operaci\'on l\'ogica, resulta muy pertinente el uso de recursos de c\'omputo para realizar esta tarea.

Sin embargo, debe haber claridad en el sentido de que el an\'alisis num\'erico no es la panacea en la soluci\'on de problemas matem\'aticos; los m\'etodos num\'ericos arrojan \textit{aproximaciones}, es decir, est\'an sujetos a un error. Esto, que en el pasado se puede ser tan peque\~no que con los recursos de c\'alculo de la pantalla, siempre est\'a presente y debe considerarse su manejo en el desarrollo de las soluciones requeridas.

Es muy posible que se d\'e a conocer diversos sistemas de c\'omputo que proporcionen soluciones anal\'iticas. Estos m\'etodos son \'utiles y pueden ser empleados siempre y cuando se compruebe que el proceso iterativo ha sido suficientemente eficaz que el error es m\'inimo y que su uso se justifica en la pr\'actica de la Ingenier\'ia.


ecuaci\'on que tenga ra\'ices exactas o enteras y, en la generalidad de los casos, los modelos nunca ser\'an lineales. Las abstracciones que se hacen en el sal\'on de clase son s\'olo ensayos ideales. Ante esta situaci\'on, el ingeniero seguramente acudir\'a a los m\'etodos num\'ericos para resolver las situaciones que se le presentan en la vida profesional.

\section{Aproximaci\'on num\'erica y errores}

Una \textit{aproximaci\'on} es un valor cercano a uno considerado como real o verdadero. Esta cercan\'ia, o diferencia, se conoce como \textit{error}.

Normalmente, la consideraci\'on de la validez de una aproximaci\'on depende de la cota de error que el experimentador considere pertinente en funci\'on del contexto del fen\'omeno bajo estudio. Esto implica que tambi\'en debe considerarse que una magnitud debe ser un valor real, que en el \'ambito de la Ingenier\'ia pocas veces se conoce, lo que obliga a adoptar convenciones.

\subsection*{Exactitud y precisi\'on}

En Ingenier\'ia, se denomina \textit{exactitud} a la capacidad de un instrumento de medir un valor cercano al de la magnitud real. Exactitud implica precisi\'on, pero no al contrario. Exactitud y precisi\'on no son equivalentes. Exactitud es capacidad para acercarse a la magnitud real, y precisi\'on es la capacidad de generar resultados similares. La precisi\'on se logra cuando un instrumento para repetir mediciones exactas cuando estas se realizan consecutivamente.

De acuerdo con la definici\'on de aproximaci\'on num\'erica, la exactitud se aplica en los m\'etodos num\'ericos en cuanto a la capacidad del m\'etodo de generar un resultado muy cercano al valor real; se percibe la cercan\'ia entre la exactitud y el concepto de error. Por otra parte, los m\'etodos num\'ericos a trav\'es de iteraciones generan valores aproximados cada vez m\'as exactos, es decir, estas iteraciones deber\'an ser precisas. Dado lo anterior, los m\'etodos num\'ericos deber\'an tener como cualidades la exactitud y la precisi\'on.

\section{Definici\'on de errores}

Una actividad frecuente del profesional de la Ingenier\'ia consiste en trabajar con modelos matem\'aticos representativos de un fen\'omeno f\'isico. Estos modelos son abstracciones matem\'aticas que distan mucho de representar exactamente el fen\'omeno bajo estudio debido principalmente a las carencias y dificultades que a\'un posee el humano de la comprensi\'on total de la naturaleza.

Como consecuencia de esto existen diferencias entre los resultados obtenidos experimentalmente y los emanados propiamente del modelo matem\'atico.

A las diferencias cuantitativas entre los dos modelos se les denomina \textit{Errores}.

\textbf{Ejemplo.} Sea $h$ la altura a la que se encuentra un cuerpo, $g$ la constante de la aceleraci\'on de la gravedad y $t$ el tiempo que dura la ca\'ida, se define el modelo matem\'atico como:
$$
t = \sqrt{\frac{2h}{g}}
$$
\subsection{Clasificaci\'on de los errores}

Las diferencias (errores) son m\'ultiples y de diversa naturaleza, aunque pueden separarse en dos grupos gen\'ericos:

\begin{itemize}
  \item Los errores que provienen del modelado te\'orico (o abstracci\'on matem\'atica) del fen\'omeno real; estos errores se denominan \textit{Errores del modelo o inherentes}. Los errores inherentes son producto de factores intr\'insecos a la naturaleza, al ambiente y las personas mismas. Los errores inherentes son imposibles de remediar aunque pueden minimizarse; en consecuencia, no pueden cuantificarse.

  Se distinguen dos tipos de errores inherentes: \textit{Las incertidumbres} hacen referencia a las dimensiones f\'isicas que nunca podr\'an ser medidas en forma exacta debido a la naturaleza de la materia y a las imperfecciones de los instrumentos de medici\'on. \textit{Las verdaderas equivocaciones} son las situaciones que se producen en la lectura de instrumentos de medici\'on o en el traslado de informaci\'on y que son inadvertidas a las personas; un claro ejemplo de estas situaciones es la denominada \textit{esquema de taller}.
  
  \item \textit{Los errores del m\'etodo} son producto de la limitante en la representaci\'on y manipulaci\'on de cantidades num\'ericas utilizadas en los c\'alculos necesarios en el desarrollo del modelo matem\'atico. Es de destacar que los dispositivos de c\'alculo (tales como calculadoras y computadoras) utilizan y manipulan cantidades en forma imprecisa.
  
  Existen dos grandes tipos de errores del m\'etodo: \textbf{El truncamiento} se provoca ante la imposibilidad de manipular, por parte de un instrumento de c\'omputo, una cantidad infinita de t\'erminos o cifras. Los t\'erminos o cifras omitidas (que son infinitas en n\'umero) introducen un error en los resultados calculados. \textbf{El redondeo} se produce por el mismo motivo que el truncamiento pero, a diferencia de \'este, las cifras omitidas s\'i son consideradas en la cifra resultante. Esta consideraci\'on se hace aplicando el siguiente esquema al d\'igito menos significativo (dms) de la cifra a redondear de acuerdo al siguiente esquema:
  \begin{enumerate}
    \item Si el dms es mayor a 5, se incrementa en una unidad la cifra anterior.
    \item Si el dms es menor a 5, la cifra anterior no se modifica.
    \item Si el dms es igual a 5, deber\'a observarse a la cifra anterior; si \'esta es par no sufre modificaci\'on, pero por el contrario, si es impar, deber\'a incrementarse en una unidad.
  \end{enumerate}
  
  Quiz\'as se conozca una versi\'on pr\'actica y popular del redondeo sim\'etrico en el cual la consideraci\'on tres se incluye en la primera de este esquema. Finalmente, existen tambi\'en esquemas que permiten minimizar la ocurrencia de estos errores, de igual forma es importante destacar que los errores del m\'etodo s\'i pueden ser cuantificados.
\end{itemize}

\subsection{Cuantificaci\'on de errores}

Los errores se cuantifican de dos formas diferentes:

\begin{enumerate}
  \item \textbf{Error absoluto}. El error absoluto es la diferencia absoluta entre un valor real y un aproximado. Est\'a dado por la siguiente f\'ormula:
  $$
  E = |V_\text{real} - V_\text{aprox}|
  $$
  El error absoluto recibe este nombre ya que posee las mismas dimensiones que la variable bajo estudio.

  \item \textbf{Error relativo}. Corresponde a la expresi\'on en porcentaje de un error absoluto; en consecuencia, este error es adimensional.
$e = \left| \frac{V_\text{real} - V_\text{aprox}}{V_\text{real}} \right| \times 100\,\%$
\end{enumerate}

La diferencia entre la preferencia en el uso de los dos tipos de error consiste precisamente en la presencia de las dimensiones f\'isicas. Debido a las unidades de medici\'on utilizadas, el manejo y la percepci\'on del error absoluto suele ser enga\~noso o dif\'icil de comprender r\'apidamente. Sin embargo, el manejo de porcentajes (o valores relativos) resulta m\'as natural y sencillo de comprender. Sin embargo, el uso de estos dos tipos de errores est\'a sujeto siempre al objetivo de las actividades desarrolladas.

\subsection*{Consideraciones sobre el Valor Real ($V_\text{real}$)}

Las expresiones que definen a los errores absoluto y relativo requieren del conocimiento de la variable $V_\text{real}$ que representa un valor ideal que no posee error alguno. Como podr\'a suponerse, en la realidad resulta imposible determinar este valor.

Una pr\'actica com\'un en los an\'alisis elementales sobre errores es considerar como un valor real a los resultados arrojados por la medici\'on experimental de los fen\'omenos o a los valores aproximados como los proporcionados por los modelos matem\'aticos (o viceversa). El lector ha percibido que en ambos valores existe un error, por lo cual ninguno de ellos puede ser considerado como valor real. En realidad, ambos valores son valores aproximados. Por ejemplo, los valores nominales son considerados como reales.

Para lograr un resultado coherente, en la pr\'actica debe sustituirse al valor real por un valor que se considere posee un error menor. Por ejemplo, en un proceso de mediciones suele utilizarse como valor real a los valores nominales citados en las especificaciones de los objetos a medir.

En el caso del an\'alisis num\'erico, dado que los resultados se obtienen a partir de procesos iterativos que se mejoran los inicialmente obtenidos, debe partirse del supuesto que el \'ultimo valor obtenido posee un nivel menor de error que el valor previo. Dado lo anterior, los errores absoluto y relativo se calcular\'an de la siguiente forma:

\begin{itemize}
  \item \textbf{Error absoluto}:
  $$
  E = |V_i - V_{i-1}|
  $$

  \item \textbf{Error relativo}:
  $$
  e = \left| \frac{V_i - V_{i-1}}{V_i} \right| \times 100\,\%
  $$
\end{itemize}

En ambos casos, $V_i$ es el valor de la \'ultima iteraci\'on y $V_{i-1}$ es el valor de la iteraci\'on anterior i--1.

\subsection*{Magnitud de los errores por truncamiento y por redondeo}

Lamentablemente, la literatura especializada sobre el tratamiento de errores es escasa, por lo que sin embargo resulta muy importante conocer la magnitud de los errores que se cometen, en este caso, en el desarrollo de m\'etodos num\'ericos. Un estudio sobre errores muy difundido entre la comunidad dedicada al desarrollo del An\'alisis num\'erico es la desarrollada por Daniel McCracken. El referido estudio est\'a enfocado al manejo de datos num\'ericos en computadora y pertenece a un momento hist\'orico en el cual los recursos de c\'omputo eran a\'un muy limitados en comparaci\'on con los disponibles en los inicios del siglo XXI. En realidad, las conclusiones de McCracken siguen vigentes hoy en d\'ia.

Una aportaci\'on importante sobre el estudio de los errores consiste en la cuantificaci\'on de la magnitud de los que se cometen en el manejo de los datos en forma inherente al uso de la aritm\'etica de punto flotante. McCracken concluye que las magnitudes de los errores cometidos por truncamiento son mayores a las cometidas por el uso del redondeo sim\'etrico \cite{McCrackenDorn1984}. Asimismo, se concluye tambi\'en que la magnitud del error por redondeo sim\'etrico es independiente de la cantidad en s\'i misma siendo producto del tama\~no de la mantisa que se utilice para hacer los c\'alculos. El m\'aximo error absoluto debido al redondeo sim\'etrico se calcula a trav\'es de la expresi\'on:

$$
\frac{1}{2} \cdot 10^{-t+1} \qquad \text{donde } t \text{ es el tama\~no de la mantisa}
$$

\noindent \textbf{Ejemplo.} Utilizando una mantisa de 3 cifras, determine el m\'aximo error absoluto cometido en las siguientes cifras:
\begin{enumerate}
  \item 10.334
  \item 123293.967
\end{enumerate}

En ambos casos, las cantidades est\'an definidas con una mantisa de tama\~no tres, $t = 3$, para lo cual sustituyendo en la ecuaci\'on correspondiente:

$$
\frac{1}{2} \cdot 10^{-t+1} = \frac{1}{2} \cdot 10^{-3+1} = 0.0005
$$

Se observa que las cantidades 1 y 2 son muy diferentes en cuanto a magnitud; no obstante, el m\'aximo error absoluto presente en cada una de ellas es igual.

Es importante establecer que en la realizaci\'on de c\'alculos no es trascendente conocer el signo algebraico de los errores, lo importante es conocer la diferencia entre los valores de trabajo, es decir, su distancia en valor absoluto. Esta debe ser siempre menor que una cantidad de error permitida para considerar v\'alido el c\'alculo. En la pr\'actica de la Ingenier\'ia, a esta cantidad de error permitida se le conoce como \textbf{tolerancia}.

Las tolerancias suelen expresarse en forma de porcentajes (errores relativos) y casi siempre est\'an enfocadas hacia el n\'umero de cifras significativas que deben utilizarse en la aproximaci\'on. Se puede demostrar que si el siguiente criterio se cumple, puede tenerse la seguridad de que el resultado es correcto en al menos $n$ cifras significativas:

$$
\text{tol} = (0.5 \times 10^{2 - n}) \quad [\%]
$$

\noindent \textbf{Ejemplo.} Calcule el valor de la funci\'on $e^1$ utilizando la serie:

$$
e^x = \sum_{i=0}^{\infty} \frac{x^i}{i!} = 1 + x + \frac{x^2}{2!} + \frac{x^3}{3!} + \ldots
$$

\noindent variando el n\'umero de t\'erminos de la serie utilizados y utilizando cinco cifras exactas. Para este ejemplo, la tolerancia es $tol = 0.5 \cdot 10^{-5} = 0.00005$. Si se considera como valor real el obtenido directamente de una calculadora, el resultado se muestra en la siguiente tabla:

\begin{table}[H]
\centering
\caption{Errores en el c\'alculo de series infinitas}
\begin{tabular}{|c|c|c|}
\hline
\textbf{T\'ermino} & \textbf{Valor} & \textbf{Error} \\
\hline
1 & 1 & 1.71828 \\
2 & 2 & 0.71828 \\
3 & 2.5 & 0.21828 \\
4 & 2.66667 & 0.15161 \\
5 & 2.70833 & 0.00995 \\
6 & 2.71667 & 0.00161 \\
7 & 2.71806 & 0.00022 \\
\hline
\end{tabular}
\end{table}

Una segunda aportaci\'on del estudio de McCracken es el establecimiento de un proceso para medir la propagaci\'on de los errores ocasionados por el uso de la aritm\'etica de punto flotante. A partir del establecimiento del m\'aximo error absoluto cometido y de la operaci\'on aritm\'etica utilizada se demuestra que en este tipo de procesos el orden en que se realiza las operaciones s\'i modifica el resultado.

\noindent \textbf{Ejemplo.} Sumar las cantidades siguientes, primero en orden ascendente y luego en orden descendente, considerando una mantisa normalizada de cuatro d\'igitos as\'i como redondeo sim\'etrico en cada operaci\'on intermedia; por otra parte, realice la suma exacta (con todos los d\'igitos posibles en una calculadora) y considere este valor como exacto. Calcule el error relativo que se comete en cada caso.

\begin{enumerate}
  \item 2.6855$\times10^4$
  \item 0.9567$\times10^3$
  \item 0.0053$\times10^2$
  \item 0.1111$\times10^1$
\end{enumerate}

Para las alternativas solicitadas, en las tablas respectivas se mostrar\'a la cantidad normalizada as\'i como el subtotal, es decir, la suma redondeada en una mantisa normalizada de tama\~no 4.

\bigskip

\noindent El valor \textit{exacto}, obtenido a trav\'es de una calculadora es: \textbf{3643.341}.

\bigskip

El procedimiento consiste en normalizar las cantidades (igualando el exponente de la base diez en cada cantidad) y sumarlas en forma ascendente o descendente, seg\'un sea el caso; en la suma de cada par de cantidades, se redondea el resultado manteniendo la mantisa en el tama\~no preestablecido.


\begin{table}[H]
\centering
\caption{Cuadro 2: Suma descendente}
\begin{tabular}{|c|c|c|}
\hline
\textbf{Cantidad} & \textbf{Cantidad Normalizada} & \textbf{Subtotal} \\
\hline
$0{,}2685r10^4$ & $0{,}2685r10^4$ & $0{,}2685r10^4$ \\
$0{,}9567r10^3$ & $0{,}09567r10^4$ & $0{,}3642r10^4$ \\
$0{,}0053r10^2$ & $0{,}0001r10^4$ & $0{,}3643r10^4$ \\
$0{,}1111r10^1$ & $0{,}0001r10^4$ & $0{,}3644r10^4$ \\
\hline
\end{tabular}
\end{table}

\begin{table}[H]
\centering
\caption{Cuadro 3: Suma ascendente}
\begin{tabular}{|c|c|c|}
\hline
\textbf{Cantidad} & \textbf{Cantidad Normalizada} & \textbf{Subtotal} \\
\hline
$0{,}1111r10^1$ & $0{,}1111r10^1$ & $0{,}1111r10^1$ \\
$0{,}0053r10^2$ & $0{,}0530r10^1$ & $0{,}1614r10^1$ \\
$0{,}9567r10^3$ & $95{,}67r10^1$ & $95{,}8341r10^1$ \\
$0{,}2685r10^4$ & $268{,}5r10^1$ & $363{,}3341r10^1$ \\
\hline
\end{tabular}
\end{table}

\begin{table}[H]
\centering
\caption{Cuadro 4: Comparaci\'on de resultados}
\begin{tabular}{|l|c|c|c|}
\hline
\textbf{Resultado} & \textbf{Error absoluto} & \textbf{Error relativo} \\
\hline
Valor exacto & 3643,341 & -- \\
Suma descendente & $0{,}3664r10^4$ & 20,659 & $0{,}56703\%$ \\
Suma ascendente & $363{,}3341r10^1$ & 10 & $0{,}27447\%$ \\
\hline
\end{tabular}
\end{table}

\medskip

En el cuadro dos se muestra la suma ascendente y en el cuadro tres se muestra la suma en forma descendente. Finalmente, los resultados se incluyen en el cuadro cuatro.

Finalmente, este estudio arroja tres importantes conclusiones que deben considerarse en el dise\~no de algoritmos para ejecutar m\'etodos num\'ericos.

\textbf{Las conclusiones de McCracken son las siguientes:}
\begin{enumerate}
    \item Cuando se van a sumar y/o restar n\'umeros, se debe trabajar siempre con los m\'as peque\~nos primero.
    \item De ser posible, evitar la sustracci\'on de dos n\'umeros aproximadamente iguales. Una expresi\'on que contenga dicha sustracci\'on puede a menudo ser reescrita para evitarla.
    \item Una expresi\'on del tipo $(a - b - c)$ puede reescribirse de la forma $a - \frac{b + c}{e}$ como $\frac{ae - (b + c)}{e}$. Si hay n\'umeros aproximadamente iguales dentro del par\'entesis, ejecutar la resta antes que la multiplicaci\'on. Esto evitar\'a complicar el problema con errores de redondeo adicionales.
    \item Cuando no se aplica ninguna de las reglas anteriores, debe minimizarse el n\'umero de operaciones aritm\'eticas.
\end{enumerate}

\noindent Queda como labor voluntaria analizar estas conclusiones y comprobar la forma en que fueron obtenidas.
\section{Convergencia y estabilidad de un m\'etodo num\'erico}

Matem\'aticamente, la \textbf{convergencia} es la propiedad de algunas sucesiones y series de tender progresivamente a un l\'imite, de tal forma, si este l\'imite existe, se dice que la sucesi\'on o la serie \textit{converge}. 

En forma an\'aloga, si un m\'etodo num\'erico en su funcionamiento iterativo nos proporciona aproximaciones cada vez m\'as cercanas al valor buscado, se dice que el m\'etodo converge. La convergencia se mide a trav\'es de los errores; si el error entre dos aproximaciones sucesivas se reduce, el m\'etodo converge; se debe cumplir que:

$$
|x_n - x_{n-1}| \leq |x_{n-1} - x_{n-2}|
$$

Es decir, la diferencia en\'esima \( (x_n - x_{n-1}) $ debe ser menor que la diferencia \( (n-1)$\'esima \( x_{n-1} - x_{n-2} $.

Se dice que un sistema (o un proceso) es \textbf{estable} si a peque\~nas variaciones en la entrada o en la excitaci\'on corresponden peque\~nas variaciones en la salida o en la respuesta. La estabilidad de un m\'etodo num\'erico tiene que ver con la manera en que los errores num\'ericos se propagan a lo largo del algoritmo. 

Cuando un m\'etodo converge, lo m\'as deseable es que en los resultados que se obtengan, los niveles de error se disminuyan en la forma m\'as r\'apida posible. Sin embargo, ocurre que durante la operaci\'on del algoritmo, ya sea por el manejo de los datos num\'ericos o bien por la naturaleza propia del modelo matem\'atico con el que se est\'e trabajando, los errores entre aproximaciones no disminuyan en forma progresiva, sino que incluso aumenten en alguna etapa del proceso para despu\'es reducirse mostrando un comportamiento aleatorio.

La \textbf{robustez} de un m\'etodo num\'erico radica en su convergencia y su estabilidad. Pueden utilizarse m\'etodos cuya prueba de convergencia indique la pertinencia de su uso, pero que durante su aplicaci\'on se obtengan resultados inestables que repercutan en el n\'umero de iteraciones y en consecuencia en el tiempo invertido en la soluci\'on. El ideal lo constituyen m\'etodos que a la vez de ser convergentes resulten estables.

\section{Aproximaci\'on de funciones por medio de polinomios}

Particularmente en el manejo de funciones trascendentes, la soluci\'on anal\'itica de problemas puede ser dif\'icil y complicada; incluso esta situaci\'on podr\'ia ocurrir en el \'ambito de la soluci\'on num\'erica. Cuando esto ocurre, una herramienta de soluci\'on posible es utilizar una representaci\'on aproximada de la funci\'on a trav\'es de funciones m\'as sencillas. Algunas de estas aproximaciones son:

\begin{itemize}
    \item Funciones peri\'odicas (senos y cosenos) a trav\'es de las series de Fourier
    \item Segmentar la funci\'on a trav\'es de una secuencia de l\'ineas rectas
    \item La series de Taylor
\end{itemize}

La expansi\'on en series de Taylor busca obtener una aproximaci\'on a \( f(x) $ a trav\'es de un polinomio de la forma:

\begin{equation}
P(x) = a_n x^n + a_{n-1} x^{n-1} + a_{n-2} x^{n-2} + a_{n-3} x^{n-3} + \ldots + a_1 x + a_0
\end{equation}

En la vecindad del punto \( x = x_0 $ para sus primeras \( n $ derivadas. Por lo anterior, se requiere que \( f(x) $ tenga \( n - 1 $ derivadas en el intervalo \( a \leq x \leq b $, es decir que:

\begin{align}
P(x_0) &= f(x_0) \\
P'(x_0) &= f'(x_0) \\
P''(x_0) &= f''(x_0) \\
&\vdots \\
P^{(n)}(x_0) &= f^{(n)}(x_0)
\end{align}

Es necesario determinar los coeficientes \( a_i $ del polinomio (1) y las derivadas evaluadas en \( x_0 = 0 $:

\begin{align}
P(0) &= a_0 \Rightarrow a_0 = f(0) \\
P'(0) &= a_1 \Rightarrow a_1 = f'(0) \\
P''(0) &= 2a_2 \Rightarrow a_2 = \frac{1}{2} f''(0) \\
P'''(0) &= 3!a_3 \Rightarrow a_3 = \frac{1}{3!} f'''(0) \\
&\vdots \\
P^{(k)}(0) &= k! a_k \Rightarrow a_k = \frac{1}{k!} f^{(k)}(0)
\end{align}

para \( n = 0, 1, 2, 3, 4, \ldots, n $.

Sustituyendo (2) y (3) en (1):

\begin{align}
P(x) &= f(0) + f'(0)x + \frac{f''(0)}{2!}x^2 + \frac{f'''(0)}{3!}x^3 + \cdots + \frac{f^{(n)}(0)}{n!}x^n = \sum_{k=0}^n \frac{x^k}{k!} f^{(k)}(0)
\end{align}

La expresi\'on (6) representa la \textbf{Serie de McLaurin}.

En un caso particular, es probable que se requiera que el polinomio \( P(X) $ sea igual a la funci\'on \( f(x) $ en un punto \( X $ diferente de cero, es decir, \( X = a \neq 0 $, se procede de la misma forma:

\begin{align}
P(x_a) &= f(x_a) \\
P'(x_a) &= f'(x_a) \\
P''(x_a) &= f''(x_a) \\
&\vdots \\
P^{(n)}(x_a) &= f^{(n)}(x_a)
\end{align}

Esta consideraci\'on genera un crecimiento de la abscisa, por lo cual la expresi\'on general queda:

\begin{align}
P(x) = \sum_{k=0}^n \frac{(x - a)^k}{k!} f^{(k)}(a)
\end{align}

A la ecuaci\'on (5) se le conoce como \textbf{polinomio de Taylor} de grado \( n $ para la funci\'on \( f(x) $ en torno de \( x = a $.

\textbf{Ejemplo.} Calcule los polinomios de Taylor de grados 1 y 3 para \( f(x) = \cos x $ en \( x = \frac{\pi}{7} $.


\begin{align}
P_1(x) &= \sum_{k=0}^1 \frac{(x - \tfrac{\pi}{7})^k}{k!} f^{(k)}\left( \tfrac{\pi}{7} \right) 
= \frac{(x - \tfrac{\pi}{7})^0}{0!} f^{(0)}\left( \tfrac{\pi}{7} \right) + \frac{(x - \tfrac{\pi}{7})^1}{1!} f^{(1)}\left( \tfrac{\pi}{7} \right) \tag{7} \\
\\
P_3(x) &= \sum_{k=0}^3 \frac{(x - \tfrac{\pi}{7})^k}{k!} f^{(k)}\left( \tfrac{\pi}{7} \right) 
= \frac{(x - \tfrac{\pi}{7})^0}{0!} f^{(0)}\left( \tfrac{\pi}{7} \right) + \frac{(x - \tfrac{\pi}{7})^1}{1!} f^{(1)}\left( \tfrac{\pi}{7} \right) \\
&\quad + \frac{(x - \tfrac{\pi}{7})^2}{2!} f^{(2)}\left( \tfrac{\pi}{7} \right) + \frac{(x - \tfrac{\pi}{7})^3}{3!} f^{(3)}\left( \tfrac{\pi}{7} \right) \tag{8}
\end{align}

Las derivadas evaluadas en \( \tfrac{\pi}{7} $:

\begin{align}
f(x) &= \cos(x) \Rightarrow f\left( \tfrac{\pi}{7} \right) = 0 \\
f'(x) &= -\sin(x) \Rightarrow f'\left( \tfrac{\pi}{7} \right) = -1 \\
f''(x) &= -\cos(x) \Rightarrow f''\left( \tfrac{\pi}{7} \right) = 0 \\
f'''(x) &= \sin(x) \Rightarrow f'''\left( \tfrac{\pi}{7} \right) = 1 \tag{9}
\end{align}

Sustituyendo (9) en (7) y en (8):

\begin{align}
P_1(x) &= 1 \cdot 0 + \frac{(x - \tfrac{\pi}{7})}{1} \cdot (-1) = -x + \tfrac{\pi}{7} \\
\\
P_3(x) &= 1 \cdot 0 + \frac{(x - \tfrac{\pi}{7})}{1} \cdot (-1) + \frac{(x - \tfrac{\pi}{7})^2}{2} \cdot 0 
+ \frac{(x - \tfrac{\pi}{7})^3}{6} \cdot 1 \\
&= 0.167x^3 - 0.785x^2 + 0.234x + 0.925 \tag{1}
\end{align}

La figura (1) muestra gr\'aficamente cada una de las aproximaciones a \( f(x) $.

\subsection*{6.1. Residuo del polinomio de Taylor}

No se debe perder de vista que el polinomio de Taylor es una aproximaci\'on a la funci\'on \( f(x) $; conlleva un error que no suele ser considerado pero que en funci\'on de su orden pudiera llegar a ser significativo. De tal forma, a la expresi\'on:

\begin{equation}
f(x) = \sum_{k=0}^n \frac{(x - a)^k}{k!} f^{(k)}(a) + E_n(x) \tag{10}
\end{equation}

se le conoce como \textbf{F\'ormula de Taylor con residuo}. Para calcular \( E_n(x) $ se eval\'ua la ecuaci\'on (6.1) para diversos \'ordenes:

\textbf{Primer orden:}

\begin{equation}
f(x) = f(a) + (x - a)f'(a) + E_1(x)
\end{equation}

\begin{figure}[H]
    \centering
    \includegraphics[width=0.6\textwidth]{figura1_aproximaciones.png}
    \caption{Aproximaciones a la funci\'on \( \sen(x) $}
\end{figure}

Despejando \( E_1(x) $:

\begin{equation}
E_1(x) = f(x) - f(a) - (x - a)f'(a) \tag{11}
\end{equation}

La ecuaci\'on (11) puede expresarse de forma integral:

\begin{equation}
E_1(x) = \int_a^x f'(t)dt - f'(a) \int_a^x dt 
= \int_a^x \left[ f'(t) - f'(a) \right] dt \tag{12}
\end{equation}

Integrando por partes:

\begin{align*}
u &= f'(t) - f'(a) \quad & dv &= dt \\
du &= f''(t)dt \quad & v &= t
\end{align*}

\begin{align*}
E_1(x) &= \left[ \left( f'(t) - f'(a) \right) \cdot t \right]_a^x - \int_a^x t \cdot f''(t) dt \\
E_1(x) &= \left[ f'(x) - f'(a) \right] \cdot x - \left[ f'(a) - f'(a) \right] \cdot a - \int_a^x t \cdot f''(t) dt \\
E_1(x) &= (x - a) f'(x) - \int_a^x t \cdot f''(t) dt
\end{align*}

o, m\'as com\'unmente:

\begin{equation}
E_1(x) = x \int_a^x f''(t) dt - \int_a^x t \cdot f''(t) dt
\end{equation}

\begin{equation}
E_1(x) = \int_a^x (x - t) f''(t) dt \tag{13}
\end{equation}

Para un segundo orden el resultado es:

\begin{equation}
E_2(x) = \frac{1}{2!} \int_a^x (x - t)^2 f^{(3)}(t) dt \tag{14}
\end{equation}

Y para el \( n $-\'esimo orden:

\begin{equation}
E_n(x) = \frac{1}{n!} \int_a^x (x - t)^n f^{(n+1)}(t) dt \tag{15}
\end{equation}

La ecuaci\'on (15) es el error cometido al aproximar la funci\'on \( f(x) $ con un polinomio de Taylor de grado \( n $.

\subsection*{6.2 Estimaci\'on del error de la aproximaci\'on de Taylor}

Dado que la aproximaci\'on de Taylor representa una serie con un n\'umero infinito de t\'erminos, no es posible encontrar un valor exacto para \( E_n(x) $, por lo que es necesario hacer algunas consideraciones: sup\'ongase que \( m $ y \( M $ son los valores m\'inimo y m\'aximo respectivamente, que adquiere la funci\'on \( f^{(n+1)}(t) $ en el intervalo \( [a,x] $. Sustituyendo estos supuestos en la ecuaci\'on (15):

$$
[E_n(x)]_m = \frac{m (x - a)^{(n+1)}}{(n+1)!} \quad ; \quad [E_n(x)]_M = \frac{M (x - a)^{(n+1)}}{(n+1)!}
$$

Ambas expresiones son las cotas de error, es decir:

$$
\frac{m (x - a)^{(n+1)}}{(n+1)!} \leq E_n(x) \leq \frac{M (x - a)^{(n+1)}}{(n+1)!}
$$

Establecer los valores de $ m $ y $ M $ es un problema complicado. En aplicaciones reales se toma un criterio pr\'actico que consiste en evaluar el t\'ermino $ n + 1 $ de la serie en alg\'un punto de inter\'es que est\'e en la vecindad de $ x_0 = a $. Por ejemplo, se desea estimar el error cometido al aproximar $ y = \sen(x) $, para $ x_0 = 0 $, a trav\'es de un polinomio de sexto orden. El polinomio de Taylor de $ \sen(x) $ es ampliamente conocido:

$$
\sen(x) = x - \frac{x^3}{3!} + \frac{x^5}{5!}
$$

El siguiente t\'ermino de la serie $ (n+1) $ para estimar el error es: $ E_6 \leq \frac{x^7}{7!} $. Se define como una desigualdad porque el valor real del error estar\'a m\'as cerca del punto pivote, en este caso de $ x_0 = 0 $, por lo que el error ser\'a menor. Esto se comprueba con los siguientes valores de $ x_0 $:

$$
\begin{aligned}
x = \pi \quad &\Rightarrow \quad E_6(\pi) = \frac{\pi^7}{7!} \approx 0.5993 \\
x = \frac{\pi}{2} \quad &\Rightarrow \quad E_6\left(\frac{\pi}{2}\right) = \frac{(\pi/2)^7}{7!} \approx 0.0047 \\
x = \frac{\pi}{4} \quad &\Rightarrow \quad E_6\left(\frac{\pi}{4}\right) = \frac{(\pi/4)^7}{7!} \approx 0.000037
\end{aligned}
$$
Conforme el valor de $ x_0 $ se acerca al punto en el cual se defini\'o el polinomio (en este caso $ x_0 = 0 $) el error disminuye.

