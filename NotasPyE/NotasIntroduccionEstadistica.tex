%===========================================
\section{Estad\'istica: Fundamentos}
%===========================================
%____________________________________________________________
%____________________________________________________________
%
%____________________________________________________________
\subsection*{Definici\'on de Estad\'istica}
%____________________________________________________________
La Estad\'istica es una ciencia formal que estudia la recolecci\'on, an\'alisis e interpretaci\'on de datos de una muestra representativa, ya sea para ayudar en la toma de decisiones o para explicar condiciones regulares o irregulares de alg\'un fen\'omeno o estudio aplicado, de ocurrencia en forma aleatoria o condicional. Sin embargo, la estad\'istica es m\'as que eso, es decir, es el veh\'iculo que permite llevar a cabo el proceso relacionado con la investigaci\'on cient\'ifica. Es transversal a una amplia variedad de disciplinas, desde la f\'isica hasta las ciencias sociales, desde las ciencias de la salud hasta el control de calidad. Se usa para la toma de decisiones en \'areas de negocios o instituciones gubernamentales.\bigskip

La Estad\'istica es mucho m\'as que s\'olo n\'umeros apilados y gr\'aficas bonitas. Es una ciencia con tanta antiguedad como la escritura, y es por s\'i misma auxiliar de todas las dem\'as ciencias. Los mercados, la medicina, la ingenier\'ia, los gobiernos, etc.  La Estad\'istica que conocemos hoy en d\'ia debe gran parte de su realizaci\'on a los trabajos matem\'aticos de aquellos hombres que desarrollaron la teor\'ia de las probabilidades, con la cual se adhiri\'o a la Estad\'istica a las ciencias formales. 

\begin{Def}
La Estad\'istica es la ciencia cuyo objetivo es reunir una informaci\'on cuantitativa concerniente a individuos, grupos, series de hechos, etc. y deducir de ello gracias al an\'alisis de estos datos unos significados precisos o unas previsiones para el futuro.
\end{Def}

La estad\'istica, en general, es la ciencia que trata de la recopilaci\'on, organizaci\'on presentaci\'on, an\'alisis e interpretaci\'on de datos num\'ericos con el fin de realizar una toma de decisi\'on m\'as efectiva. Otros autores la definen como la expresi\'on cuantitativa del conocimiento dispuesta en forma adecuada para el escrutinio y an\'alisis.

\begin{itemize}
\item la palabra estad\'istica, en primer t\'ermino se usa para referirse a la informaci\'on estad\'istica; 
\item tambi\'en se utiliza para referirse al conjunto de t\'ecnicas y m\'etodos que se utilizan para analizar la informaci\'on estad\'istica; y
\item  el t\'ermino estad\'istico, en singular y en masculino, se refiere a una medida derivada de una muestra.
\end{itemize}

Los m\'etodos estad\'isticos tradicionalmente se utilizan para prop\'ositos descriptivos, para organizar y resumir datos num\'ericos. La estad\'istica descriptiva, por ejemplo trata de la tabulaci\'on de datos, su presentaci\'on en forma gr\'afica o ilustrativa y el c\'alculo de medidas descriptivas.

%____________________________________________________________
\subsection{Historia de la Estad\'istica}
%____________________________________________________________
Es dif\'icil conocer los or\'igenes de la Estad\'istica. Desde los comienzos de la civilizaci\'on han existido formas sencillas de estad\'istica, pues ya se utilizaban representaciones gr\'aficas y otros s\'imbolos en pieles, rocas, palos de madera y paredes de cuevas para contar el n\'umero de personas, animales o ciertas cosas. Su origen inicia posiblemente en la isla de Cerde\~na, donde existen monumentos prehist\'oricos pertenecientes a los Nuragas, las primeros habitantes de la isla; estos monumentos constan de bloques de basalto superpuestos sin mortero y en cuyas paredes de encontraban grabados toscos signos que han sido interpretados con mucha verosimilidad como muescas que serv\'ian para llevar la cuenta del ganado y la caza.\medskip

Los babilonios usaban peque\~nas tablillas de arcilla para recopilar datos en tablas sobre la producci\'on agr\'icola y los g\'eneros vendidos o cambiados mediante trueque.  Otros vestigios pueden ser hallados en el antiguo Egipto, cuyos faraones lograron recopilar, hacia el a\~no 3050 antes de Cristo,  datos relativos a la poblaci\'on y la riqueza del pa\'is. De acuerdo al historiador griego Her\'odoto, dicho registro de riqueza y poblaci\'on se hizo con el objetivo de preparar la construcci\'on de las pir\'amides. En el mismo Egipto, Rams\'es II hizo un censo de las tierras con el objeto de verificar un nuevo reparto. \medskip

En el antiguo Israel la Biblia da referencias, en el libro de los N\'umeros, de los datos estad\'isticos obtenidos en dos recuentos de la poblaci\'on hebrea. El rey David por otra parte, orden\'o a Joab, general del ej\'ercito hacer un censo de Israel con la finalidad de conocer el n\'umero de la poblaci\'on.Tambi\'en los chinos efectuaron censos hace m\'as de cuarenta siglos. Los griegos efectuaron censos peri\'odicamente con fines tributarios, sociales (divisi\'on de tierras) y militares (c\'alculo de recursos y hombres disponibles). La investigaci\'on hist\'orica revela que se realizaron 69 censos para calcular los impuestos, determinar los derechos de voto y ponderar la potencia guerrera.\medskip

Pero fueron los romanos, maestros de la organizaci\'on pol\'itica, quienes mejor supieron emplear los recursos de la estad\'istica. Cada cinco a\~nos realizaban un censo de la poblaci\'on y sus funcionarios p\'ublicos ten\'ian la obligaci\'on de anotar nacimientos, defunciones y matrimonios, sin olvidar los recuentos peri\'odicos del ganado y de las riquezas contenidas en las tierras conquistadas. Para el nacimiento de Cristo suced\'ia uno de estos empadronamientos de la poblaci\'on bajo la autoridad del imperio.  Durante los mil a\~nos siguientes a la ca\'ida del imperio Romano se realizaron muy pocas operaciones Estad\'isticas, con la notable excepci\'on de las relaciones de tierras pertenecientes a la Iglesia, compiladas por Pipino el Breve en el 758 y por Carlomagno en el 762 DC. Durante el siglo IX se realizaron en Francia algunos censos parciales de siervos. En Inglaterra, Guillermo el Conquistador recopil\'o el Domesday Book o libro del Gran Catastro para el a\~no 1086, un documento de la propiedad, extensi\'on y valor de las tierras de Inglaterra. Esa obra fue el primer compendio estad\'istico de Inglaterra. Aunque Carlomagno, en Francia; y Guillermo el Conquistador, en Inglaterra, trataron de revivir la t\'ecnica romana, los m\'etodos estad\'isticos permanecieron casi olvidados durante la Edad Media.\medskip

Durante los siglos XV, XVI, y XVII, hombres como Leonardo de Vinci, Nicol\'as Cop\'ernico, Galileo, Neper, William Harvey, Sir Francis Bacon y Ren\'e Descartes, hicieron grandes operaciones al m\'etodo cient\'ifico, de tal forma que cuando se crearon los Estados Nacionales y surgi\'o como fuerza el comercio internacional exist\'ia ya un m\'etodo capaz de aplicarse a los datos econ\'omicos. Para el a\~no 1532 empezaron a registrarse en Inglaterra las defunciones debido al temor que Enrique VII ten\'ia por la peste. M\'as o menos por la misma \'epoca, en Francia la ley exigi\'o a los cl\'erigos registrar los bautismos, fallecimientos y matrimonios. Durante un brote de peste que apareci\'o a fines de la d\'ecada de 1500, el gobierno ingl\'es comenz\'o a publicar estad\'istica semanales de los decesos. Esa costumbre continu\'o muchos a\~nos, y en 1632 estos Bills of Mortality (Cuentas de Mortalidad) conten\'ian los nacimientos y fallecimientos por sexo.\medskip

En 1662, el capit\'an John Graunt us\'o documentos que abarcaban treinta a\~nos y efectu\'o predicciones sobre el n\'umero de personas que morir\'ian de varias enfermedades y sobre las proporciones de nacimientos de varones y mujeres que cabr\'ia esperar. El trabajo de Graunt, condensado en su obra \textit{Natural and
Political Observations...Made upon the Bills of Mortality}, fue un esfuerzo innovador en el an\'alisis estad\'istico.\medskip

Los eruditos del siglo XVII demostraron especial inter\'es por la Estad\'istica Demogr\'afica como resultado de la especulaci\'on sobre si la poblaci\'on aumentaba, decrec\'ia o permanec\'ia est\'atica. En los tiempos modernos tales m\'etodos fueron resucitados por algunos reyes que necesitaban conocer las riquezas monetarias y el potencial humano de sus respectivos pa\'ises. El primer empleo de los datos estad\'isticos para fines ajenos a la pol\'itica tuvo lugar en 1691 y estuvo a cargo de Gaspar Neumann, un profesor alem\'an que viv\'ia en Breslau. Este investigador se propuso destruir la antigua creencia popular de que en los a\~nos terminados en siete mor\'ia m\'as gente que en los restantes, y para lograrlo hurg\'o pacientemente en los archivos parroquiales de la ciudad. Despu\'es de revisar miles de partidas de defunci\'on pudo demostrar que en tales a\~nos no fallec\'ian m\'as personas que en los dem\'as.  Los procedimientos de Neumann fueron conocidos por el astr\'onomo ingl\'es Halley, descubridor del cometa que lleva su nombre, quien los aplic\'o al estudio de la vida humana. Sus c\'alculos sirvieron de base para las tablas de mortalidad que hoy utilizan todas las compa\~n\'ias de seguros. \medskip

Durante el siglo XVII y principios del XVIII, matem\'aticos como Bernoulli, Francis Maseres, Lagrange y Laplace desarrollaron la teor\'ia de probabilidades. No obstante durante cierto tiempo, la teor\'ia de las probabilidades limit\'o su aplicaci\'on a los juegos de azar y hasta el siglo XVIII no comenz\'o a aplicarse a los grandes problemas cient\'ificos. Godofredo Achenwall, profesor de la Universidad de Gotinga, acu\~n\'o en 1760 la palabra estad\'istica, que extrajo del t\'ermino italiano statista (estadista). Cre\'ia, y con sobrada raz\'on, que los datos de la nueva ciencia ser\'ian el aliado m\'as eficaz del gobernante consciente. La ra\'iz remota de la palabra se halla, por otra parte, en el t\'ermino latino status, que significa estado o situaci\'on; Esta etimolog\'ia aumenta el valor intr\'inseco de la palabra, por cuanto la estad\'istica revela el sentido cuantitativo de las m\'as variadas situaciones. Jacques Qu\'etelect es quien aplica las Estad\'isticas a las ciencias sociales. Este interpret\'o la teor\'ia de la probabilidad para su uso en las ciencias sociales y resolver la aplicaci\'on del principio de promedios y de la variabilidad a los fen\'omenos sociales. Qu\'etelect fue el primero en realizar la aplicaci\'on pr\'actica de todo el m\'etodo Estad\'istico, entonces conocido, a las diversas ramas de la ciencia.\medskip

Entretanto, en el per\'iodo del 1800 al 1820 se desarrollaron dos conceptos matem\'aticos fundamentales para la teor\'ia Estad\'istica; la teor\'ia de los errores de observaci\'on, aportada por Laplace y Gauss; y la teor\'ia de los m\'inimos cuadrados desarrollada por Laplace, Gauss y Legendre. A finales del siglo XIX, Sir Francis Gaston ide\'o el m\'etodo conocido por Correlaci\'on, que ten\'ia por objeto medir la influencia relativa de los factores sobre las variables. De aqu\'i parti\'o el desarrollo del coeficiente de correlaci\'on creado por Karl Pearson y otros cultivadores de la ciencia biom\'etrica como J. Pease Norton, R. H. Hooker y G. Udny Yule, que efectuaron amplios estudios sobre la medida de las relaciones. Los progresos m\'as recientes en el campo de la Estad\'istica se refieren al ulterior desarrollo del c\'alculo de probabilidades, particularmente en la rama denominada indeterminismo o relatividad, se ha demostrado que el determinismo fue reconocido en la F\'isica como resultado de las investigaciones at\'omicas y que este principio se juzga aplicable tanto a las ciencias sociales como a las f\'isicas.
%____________________________________________________________
\subsection*{Etapas de Desarrollo de la Estad\'istica}
%____________________________________________________________
La historia de la estad\'istica est\'a resumida en tres grandes etapas o fases.
\begin{itemize}
\item Primera Fase.  \textbf{Los Censos:} Desde el momento en que se constituye una autoridad pol\'itica, la idea de inventariar de una forma m\'as o menos regular la poblaci\'on y las riquezas existentes en el territorio est\'a ligada a la conciencia de soberan\'ia y a los primeros esfuerzos administrativos.
\item Segunda Fase.  \textbf{De la Descripci\'on de los Conjuntos a la Aritm\'etica Pol\'itica:} Las ideas mercantilistas extra\~nan una intensificaci\'on de este tipo de investigaci\'on. Colbert multiplica las encuestas sobre art\'iculos manufacturados, el comercio y la poblaci\'on: los intendentes del Reino env\'ian a Par\'is sus memorias. Vauban, m\'as conocido por sus fortificaciones o su Dime Royale, que es la primera propuesta de un impuesto sobre los ingresos, se se\~nala como el verdadero precursor de los sondeos. M\'as tarde, Buf\'on se preocupa de esos problemas antes de dedicarse a la historia natural. La escuela inglesa proporciona un nuevo progreso al superar la fase puramente descriptiva. Sus tres principales representantes son Graunt, Petty y Halley. El pen\'ultimo es autor de la famosa Aritm\'etica Pol\'itica. Chaptal, ministro del interior franc\'es, publica en 1801 el primer censo general de poblaci\'on, desarrolla los estudios industriales, de las producciones y los cambios, haci\'endose sistem\'aticos durantes las dos terceras partes del siglo XIX.
\item Tercera Fase.  \textbf{Estad\'istica y C\'alculo de Probabilidades:} El c\'alculo de probabilidades se incorpora r\'apidamente como un instrumento de an\'alisis extremadamente poderoso para el estudio de los fen\'omenos econ\'omicos y sociales y en general para el estudio de fen\'omenos cuyas causas son demasiados complejas para conocerlos totalmente y hacer posible su an\'alisis.
\end{itemize}
%____________________________________________________________
\subsection*{Divisi\'on de la Estad\'istica}
%____________________________________________________________

La Estad\'istica para su mejor estudio se ha dividido en dos grandes ramas: \textbf{la Estad\'istica Descriptiva y la Estad\'istica Inferencial}.
\begin{itemize}
\item \textbf{Descriptiva:} consiste sobre todo en la presentaci\'on de datos en forma de tablas y gr\'aficas. Esta comprende cualquier actividad relacionada con los datos y est\'a dise\~nada para resumir o describir los mismos sin factores pertinentes adicionales; esto es, sin intentar inferir nada que vaya m\'as all\'a de los datos, como tales.
\item \textbf{Inferencial:} se realiza a partir de muestras,  observaciones hechas s\'olo acerca de una parte de un conjunto numeroso de elementos y esto implica que su an\'alisis requiere de generalizaciones que van m\'as all\'a de los datos. Como consecuencia, la caracter\'istica m\'as importante del reciente crecimiento de la estad\'istica ha sido un cambio en el \'enfasis de los m\'etodos que describen a m\'etodos que sirven para hacer generalizaciones. La Estad\'istica Inferencial investiga o analiza una poblaci\'on a partir de una muestra.
\end{itemize}

%____________________________________________________________
\subsection*{Errores Estad\'isticos Comunes}
%____________________________________________________________

Al momento de recopilar los datos que ser\'an procesados es posible cometer errores as\'i como durante los c\'omputos de los mismos. No obstante, hay otros errores que no tienen nada que ver con la digitaci\'on y que no son tan f\'acilmente identificables. Algunos de \'estos errores son:
\begin{itemize}
\item\textbf{Sesgo:} Es imposible ser completamente objetivo o no tener ideas preconcebidas antes de comenzar a estudiar un problema, y existen muchas maneras en que una perspectiva o estado mental pueda influir en la recopilaci\'on y en el an\'alisis de la informaci\'on. En estos casos se dice que hay un sesgo cuando el individuo da mayor peso a los datos que apoyan su opini\'on que a aquellos que la contradicen. Un caso extremo de sesgo ser\'ia la situaci\'on donde primero se toma una decisi\'on y despu\'es se utiliza el an\'alisis estad\'istico para justificar la decisi\'on ya tomada.
\item\textbf{Datos No Comparables:} el establecer comparaciones es una de las partes m\'as importantes del an\'alisis estad\'istico, pero es extremadamente importante que tales comparaciones se hagan entre datos que sean comparables.
\item \textbf{Proyecci\'on descuidada de tendencias:} la proyecci\'on simplista de tendencias pasadas hacia el futuro es uno de los errores que m\'as ha desacreditado el uso del an\'alisis estad\'istico.
\item\textbf{Muestreo Incorrecto:} en la mayor\'ia de los estudios sucede que el volumen de informaci\'on disponible es tan inmenso que se hace necesario estudiar muestras, para derivar conclusiones acerca de la poblaci\'on a que pertenece la muestra. Si la muestra se selecciona correctamente, tendr\'a b\'asicamente las mismas propiedades que la poblaci\'on de la cual fue extra\'ida; pero si el muestreo se realiza incorrectamente, entonces puede suceder que los resultados no signifiquen nada
\end{itemize}


%____________________________________________________________
\subsection*{T\'erminos comunes utilizados en estad\'istica}
%____________________________________________________________


\begin{itemize}
  \item Variable: Caracter\'istica o fen\'omeno que puede tomar distintos valores.
  \item Dato: Mediciones o cualidades que han sido recopiladas como resultado de observaciones.
  \item Poblaci\'on: \'area o conjunto del cual son extra\'idos los datos; es el conjunto de elementos o individuos que poseen una caracter\'istica com\'un y medible acerca de la cual se desea informaci\'on. Tambi\'en llamada \emph{universo}.
  \item Muestra: Subconjunto de la poblaci\'on, seleccionado de acuerdo con una regla o plan de muestreo.
  \item Censo: Recopilaci\'on de todos los datos de inter\'es para la investigaci\'on de la poblaci\'on.
  \item Estad\'istica: Funci\'on o f\'ormula que depende de los datos de la muestra (es variable).
  \item Par\'ametro: Caracter\'istica medible de la poblaci\'on.
\end{itemize}

\begin{Ejem}
La universidad est\'a interesada en determinar el ingreso de las familias de sus alumnos.
\begin{itemize}
  \item \textbf{Variable:} Ingreso \emph{per c\'apita} de las familias.
  \item \textbf{Dato:} Ingreso \emph{per c\'apita} de la familia de un alumno espec\'ifico.
  \item \textbf{Poblaci\'on:} Las familias de todos los alumnos de la universidad.
  \item \textbf{Estad\'istica:} Ingreso \emph{per c\'apita} promedio de las familias seleccionadas en la muestra.
  \item \textbf{Par\'ametro:} Ingreso \emph{per c\'apita} promedio de la poblaci\'on.
\end{itemize}
\end{Ejem}

%____________________________________________________________
\subsubsection*{Muestreo}
%____________________________________________________________

Una muestra es representativa en la medida en que es imagen de la poblaci\'on. En general, el tama\~no de una muestra depende principalmente de: Nivel de precisi\'on deseado, Recursos disponibles, Tiempo involucrado en la investigaci\'on. Se debe considerar: La poblaci\'on, Los par\'ametros a medir.

Tipos de muestreo:
\begin{itemize}
\item \textbf{Muestreo aleatorio simple} Es un m\'etodo de selecci\'on de $n$ unidades extra\'idas de $N$, de tal manera que cada muestra posible tiene la misma probabilidad de ser escogida. En la pr\'actica, se enumeran las unidades de $1$ a $N$, y a continuaci\'on se seleccionan $n$ n\'umeros aleatorios entre $1$ y $N$, ya sea de tablas o de una urna con fichas numeradas.

\item \textbf{Muestreo estratificado aleatorio} Se usa cuando la poblaci\'on est\'a agrupada en pocos estratos (cada uno con muchas entidades). Consiste en sacar una muestra aleatoria simple de cada uno de los estratos, generalmente de tama\~no proporcional al estrato.

\item \textbf{Muestreo sistem\'atico} Se utiliza cuando las unidades de la poblaci\'on est\'an ordenadas. Para seleccionar una muestra de $n$ unidades, se divide la poblaci\'on en $n$ subpoblaciones de tama\~no $K = N/n$. Se toma al azar una unidad de las primeras $K$ y, a partir de ah\'i, cada $K$-\'esima unidad. Si $n_0$ es la primera unidad seleccionada, la muestra es:
$$\{\, n_0,\; n_0+K,\; n_0+2K,\; \ldots,\; n_0+(n-1)K \,\}.$$

\item \textbf{Muestreo por conglomerado} Se emplea cuando la poblaci\'on est\'a dividida en grupos o conglomerados peque\~nos. Consiste en obtener una muestra aleatoria simple de conglomerados y luego \textsc{censar} cada uno de \'estos.

\item \textbf{Muestreo en dos etapas (biet\'apico)} La muestra se toma en dos pasos:
\begin{itemize}
  \item Seleccionar una muestra de unidades primarias.
  \item Seleccionar una muestra de elementos a partir de cada unidad primaria escogida.
\end{itemize}
\end{itemize}


%-----------------------------------
\subsubsection*{Variables}
%-----------------------------------

Las variables se pueden clasificar en dos grandes grupos:

\begin{itemize}

\item \textbf{Variables categ\'oricas} Son aquellas que pueden ser representadas a trav\'es de s\'imbolos, letras o palabras. Los valores que toman se denominan \emph{categor\'ias}, y los elementos que pertenecen a estas categor\'ias se consideran id\'enticos respecto a la caracter\'istica que se est\'a midiendo.

\begin{Ejem} Variable: Profesi\'on.  Valores posibles:Programador, T\'ecnico en Control de Alimentos, T\'ecnico en Prevenci\'on de Riesgos, T\'ecnico en Control del Medio Ambiente, Qu\'imico Anal\'itico, T\'ecnico Mec\'anico,  Etc.
\end{Ejem}

Las variables categ\'oricas se dividen en dos tipos:
\begin{itemize}
  \item \textbf{Ordinales:} Las categor\'ias tienen un orden impl\'icito; admiten grados de calidad (hay relaci\'on total entre categor\'ias). \\
\begin{Ejem} Variable: Nivel de estudio de Ense\~nanza B\'asica. Valores: Primero B\'asico, Segundo B\'asico, Tercero B\'asico, \dots, Octavo B\'asico. A pesar de que admite grados de calidad, no es posible cuantificar la diferencia entre niveles adyacentes.
\end{Ejem}
  \item \textbf{Nominales:} No existe una relaci\'on de orden entre categor\'ias.
\end{itemize}

\item \textbf{Variables num\'ericas} Son aquellas que pueden tomar valores num\'ericos exclusivamente (mediciones). Se dividen en dos tipos: \emph{discretas} y \emph{continuas}.
\begin{itemize}
  \item \textbf{Discretas:} Toman valores en un conjunto finito o infinito numerable. \\
  \emph{Ejemplo (Variable: N\'umero de sillas por sala).} Valores: $0,1,2,3,\dots,n$.
  \item \textbf{Continuas:} Toman valores en un subconjunto de los n\'umeros reales, t\'ipicamente un intervalo. \\
  \emph{Ejemplo (Variable: Temperatura de Ixtapa).} Valores entre $25^\circ$ y $35^\circ$.
\end{itemize}
\end{itemize}
%-----------------------------------
\subsection{Inferencia, estimaci\'on y contraste de hip\'otesis}
%-----------------------------------
Los m\'etodos b\'asicos de la estad\'istica inferencial son la \textbf{estimaci\'on} y el \textbf{contraste de hip\'otesis}; juegan un papel fundamental en la investigaci\'on.
\begin{itemize}
  \item Calcular par\'ametros de la distribuci\'on de medias o proporciones muestrales de tama\~no $n$, extra\'idas de una poblaci\'on de media y varianza conocidas.
  \item Estimar la media o la proporci\'on de una poblaci\'on a partir de la media o proporci\'on muestral.
  \item Utilizar distintos tama\~nos muestrales para controlar la confianza y el error admitido.
  \item Contrastar resultados obtenidos a partir de muestras.
  \item Visualizar gr\'aficamente, mediante las respectivas curvas normales, las estimaciones realizadas.
\end{itemize}

En la mayor\'ia de las investigaciones resulta imposible estudiar a todos los individuos de la poblaci\'on (por costo o inaccesibilidad). Mediante la inferencia estad\'istica se obtienen conclusiones para una poblaci\'on no observada en su totalidad, a partir de estimaciones o res\'umenes num\'ericos efectuados sobre la base informativa extra\'ida de una muestra. En definitiva: a partir de una poblaci\'on se extrae una muestra con alguno de los m\'etodos existentes; con sus datos se generan \emph{estad\'isticos} para realizar estimaciones o contrastes poblacionales.

Existen dos formas de estimar par\'ametros:
\begin{itemize}
  \item \textbf{Estimaci\'on puntual:} con base en los datos muestrales, se propone un \'unico valor para el par\'ametro.
  \item \textbf{Estimaci\'on por intervalo de confianza:} se determina un intervalo dentro del cual se encuentra el valor del par\'ametro con cierta probabilidad.
\end{itemize}

Si el objetivo del tratamiento inferencial es generalizar sobre poblaciones no observadas a partir de una parte de la poblaci\'on, la muestra debe ser \textbf{representativa y aleatoria}. Adem\'as, el tama\~no muestral depende de m\'ultiples factores: recursos (dinero y tiempo), importancia del tema, confiabilidad esperada, caracter\'isticas del fen\'omeno, etc. A partir de la muestra se estiman par\'ametros como la media, varianza, desviaci\'on est\'andar o la forma de la distribuci\'on.
%____________________________________________________________
\subsection{Conceptos b\'asicos}
%____________________________________________________________
Recordemos 	los conceptos elementales
\begin{itemize}
  \item Poblaci\'on: Conjunto de elementos sobre los que se observa un car\'acter com\'un. Se representa con la letra $N$ (tama\~no poblacional).
  \item Muestra: Conjunto de unidades extra\'idas de la poblaci\'on. Cuanto m\'as significativa, mejor ser\'a la muestra. Se representa con $n$ (tama\~no muestral).
  \item Unidad de muestreo: Est\'a formada por uno o m\'as elementos de la poblaci\'on. El total de unidades de muestreo constituye la poblaci\'on; son disjuntas entre s\'i.
  \item Par\'ametro: Resumen num\'erico de una variable de la poblaci\'on. Par\'ametros habituales: media poblacional $\mu$, total poblacional $T$ (p.ej.\ $T=N\mu$), proporci\'on $p$.
  \item Estimador: Un estimador $\hat{\theta}$ de un par\'ametro $\theta$ es un \emph{estad\'istico} usado para conocer el par\'ametro desconocido.
  \item Estad\'istico: Funci\'on de los valores muestrales; es una variable aleatoria cuya distribuci\'on se denomina \emph{distribuci\'on muestral del estad\'istico}.
  \item Estimaci\'on: A partir de la muestra se extrapola el resultado a la poblaci\'on. Puede ser \emph{puntual} o por \emph{intervalo de confianza}.
  \item Prueba de Hip\'otesis: Determina si, con datos muestrales, es aceptable que una caracter\'istica o par\'ametro poblacional tome cierto valor o pertenezca a un conjunto de valores.
  \item Intervalos de confianza: Proporci\'on de veces que acertar\'iamos al afirmar que el par\'ametro $\theta$ est\'a dentro del intervalo al seleccionar muchas muestras.
\end{itemize}


%____________________________________________________________
\subsubsection*{Estad\'istico y distribuci\'on muestral}
%____________________________________________________________
El objetivo de la inferencia es generalizar resultados de la muestra a la poblaci\'on. Interesa estudiar la distribuci\'on de ciertas funciones de la muestra (estad\'isticos muestrales).

Sea $x_1,\ldots,x_n$ una muestra aleatoria simple (m.a.s.) de la v.a.\ $X$ con funci\'on de distribuci\'on $F_0$. Un \emph{estad\'istico} $T$ es cualquier funci\'on de la muestra que no contiene cantidades desconocidas.

Los estad\'isticos m\'as usuales (para un car\'acter cuantitativo) y sus distribuciones asociadas:
\begin{align}
\text{Media muestral:}\quad & \bar X=\frac{1}{n}\sum_{i=1}^n X_i, 
& \bar X \sim \mathcal{N}\!\left(\mu,\frac{\sigma^2}{n}\right). \label{eq:mean}\\[4pt]
\text{Cuasivarianza:}\quad & s^2=\frac{1}{n-1}\sum_{i=1}^n (X_i-\bar X)^2,
& \frac{(n-1)s^2}{\sigma^2}\sim \chi^2_{\,n-1}. \label{eq:var}\\[4pt]
\text{Total muestral:}\quad & t=\sum_{i=1}^n X_i, 
& t \sim \mathcal{N}(n\mu,\,n\sigma^2). \label{eq:total}
\end{align}
% Nota: en \eqref{eq:mean} y \eqref{eq:total} la normalidad exacta requiere normalidad poblacional (o n grande, por CLT).
%____________________________________________________________
\subsection{Estimaci\'on puntual}
%____________________________________________________________
Un estimador de un par\'ametro poblacional es una funci\'on de los datos muestrales. Por ejemplo, para estimar la talla media de un grupo, se extrae una muestra y se usa la media muestral como estimaci\'on puntual. Sea $X_1,\ldots,X_n$ una m.a.s. Decimos que $\hat{\theta}$ es estimador de $\theta$ si el estad\'istico empleado para conocer $\theta$ es $\hat{\theta}$.
%____________________________________________________________
\subsubsection*{Propiedades deseables de un estimador}
%____________________________________________________________

\begin{itemize}
  \item \textbf{Insesgadez:} $\,\mathbb{E}[\hat{\theta}]=\theta$. Si no coincide, es sesgado.
  \item \textbf{Eficiencia:} Dados $\hat{\theta}_1$ y $\hat{\theta}_2$ para $\theta$, $\hat{\theta}_1$ es m\'as eficiente si $\operatorname{Var}(\hat{\theta}_1)<\operatorname{Var}(\hat{\theta}_2)$.
  \item \textbf{Suficiencia:} Usa toda la informaci\'on de la muestra relativa a $\theta$.
  \item \textbf{Consistencia:} $\hat{\theta}_n \xrightarrow{p} \theta$, es decir, $\lim_{n\to\infty} \mathbb{P}\big(|\hat{\theta}_n-\theta|<\varepsilon\big)=1$ para todo $\varepsilon>0$.
\end{itemize}
%____________________________________________________________
\subsubsection*{M\'etodos para obtener estimadores puntuales}
%____________________________________________________________

\begin{itemize}
  \item \textbf{M\'etodo de los momentos:} iguala momentos poblacionales a muestrales (suelen ser consistentes).
  \item \textbf{M\'inimos cuadrados:} minimiza una funci\'on de p\'erdida (p.ej.\ suma de cuadrados de residuos).
  \item \textbf{M\'axima verosimilitud:} elige el valor del par\'ametro que hace m\'as veros\'imil la muestra (suele dar estimadores consistentes y eficientes).
\end{itemize}

La probabilidad de que $\bar X$ sea \emph{exactamente} igual a $\mu$ es cero ($\mathbb{P}[\bar X=\mu]=0$) en variables continuas.
%____________________________________________________________
\subsection{Estimaci\'on por intervalos de confianza}
%____________________________________________________________
Un intervalo de confianza est\'a determinado por dos valores dentro de los cuales afirmamos que est\'a el verdadero par\'ametro con cierta probabilidad (nivel de confianza $1-\alpha$).  \textbf{Variabilidad del par\'ametro:} si no se conoce $\sigma$, puede aproximarse con datos previos o un estudio piloto.\textbf{Error de la estimaci\'on:} amplitud del intervalo (precisi\'on). A mayor precisi\'on, intervalo m\'as estrecho y mayor tama\~no muestral. Denotemos $E=U-L$. \textbf{Nivel de confianza $(1-\alpha)$:} probabilidad de que el intervalo contenga al par\'ametro (t\'ipicamente $95\%$ o $99\%$). \textbf{Nivel de significaci\'on $\alpha$:} $\alpha=1-(1-\alpha)$. Para $95\%$, $\alpha=0{,}05$. \textbf{Valor cr\'itico:} para normal est\'andar $Z$, $z_{\alpha/2}$ satisface $\mathbb{P}(|Z|\le z_{\alpha/2})=1-\alpha$ (p.ej.\ si $\alpha=0{,}05$, $z_{0.025}\approx 1{,}96$).

\paragraph{Unilaterales y bilaterales.}
\begin{align*}
\text{Unilateral:}& \quad \mathbb{P}(Z \le z_{\alpha}) = 1-\alpha \quad \text{o} \quad \mathbb{P}(Z \ge z_{1-\alpha}) = 1-\alpha.\\
\text{Bilateral:}& \quad \mathbb{P}\big(-z_{\alpha/2} \le Z \le z_{\alpha/2}\big)=1-\alpha.
\end{align*}

%____________________________________________________________
\subsection{Ejercicios}
%____________________________________________________________

\begin{Ejer}
Indica si cada variable es: \textbf{Nominal, Ordinal, Discreta, Continua, Binaria, Polit\'omica, Intervalo o Raz\'on}.

\begin{enumerate}
  \item Nivel de satisfacci\'on: muy insatisfecho, insatisfecho, neutral, satisfecho, muy satisfecho.
  \item N\'umero de goles en un partido.
  \item Color de ojos: azul, verde, caf\'e, negro, gris.
  \item Ingreso mensual en pesos.
  \item Edad en a\~nos.
  \item Clasificaci\'on de hoteles: 1, 2, 3, 4, 5 estrellas.
  \item Licencia de conducir: s\'i / no.
  \item Nacionalidad: mexicana, argentina, chilena, espa\~nola, colombiana.
  \item Distancia recorrida en kil\'ometros.
  \item N\'umero de hijos en una familia.
  \end{enumerate}
\end{Ejer}  
  
  
  \begin{Ejer}
Indica si cada variable es: \textbf{Nominal, Ordinal, Discreta, Continua, Binaria, Polit\'omica, Intervalo o Raz\'on}.

\begin{enumerate}
  \item Peso corporal en kilogramos.
  \item Temperatura corporal en grados Celsius.
  \item Tipo de sangre: A, B, AB, O, RH-positivo.
  \item Marca de celular: Apple, Samsung, Huawei, Xiaomi, Motorola.
  \item N\'umero de estudiantes en un sal\'on.
  \item Hora del d\'ia (1:00, 2:00, ?, 12:00).
  \item Grado de dolor: sin dolor, leve, moderado, severo, insoportable.
  \item Examen final: aprobado / reprobado.
  \item Rango militar: soldado, cabo, sargento, teniente, capit\'an.
  \item Tiempo de reacci\'on en segundos.
\end{enumerate}
\end{Ejer}  
  
  
  \begin{Ejer}
Indica si cada variable es: \textbf{Nominal, Ordinal, Discreta, Continua, Binaria, Polit\'omica, Intervalo o Raz\'on}.

\begin{enumerate}
  \item Estado civil: soltero, casado, divorciado, viudo, uni\'on libre.
  \item Fumador: s\'i / no.
  \item Ocupaci\'on: estudiante, empleado, desempleado, jubilado, empresario.
  \item Estatura en metros.
  \item Nivel educativo: primaria, secundaria, preparatoria, licenciatura, posgrado.
\end{enumerate}
\end{Ejer}


\begin{Ejer}
Considera los siguientes tipos de variables Nominales:
  \begin{enumerate}
    \item Sexo: NA, NA, NA, Femenino, NA, Femenino, Femenino, Femenino, NA, Masculino, Femenino, Femenino, Masculino, Femenino, NA, Masculino, NA, NA, Masculino, Masculino, Masculino, Masculino, NA, Femenino, NA.
    \item Nacionalidad: Boliviana, Peruana, Venezolana, Cubana, Mexicana, Colombiana, Colombiana, Brasile\~na, Colombiana, Venezolana, Chilena, Argentina, Mexicana, Mexicana, Argentina, Mexicana, Argentina, Brasile\~na, Mexicana, Uruguaya, Argentina, Argentina, Colombiana, Cubana, Boliviana, Peruana, Boliviana, Boliviana, Peruana, Uruguaya, Chilena, Uruguaya, Venezolana, Uruguaya, Argentina, Venezolana, Uruguaya, Cubana, Venezolana, Cubana, Chilena, Argentina, Peruana, Boliviana, Cubana, Venezolana, Colombiana, Mexicana, Uruguaya, Argentina .
    \item Color de ojos: Gris, Azul, Ambar, Negro, Avellana, Caf\'e, Avellana, Azul, Verde, Ambar, Avellana, Caf\'e, Caf\'e, Azul, Verde, Azul, Avellana, Verde, Verde, Verde, Gris, Negro, Avellana, Negro, Gris, Negro, Avellana, Azul, Ambar, Verde.
    \item Marca de celular: Apple, Samsung, LG, Huawei, OnePlus, Sony, Apple, Samsung, Huawei, Huawei, Nokia, OnePlus, Motorola, Xiaomi, Samsung, Apple, Apple, Motorola, Samsung, Samsung, Huawei, Motorola, Nokia, OnePlus, Huawei, Huawei, Apple, LG, Apple, Xiaomi, LG, OnePlus, OnePlus, LG, Sony, Samsung, Apple, Xiaomi, Oppo, Motorola, LG, Samsung, Motorola, Samsung, Nokia, OnePlus, OnePlus, Oppo, Sony, Nokia, Huawei, LG, Sony, Xiaomi, LG, Huawei, LG, Nokia, Xiaomi, Sony, OnePlus, LG, Xiaomi, OnePlus, Oppo, Nokia, Huawei, Xiaomi, Oppo, Oppo .
    \item Estado civil: Viudo, Viudo, Union libre, Soltero, Viudo, Union libre, Divorciado, Casado, Union libre, Viudo, Viudo, Soltero, Divorciado, Divorciado, Divorciado, Union libre, Casado, Union libre, Casado, Soltero, Viudo, Casado, Casado, Divorciado, Union libre, Soltero, Divorciado, Divorciado, Casado, Union libre, Soltero, Viudo, Viudo, Soltero, Divorciado, Viudo, Viudo, Soltero, Union libre, Divorciado.
  \end{enumerate}
\end{Ejer}


\begin{Ejer} \textbf{Ordinales}
  \begin{enumerate}
    \item Nivel educativo: posdoctorado, preparatoria, preparatoria, doctorado, primaria, doctorado, preparatoria, secundaria, preparatoria, preparatoria, primaria, doctorado, secundaria, secundaria, secundaria, licenciatura, doctorado, primaria, maestria, posdoctorado, maestria, preparatoria, posdoctorado, doctorado, preparatoria, doctorado, maestria, primaria, licenciatura, secundaria, maestria, posdoctorado, doctorado, licenciatura, doctorado, posdoctorado, maestria, primaria, secundaria, preparatoria.
    \item Nivel de satisfacci\'on: satisfecho, muy satisfecho, muy insatisfecho, neutral, muy satisfecho, muy insatisfecho, muy satisfecho, insatisfecho, muy insatisfecho, insatisfecho, muy satisfecho, insatisfecho, muy satisfecho, muy satisfecho, mas o menos insatisfecho, insatisfecho, satisfecho, muy satisfecho, neutral, mas o menos insatisfecho, muy insatisfecho, mas o menos satisfecho, insatisfecho, neutral, mas o menos satisfecho, satisfecho, mas o menos satisfecho, satisfecho, neutral, satisfecho, muy insatisfecho, muy insatisfecho, neutral, satisfecho, insatisfecho, mas o menos satisfecho, mas o menos insatisfecho, muy insatisfecho, neutral, neutral, satisfecho, neutral, insatisfecho, mas o menos insatisfecho, mas o menos insatisfecho, insatisfecho, mas o menos satisfecho, satisfecho, neutral, insatisfecho, insatisfecho, muy insatisfecho, muy satisfecho, mas o menos satisfecho, mas o menos satisfecho.
    \item Grado de dolor: severo, leve, leve, severo, leve, leve, leve, moderado, moderado, moderado, moderado, severo, severo, moderado, leve, severo, leve, severo, severo, severo, leve, moderado, severo, severo, leve, leve, leve, severo, severo, leve .
    \item Rese\~ nas de hoteles: 4 estrellas, 5 estrellas, 1.5 estrellas, 4.5 estrellas, 5 estrellas, 4.5 estrellas, 2.5 estrellas, 4 estrellas, 4.5 estrellas, 1.5 estrellas, 3.5 estrellas, 1 estrella, 3.5 estrellas, 3 estrellas, 5 estrellas, 1.5 estrellas, 2 estrellas, 2 estrellas, 1.5 estrellas, 4.5 estrellas, 4.5 estrellas, 1.5 estrellas, 3.5 estrellas, 4 estrellas, 1 estrella, 4 estrellas, 4.5 estrellas, 3.5 estrellas, 1 estrella, 4 estrellas, 4 estrellas, 2 estrellas, 1 estrella, 4.5 estrellas, 1 estrella.
    \item Rangos: Principiante, Principiante, Aprendiz, Intermedio, Experto, Intermedio, Experto, Aprendiz, Principiante, Avanzado, Intermedio, Experto, Experto, Experto, Avanzado, Intermedio, Intermedio, Principiante, Principiante, Principiante, Principiante, Principiante, Intermedio, Aprendiz, Aprendiz, Aprendiz, Aprendiz, Intermedio, Aprendiz, Principiante, Intermedio, Avanzado, Avanzado, Intermedio, Avanzado, Principiante, Intermedio, Avanzado, Intermedio, Avanzado.
  \end{enumerate}
\end{Ejer}


\begin{Ejer}  \textbf{Cuantitativas Discretas}
  \begin{enumerate}
    \item N\'umero de hijos en una familia:2, 2, 4, 4, 1, 3, 1, 3, 3, 3, 5, 2, 1, 5, 3, 3, 3, 3, 1, 3, 3, 2, 5, 5, 2, 5, 2, 3, 1, 3 .
    \item N\'umero de estudiantes en un sal\'on: 35, 37, 13, 6, 40, 42, 11, 24, 24, 33, 8, 34, 21, 23, 34, 24, 45, 39, 38, 35, 9, 39, 38, 45, 7, 35, 22, 28, 38, 6, 5, 45, 36, 18, 37, 45, 13, 31, 23, 43.
    \item N\'umero de autos por familia:2, 1, 0, 0, 1, 1, 0, 0, 3, 3, 1, 0, 1, 1, 1, 0, 1, 1, 3, 2, 1, 2, 1, 0, 1, 1, 2, 1, 2, 1, 1, 2, 1, 1, 0, 1, 0, 1, 0, 0, 1, 1, 1, 2, 0, 0, 1, 0, 1, 2, 2, 0, 0, 1, 1, 0, 1, 0, 1, 0, 1, 1, 2, 2, 1, 2, 1, 0, 0, 0, 0, 0, 3, 3, 1.
    \item Goles en un partido:0, 1, 1, 0, 0, 1, 2, 0, 4, 0, 0, 0, 3, 0, 2, 3, 1, 1, 0, 1, 3, 0, 0, 4, 0, 1, 0, 4, 3, 3, 1, 2, 0, 0, 4, 4, 1, 2, 0, 0, 1, 0, 0, 4, 1, 3, 1, 0, 1, 0, 0, 3, 0, 0, 0, 0, 0, 0, 3, 4, 3, 1, 1, 0, 0, 3, 0, 1, 1, 0, 1, 0, 1, 1, 0, 3, 0, 1, 3, 0, 0, 3, 1, 3, 1, 3, 0, 1, 2, 0, 1, 1, 3, 0, 1, 0, 0, 1, 1, 2 .
    \item Accidentes en una vialidad: 22, 24, 35, 8, 12, 33, 17, 31, 24, 12, 6, 14, 26, 8, 41, 3, 15, 7, 3, 23, 10, 17, 3, 7, 1, 11, 19, 4, 33, 7, 10, 2, 2, 19, 1, 15, 7, 16, 13, 3, 26, 4, 16, 18, 27, 18, 17, 32, 25, 22, 34, 40, 47, 35, 19, 27, 12, 44, 37, 27, 40, 18, 28, 16, 31, 37, 32, 44, 29, 35, 25, 28, 27, 31, 33, 21, 41, 20, 39, 28, 22, 34, 27, 28, 37, 26, 29, 37, 38, 26, 38, 53, 39, 41, 35, 57, 27, 41, 41, 72, 53, 64, 67, 46, 44, 31, 47, 56, 56, 63, 47, 58, 44, 32, 70, 53, 59, 51, 70, 44, 51, 52, 46, 44, 46, 41, 54, 63, 46, 47, 37, 60, 34, 49, 63, 42, 48, 44, 65, 63, 45, 63, 42, 58, 43, 54, 90, 55, 57, 54, 55, 46, 46, 46, 55, 23, 39, 35, 44, 58, 68, 44, 11, 23, 24, 13, 47, 49.
  \end{enumerate}
\end{Ejer}

\begin{Ejer} \textbf{Cuantitativas Continuas}
  \begin{enumerate}
    \item Estatura en metros: 1.57, 1.97, 1.85, 1.78, 1.62, 1.67, 1.92, 1.96, 1.76, 1.94, 1.85, 1.71, 1.65, 1.98, 1.56, 1.84, 1.91, 1.61, 1.8, 1.58, 1.7, 1.62, 1.76, 1.84, 1.95, 1.67, 1.62, 1.65, 1.61, 1.56, 1.63, 1.96, 1.53, 1.75, 1.9, 1.92, 1.79, 1.62, 1.79, 1.57, 1.74, 1.55, 1.65, 1.84, 1.6, 1.7, 1.9, 1.61, 1.54, 1.89, 1.84, 1.76, 1.61, 1.59, 1.81, 1.52, 1.61, 1.91, 1.65, 1.51, 1.98, 1.54, 1.87, 1.92, 1.51, 1.93, 1.78, 1.67, 1.66, 1.82, 1.56, 1.66, 1.79, 1.88, 1.78, 1.84, 1.69, 1.95, 1.99, 1.88, 1.51, 1.76, 1.9, 1.6, 1.69, 1.77, 1.87, 1.67, 1.66, 1.71, 1.64, 1.81, 1.52, 1.8, 1.71, 1.52, 2, 1.81, 1.92, 1.7 .
    \item Peso corporal en kilogramos:100.5, 91.3, 60.4, 83.2, 49.6, 103.1, 60.3, 109.3, 83.5, 63.9, 106, 50, 47, 89.7, 108.5, 78.9, 82.7, 60.7, 98.7, 85.2, 48.7, 106.7, 63.9, 84.1, 69.5, 53.3, 108.9, 91.8, 108.6, 54.5, 95.1, 90.6, 115.9, 88.5, 67.7, 115.1, 108.3, 76.8, 81.4, 102.6, 63.9, 105.9, 106.7, 76.3, 113.7, 50.3, 105.8, 81.4, 67.9, 91.3, 68.9, 93.9, 113.7, 87.7, 92.8, 76.2, 104.7, 109.7, 72.6, 81.6, 112.2, 79.8, 60.7, 95.7, 100.1, 94, 60.5, 117.1, 45.5, 112.7, 51.7, 107.8, 86.6, 90.3, 105.9, 64.7, 48, 55.4, 52.9, 58.2, 117.1, 59.6, 69.9, 96.9, 97, 66.5, 67.4, 77.2, 73.7, 113.
    \item  Nivel de contaminaci\'on en la delegaci\'on Iztapalapa en el mes de julio: 0.7, 0.4, 0.4, 0.5, 0.3, 0.3, 0.7, 0.5, 0.2, 0.2, 0.5, 0.8, 0.3, 0.9, 0.5, 0.2, 0.4, 0.4, 0.7, 0.6, 0.8, 0.3, 1, 0.7, 1.5, 0.1, 0.1, 0.9, 0.1, 0.3, 0.9, 0.3, 1.2, 1.2, 1.2, 0.6, 0.7, 0.4, 1.8, 0.2, 0.6, 0.7, 1.7, 0.4, 0.9, 0.2, 0.8, 1, 0.7, 0.6, 0.7, 0.3, 0.5, 0.2, 1.2, 0.5, 0.8, 0.2, 0.4, 0.3, 0.4, 0.5, 0.5, 1.4, 0.4.
    \item Temperatura en grados Celsius:20.1, 23.1, 30.6, 32.3, 31.7, 5, 21.1, 27.4, 21.3, 29, 19.8, 15.3, 28.7, 25.1, 14.3, 23.3, 28.7, 17.7, 30.3, 26.7, 26.2, 11.4, 25.5, 15, 23.8, 23.9, 16.2, 26, 12.7, 32.8, 19.2, 37.2, 10.3, 25, 14.9, 30.9, 19, 23.64, 24.1, 13.1, 17.6, 31.9, 25.7, 13.6, 21.6, 14.4, 29.5, 7.75, 33.7, 35.2 .
    \item Tiempos de traslado en horas:1.5, 1.3, 1.1, 0.6, 0.9, 1.4, 0.8, 1.7, 1.7, 1.6, 1.7, 1.3, 1.5, 0.5, 1.4, 0.6, 1.5, 1.9, 0.7, 1.9, 2, 1.6, 0.7, 1.6, 0.7, 0.5, 0.6, 1.6, 1.4, 1.9, 1.9, 1, 1.7, 1.7, 1.8, 1.6, 1.4, 1.4, 0.7, 0.8, 1.5, 0.8, 1.9, 1.2, 1.9, 1.8, 1.7, 1.5, 1.5, 0.7, 1.8, 1.3, 1.8, 0.6, 1.5, 1.4, 1.3, 1.3, 0.8, 1.4, 0.6, 1.1, 1.1, 0.6, 1.7, 0.5, 2, 1.7, 0.7, 1, 0.6, 1.8, 1.1, 1.6, 1, 1.1, 1.8, 0.8, 1.1, 1.7, 1.4, 1.4, 1.2, 1.5, 1.6, 1.6, 1.5, 0.5, 1.4, 1.1, 1, 0.9, 0.6, 1.7, 1.6, 1.2, 0.6, 1.2, 1.2, 1.8 
  \end{enumerate}
\end{Ejer}

\begin{Ejer} \textbf{Binarias o dicot\'omicas y Polit\'omicas}
  \begin{enumerate}
    \item Licencia de conducir: No, Si, No, Si, Si, No, Si, No, No, No, Si, Si, No, No, Si, No, No, Si, No, No, Si, Si, Si, No, No, Si, No, Si, No, No, No, No, Si, Si, No, Si, Si, No, Si, Si, No, Si, No, No, Si, No, No, No, No, Si, Si, Si, No, Si, No, No, No, No, Si, Si, No, Si, No, Si, No, No, No, Si, No, Si, No, Si, Si, No, No .
    \item Examen: Reprobado, Aprobado, Aprobado, Aprobado, Aprobado, Reprobado, Aprobado, Reprobado, Reprobado, Aprobado, Reprobado, Reprobado, Reprobado, Reprobado, Aprobado, Reprobado, Reprobado, Aprobado, Aprobado, Reprobado, Aprobado, Aprobado, Aprobado, Aprobado, Reprobado, Aprobado, Reprobado, Aprobado, Reprobado, Reprobado, Aprobado, Reprobado, Aprobado, Reprobado, Aprobado, Reprobado, Reprobado, Aprobado, Reprobado, Reprobado, Reprobado, Aprobado, Aprobado, Aprobado, Aprobado, Reprobado, Reprobado, Reprobado, Aprobado, Reprobado.
    \item COVID: Negativo, Positivo, Negativo, Negativo, Negativo, Positivo, Positivo, Negativo, Negativo, Positivo, Negativo, Positivo, Positivo, Negativo, Negativo, Negativo, Negativo, Negativo, Negativo, Negativo, Positivo, Positivo, Negativo, Negativo, Positivo, Negativo, Positivo, Negativo, Negativo, Positivo, Positivo, Negativo, Positivo, Positivo, Negativo, Positivo, Negativo, Negativo, Negativo, Negativo, Positivo, Negativo, Positivo, Negativo, Negativo, Negativo, Negativo, Negativo, Negativo, Negativo.

    \item Tipo de sangre: A-, AB-, AB+, B+, O-, O+, B+, A-, O-, B-, A-, A-, O+, A+, O+, AB+, B+, AB+, AB-, B-, O-, O+, B+, O+, O+, O+, A+, O-, AB-, A-, B+, O-, B-, A-, A-, AB-, B+, AB-, A-, B+, A-, AB-, O-, AB+, B+, O+, B-, O-, A-, AB-, O+, AB+, AB+, B+, O-, AB+, AB+, AB-, A-, O-, AB+, AB+, A+, A+, O+, B+, AB-, O-, A-, A+, A+, B-, B-, B-, A+, A-, AB-, A-, A-, AB+.
    \item Ocupaci\'on: Desempleado, Estudiante, Empleado, Empleado, Estudiante, Empleado, Empleado, Empleado, Jubilado, Jubilado, Jubilado, Jubilado, Jubilado, Estudiante, Desempleado, Desempleado, Jubilado, Empleado, Estudiante, Estudiante, Empleado, Desempleado, Estudiante, Desempleado, Jubilado, Empleado, Jubilado, Desempleado, Desempleado, Desempleado.
  \end{enumerate}
\end{Ejer}


%____________________________________________________________
\subsection*{Refuerzo}
%____________________________________________________________

La estad\'istica es un conjunto de procedimientos para reunir, clasificar, codificar, procesar, analizar y resumir informaci\'on num\'erica adquirida sistem\'aticamente. Permite hacer inferencias a partir de una muestra para extrapolarlas a una poblaci\'on. Aunque normalmente se asocia a muchos c\'alculos y operaciones aritm\'eticas, y aunque las matem\'aticas est\'an involucradas, en su mayor parte sus fundamentos y uso apropiado pueden dominarse sin hacer referencia a habilidades matem\'aticas avanzadas.  De esta manera la estad\'istica se relaciona con el m\'etodo cient\'ifico complement\'andolo como herramienta de an\'alisis y, aunque la investigaci\'on cient\'ifica no requiere necesariamente de la estad\'istica, \'esta valida muchos de los resultados cuantitativos derivados de la investigaci\'on. La obtenci\'on del conocimiento debe hacerse de manera sistem\'atica por lo que deben planearse todos los pasos que llevan desde el planteamiento de un problema, pasando por la elaboraci\'on de hip\'otesis y la manera en que van a ser probadas; la selecci\'on de sujetos (muestreo), los escenarios, los instrumentos que se utilizar\'an para obtener los datos, definir el procedimiento que se seguir\'a para esto \'ultimo, los controles que se deben hacer para asegurar que las intervenciones son las causas m\'as probables de los cambios esperados (dise\~no); hasta la elecci\'on del plan de an\'alisis id\'oneo para el tipo de datos que se est\'an obteniendo, es aqu\'i donde la estad\'istica entra en el estudio, aunque pueden existir otras herramientas de an\'alisis si se est\'a haciendo una investigaci\'on de corte cualitativo.\medskip

Una buena planeaci\'on permitir\'a que los resultados puedan ser reproducidos, mediante la comprobaci\'on emp\'irica, por cualquier investigador interesado en refutar o comprobar las conclusiones que se hagan del estudio. De esta manera tambi\'en se lograr\'a la predicci\'on de los fen\'omenos que se est\'an estudiando, ayudando a conocer y prevenir los problemas sociales e individuales que forman parte del objeto de estudio de la psicolog\'ia. El tratamiento de los datos de la investigaci\'on cient\'ifica tiene varias etapas:
\begin{itemize}
\item En la etapa de recolecci\'on de datos del m\'etodo cient\'ifico, se define a la poblaci\'on de inter\'es y se selecciona una muestra o conjunto de personas representativas de la misma, se realizan experimentos o se emplean instrumentos ya existentes o de nueva creaci\'on, para medir los atributos de inter\'es necesarios para responder a las preguntas de investigaci\'on.Durante lo que es llamado trabajo de campo se obtienen los datos en crudo, es decir las respuestas directas de los sujetos uno por uno, se codifican (se les asignan valores a las respuestas), se capturan y se verifican para ser utilizados en las siguientes etapas.

\item En la etapa de recuento, se organizan y ordenan los datos obtenidos de la muestra. Esta ser\'a descrita en la siguiente etapa utilizando la estad\'istica descriptiva, todas las investigaciones utilizan estad\'istica descriptiva, para conocer de manera organizada y resumida las caracter\'isticas de la muestra.

\item En la etapa de an\'alisis se utilizan las pruebas estad\'isticas (estad\'istica inferencial) y en la interpretaci\'on se acepta o rechaza la hip\'otesis nula.

\item En investigaci\'on, el fen\'omeno en estudio puede ser cualitativo que implicar\'ia comprenderlo y explicarlo, o cuantitativo para compararlo y hacer inferencias. Se puede decir que si se hace an\'alisis se usan m\'etodos cuantitativos y si se hace descripci\'on se usan m\'etodos cualitativos. Medici\'on Para poder emplear el m\'etodo estad\'istico en un estudio es necesario medir las variables. 
\begin{itemize}
\item Medir: es asignar valores a las propiedades de los objetos bajo ciertas reglas, esas reglas son los niveles de medici\'on.

\item Cuantificar: es asignar valores a algo tomando un patr\'on de referencia. Por ejemplo, cuantificar es ver cu\'antos hombres y cu\'antas mujeres hay.

\item Variable: es una caracter\'istica o propiedad que asume diferentes valores dentro de una poblaci\'on de inter\'es y cuya variaci\'on es susceptible de medirse. Las variables pueden clasificarse de acuerdo al tipo de valores que puede tomar como:
\begin{itemize}
\item Discretas o categ\'oricas: en las que los valores se relacionan a nombres, etiquetas o categor\'ias, no existe un significado num\'erico directo.
\item Continuas: los valores tienen un correlato num\'erico directo, son continuos y susceptibles de fraccionarse y de poder utilizarse en operaciones aritm\'eticas De acuerdo a la cantidad de valores.
\item Dicot\'omica: s\'olo tienen dos valores posibles, la caracter\'istica est\'a ausente o presente.
\item Policot\'omica: pueden tomar tres valores o m\'as, pueden tomarse matices diferentes, en grados, jerarqu\'ias o magnitudes continuas.
\end{itemize}
\end{itemize}
\item Niveles de Medici\'on
\begin{itemize}
\item Nominal Las propiedades de la medici\'on nominal son:
\begin{itemize}
\item Exhaustiva: implica a todas las opciones. 
\item A los sujetos se les asignan categor\'ias, por lo que son mutuamente excluyentes. Es decir, la variable est\'a presente o no; tiene o no una caracter\'istica
\end{itemize}
\item Ordinal: Las propiedades de la medici\'on ordinal son:
\begin{itemize}
\item El nivel ordinal posee transitividad, por lo que se tiene la capacidad de identificar que esto es mejor o mayor que aquello, en ese sentido se pueden establecer jerarqu\'ias
\item Las distancias entre un valor y otro no son iguales.
\end{itemize}
\item Intervalo
\begin{itemize}
\item  El nivel de medici\'on por intervalo requiere distancias iguales entre cada valor. Por lo general utiliza datos cuantitativos. Por ejemplo: temperatura, atributos psicol\'ogicos (CI, nivel de autoestima, pruebas de conocimientos, etc.)
\item Las unidades de calificaci\'on son equivalentes en todos los puntos de la escala. Una escala de intervalos implica: clasificaci\'on, magnitud y unidades de tama\~nos iguales.
\item Se pueden hacer operaciones aritm\'eticas
\item Cuando se le pide al sujeto que califique una situaci\'on del 0 al 10 puede tomarse como un nivel de medici\'on por intervalo, siempre y cuando se incluya el 0.
\end{itemize}
\item Raz\'on
\begin{itemize}
\item La escala empieza a partir del 0 absoluto, por lo tanto incluye s\'olo los n\'umeros por su valor en s\'i, por lo que no pueden existir los n\'umeros con signo negativo. Por ejemplo: Peso corporal en kg., edad en a\~nos, estatura en cm.
\item Convencionalmente los datos que son de nivel absoluto o de raz\'on son manejados como los datos intervalares.
\end{itemize}
\end{itemize}
\end{itemize}

%____________________________________________________________
\subsubsection*{Malos Usos de la Estad\'istica}
%____________________________________________________________

\textbf{Datos estad\'isticos inadecuados} Los datos estad\'isticos son usados como la materia prima para un estudio estad\'istico. Cuando los datos son inadecuados, la conclusi\'on extra\'ida del estudio de los datos se vuelve obviamente inv\'alida. Por ejemplo, supongamos que deseamos encontrar el ingreso familiar t\'ipico del a\~no pasado en la ciudad Y de 50,000 familias y tenemos una muestra consistente del ingreso de solamente tres familias: 1 mill\'on, 2 millones y no ingreso. Si sumamos el ingreso de las tres familias y dividimos el total por 3, obtenemos un promedio de 1 mill\'on. Entonces, extraemos una conclusi\'on basada en la muestra de que el ingreso familiar promedio durante el a\~no pasado en la ciudad fue de  1 mill\'on. Es obvio que la conclusi\'on es falsa, puesto que las cifras son extremas y el tama\~no de la muestra es demasiado peque\~no; por lo tanto la muestra no es representativa. Hay muchas otras clases de datos inadecuados. Por ejemplo, algunos datos son respuestas inexactas de una encuesta, porque las preguntas usadas en la misma son vagas o enga\~nosas, algunos datos son toscas estimaciones porque no hay disponibles datos exactos o es demasiado costosa su obtenci\'on, y algunos datos son irrelevantes en un problema dado, porque el estudio estad\'istico no est\'a bien planeado.
%____________________________________________________________
\subsubsection*{Un sesgo del usuario}
%____________________________________________________________

Sesgo significa que un usuario d\'e los datos perjudicialmente de m\'as \'enfasis a los hechos, los cuales son empleados para mantener su predeterminada posici\'on u opini\'on. Hay dos clases de sesgos: conscientes e inconscientes. Ambos son comunes en el an\'alisis estad\'istico. Hay numerosos ejemplos de sesgos conscientes. 
\begin{itemize}
\item Un anunciante frecuentemente usa la estad\'istica para probar que su producto es muy superior al producto de su competidor. 
'.item Un pol\'itico prefiere usar la estad\'istica para sostener su punto de vista. 
\item Gerentes y l\'ideres de trabajadores pueden simult\'aneamente situar sus respectivas cifras estad\'isticas sobre la misma tabla de trato para mostrar que sus rechazos o peticiones son justificadas. 
\end{itemize}

Es casi imposible que un sesgo inconsciente est\'e completamente ausente en un trabajo estad\'istico. En lo que respecta al ser humano, es dif\'icil obtener una actitud completamente objetiva al abordar un problema.

%____________________________________________________________
\subsubsection*{ Supuestos falsos}
%____________________________________________________________

Es muy frecuente que un an\'alisis estad\'istico contemple supuestos. Un investigador debe ser muy cuidadoso en este hecho, para evitar que \'estos sean falsos. Los supuestos falsos pueden ser originados por:

\begin{itemize}
\item Quien usa los datos
\item Quien est\'a tratando de confundir (con intencionalidad)
\item Ignorancia
\item Descuido.
\end{itemize}
